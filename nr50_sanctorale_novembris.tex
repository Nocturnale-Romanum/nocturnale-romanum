% !TEX TS-program = lualatex
% !TEX encoding = UTF-8

\documentclass[nocturnale-romanum.tex]{subfiles}

\ifcsname preamble@file\endcsname
  \setcounter{page}{\getpagerefnumber{M-nr50_sanctorale_novembris}}
\fi

\begin{document}
\feast{1100}{Festa Novembris}{Proprium Sanctorum}{Festa Novembris}{1}{}{}{}{}{}{}

\feast{1101}{In Festo Omnium Sanctorum}
	{Proprium Sanctorum}{Festa Novembris}{1}{1 Novembris}
	{Duplex I. classis}{I. classis}{Omnium Sanctorum}
	{cum Octava communi --- Omnia ut propria assignantur.}
	{}
\gscore{1101I}{I}{}{Regem regum Dominum}
\gscore{1101H}{H}{}{Placare Christe servulis}
\nocturn{1}
\gscore{1101N1A1}{A}{1}{Novit Dominus}
\rubric{Psalmus TODO}
\gscore{1101N1A2}{A}{2}{Mirificavit Dominus}
\rubric{Psalmus TODO}
\gscore{1101N1A3}{A}{3}{Admirabile est nomen tuum Domine qui gloria}
\rubric{Psalmus TODO}
\versiculus{Lætámini in Dómino et exsultáte, justi.}{Et gloriámini, omnes recti corde.}
\gscore{1101N1R1}{R}{1}{Vidi Dominum sedentem}
\respref{2}{0815N3R2}{}
\respref{3}{0508N1R3}{}
\nocturn{2}
\gscore{1101N2A1}{A}{4}{Domine qui operati sunt}
\rubric{Psalmus TODO}
\gscore{1101N2A2}{A}{5}{Haec est generatio}
\rubric{Psalmus TODO}
\gscore{1101N2A3}{A}{6}{Laetamini in Domino}
\rubric{Psalmus TODO}
\versiculus{Exsúltent justi in conspéctu Dei.}{Et delecténtur in lætítia.}
\respref{4}{0624N3R1}{}
\respref{5}{APEXN3R1}{}
\respref{6}{PMEXN3R2a}{}
\nocturn{3}
\gscore{1101N3A1}{A}{7}{Timete Dominum}
\rubric{Psalmus TODO}
\gscore{1101N3A2}{A}{8}{Domine spes Sanctorum}
\rubric{Psalmus TODO}
\gscore{1101N3A3}{A}{9}{Qui diligitis Dominum laetamini}
\rubric{Psalmus TODO}
\versiculus{Justi autem in perpétuum vivent.}{Et apud Dóminum est merces eórum.}
\respref{7}{COPON3R2}{}
\respref{8}{MUVXN3R2}{}
\tedeumrubric

\feast{1102}{In Commemoratione\\Omnium Fidelium Defunctorum}
	{Proprium Sanctorum}{Festa Novembris}{2}{2 Novembris}
	{Duplex}{I. classis}{Defunctorum!Commemoratio}
	{Omnia dicuntur ut in Officio Defunctorum, pag.\ \pageref{M-ODEF}}
	{}

\feast{1103}{Die III infra Octavam Omnium Sanctorum}
	{Proprium Sanctorum}{Festa Novembris}{2}{3 Novembris}
	{Semiduplex}{(Omittitur)}{}
	{Antiphonæ et Psalmi ad omnes Horas et Versus Nocturnorum de ocurenti die, ut in Psalterio; reliqua ut in Festo præter Lectiones cum suis Responsoriis, quæ in \Rnum{1} Nocturno dicunter de Scriptura occurenti, in \Rnum{2} et \Rnum{3} ut in Festo.}
	{}

\feast{1104}{S. Caroli Episcopi et Confessoris}
	{Proprium Sanctorum}{Festa Novembris}{2}{4 Novembris}
	{Duplex m.t.v.}{III. classis}{Caroli}
	{Commune Confessoris Pontificis, pag.\ \pageref{M-COPO}.}
	{}

\feast{1105}{Diebus V-VII infra Octavam Omnium Sanctorum}
	{Proprium Sanctorum}{Festa Novembris}{2}{5-8 Novembris}
	{Semiduplicia}{(Omittitur)}{}
	{Omnia ut in die \Rnum{3} infra Octavam.}
	{}

\feast{1108}{In Octava Omnium Sanctorum}
	{Proprium Sanctorum}{Festa Novembris}{2}{8 Novembris}
	{Duplex majus}{(Omittitur)}{Omnium Sanctorum!Octava}
	{Omnia ut in die \Rnum{3} infra Octavam.}
	{}

\feast{1109}{In Dedicatione Archibasilicæ Sanctissimi Salvatoris}
	{Proprium Sanctorum}{Festa Novembris}{2}{9 Novembris}
	{Duplex II. classis}{II. classis}{Jesu Christi, Domini nostri!Dedicatio Archibasilicæ S. S.}
	{Omnia ut in Communi Dedicationis Ecclesiæ, pag.\ \pageref{M-CDED}.}
	{}

\feast{1110}{S. Andreæ Avellini Confessoris}
	{Proprium Sanctorum}{Festa Novembris}{2}{10 Novembris}
	{Duplex}{III. classis}{Andreæ Avellini}
	{Commune Confessoris non Pontificis, pag.\ \pageref{M-CONP}.}
	{}

\feast{1111}{S. Martini Episcopi et Confessoris}
	{Proprium Sanctorum}{Festa Novembris}{2}{11 Novembris}
	{Duplex}{III. classis}{Martini Episcopi}
	{Omnia ut hic propria notantur. Psalmi trium Nocturnorum ut in Communi unius Martyris, pag.\ \pageref{M-UMEX}.}
	{}
\gscore{1111I}{I}{}{Laudemus Deum nostrum... Martini}
\rubric{Hymnus \scorename{COPOF1H} pag.\ \pageref{M-COPOF1H} in tono primo pro Dominicis.}
\nocturn{1}
\gscore{1111N1A1}{A}{1}{Martinus adhuc}
\gscore{1111N1A2}{A}{2}{Sanctae Trinitatis fidem Martinus}
\gscore{1111N1A3}{A}{3}{Ego signo crucis non clypeo}
\versiculus{Amávit eum Dóminus, et ornávit eum.}{Stolam glóriæ índuit eum.}
\gscore{1111N1R1}{R}{1}{Hic est Martinus electus}
\gscore{1111N1R2}{R}{2}{Domine si adhuc populo}
\gscore{1111N1R3}{R}{3}{O beatum virum Martinum}
\nocturn{2}
\gscore{1111N2A1}{A}{4}{Confido in Domino}
\gscore{1111N2A2}{A}{5}{Tetradius cognita}
\gscore{1111N2A3}{A}{6}{O ineffabilem virum}
\versiculus{Elégit eum Dóminus sacerdótem sibi.}{Ad sacrificándum ei hóstiam laudis.}
\gscore{1111N2R1}{R}{4}{Oculis ac manibus}
\gscore{1111N2R2}{R}{5}{Beatus Martinus obitum}
\gscore{1111N2R3}{R}{6}{Dixerunt discipuli... Martinum}
\nocturn{3}
\gscore{1111N3A1}{A}{7}{Dominus Jesus Christus non se inquit}
\gscore{1111N3A2}{A}{8}{Sacerdos Dei Martine aperti}
\gscore{1111N3A3}{A}{9}{Sacerdos Dei Martine pastor}
\versiculus{Tu es sacérdos in ætérnum.}{Secúndum órdinem Melchísedech.}
\gscore{1111N3R1}{R}{7}{O beatum virum in cujus}
\gscore{1111N3R2}{R}{8}{Martinus Abrahae sinu}
\tedeumrubric

\feast{1112}{S. Martini Papæ et Martyris}
	{Proprium Sanctorum}{Festa Novembris}{2}{12 Novembris}
	{Semiduplex}{III. classis}{Martini Papæ}
	{Commune unius Martyris., pag.\ \pageref{M-UMEX}.}
	{}

\feast{1113}{S. Didaci Confessoris}
	{Proprium Sanctorum}{Festa Novembris}{2}{13 Novembris}
	{Semiduplex m.t.v.}{III. classis}{Didaci}
	{ommune Confessoris non Pontificis, pag.\ \pageref{M-CONP}.}
	{}

\feast{1114}{S. Josaphat Episcopi et Martyris}
	{Proprium Sanctorum}{Festa Novembris}{2}{14 Novembris}
	{Duplex}{III. classis}{Josaphat}
	{Commune unius Martyris, pag.\ \pageref{M-UMEX}.}
	{}

\feast{1115}{S. Alberti Magni Episcopi\\Confessoris et Ecclesiæ Doctoris}
	{Proprium Sanctorum}{Festa Novembris}{2}{15 Novembris}
	{Duplex}{III. classis}{Alberti Magni}
	{Commune Doctorum, pag.\ \pageref{M-CODO}.}
	{}

\feast{1116}{S. Gertrudis Virginis}
	{Proprium Sanctorum}{Festa Novembris}{2}{16 Novembris}
	{Duplex}{III. classis}{Gertrudis}
	{Commune aut Virginum aut non Virginum, pag.\ \pageref{M-MU}.}
	{}

\feast{1117}{S. Gregorii Thaumaturgi Episcopi et Confessoris}
	{Proprium Sanctorum}{Festa Novembris}{2}{17 Novembris}
	{Semiduplex}{III. classis}{Gregorii Thaumaturgi}
	{Commune Confessoris Pontificis, pag.\ \pageref{M-COPO}.}
	{}

\feast{1118}{In Dedicatione Basilicarum\\Ss. Apostolorum Petri et Pauli}
	{Proprium Sanctorum}{Festa Novembris}{2}{18 Novembris}
	{Duplex majus}{III. classis}{Petri et Pauli!Dedicatio}
	{Omnia de Communi Dedicationis Ecclesiæ, pag.\ \pageref{M-CDED}.}
	{}

\feast{1119}{S. Elisabeth Viduæ}
	{Proprium Sanctorum}{Festa Novembris}{2}{19 Novembris}
	{Duplex}{III. classis}{Elisabeth}
	{Commune aut Virginum aut non Virginum, pag.\ \pageref{M-MU}.}
	{}

\feast{1120}{S. Felicis de Valois Confessoris}
	{Proprium Sanctorum}{Festa Novembris}{2}{20 Novembris}
	{Duplex m.t.v.}{III. classis}{Felicis de Valois}
	{Commune Confessoris non Pontificis, pag.\ \pageref{M-CONP}.}
	{}

\feast{1121}{In Præsentatione Beatæ Mariæ Virginis}
	{Proprium Sanctorum}{Festa Novembris}{2}{21 Novembris}
	{Duplex majus}{III. classis}{Mariæ|Præsentatio}
	{Omnia de Communi Festorum B.M.V., pag.\ \pageref{M-CBMV}.}
	{}

\feast{1122}{S. Cæciliæ Virginis et Martyris}
	{Proprium Sanctorum}{Festa Novembris}{2}{22 Novembris}
	{Duplex}{III. classis}{Cæciliæ}
	{Invitatorium et Hymnus pro Martyre, de Communi Virginum, et Psalmi trium Nocturnorum ut in Communi Virginum, pag.\ \pageref{M-MU}.}
	{}
\nocturn{1}
\gscore{1122N1A1}{A}{1}{Caecilia virgo Almachium}
\gscore{1122N1A2}{A}{2}{Expansis manibus orabat}
\gscore{1122N1A3}{A}{3}{Cilicio Caecilia}
\versiculus{Spécie tua et pulchritúdine tua.}{Inténde, próspere procéde, et regna.}
\gscore{1122N1R1}{R}{1}{Cantantibus organis}
\gscore{1122N1R2}{R}{2}{O beata Caecilia}
\gscore{1122N1R3}{R}{3}{Virgo gloriosa semper Evangelium}
\nocturn{2}
\gscore{1122N2A1}{A}{4}{Domine Jesu Christe seminator casti}
\gscore{1122N2A2}{A}{5}{Beata Caecilia dixit ad}
\gscore{1122N2A3}{A}{6}{Fiat Domine cor meum et corpus}
\versiculus{Adjuvábit eam Deus vultu suo.}{Deus in médio ejus, non commovébitur.}
\gscore{1122N2R1}{R}{4}{Cilicio Caecilia}
\gscore{1122N2R2}{R}{5}{Caeciliam intra cubiculo}
\gscore{1122N2R3}{R}{6}{Domine Jesu Christe pastor}
\nocturn{3}
\gscore{1122N3A1}{A}{7}{Credimus Christum Filium Dei verum Deum}
\gscore{1122N3A2}{A}{8}{Nos scientes sanctum nomen}
\gscore{1122N3A3}{A}{9}{Tunc Valerianus perrexit ad}
\versiculus{Elégit eam Deus, et præelégit eam.}{In tabernáculo suo habitáre facit eam.}
\gscore{1122N3R1}{R}{7}{Beata Caecilia dixit Tiburtio}
\gscore{1122N3R2}{R}{8}{Caecilia me misit ad vos}
\tedeumrubric

\feast{1123}{S. Clementis Papæ et Martyris}
	{Proprium Sanctorum}{Festa Novembris}{2}{23 Novembris}
	{Duplex}{III. classis}{Clementis}
	{Commune unius Martyris, pag.\ \pageref{M-UMEX}, excepta quæ hic habentur propria.}
	{}
\nocturn{2}
\gscore{1123N2R1}{R}{4}{Orante sancto Clemente}
\gscore{1123N2R2}{R}{5}{Omnes una voce dixerunt}
\gscore{1123N2R3}{R}{6}{Dedisti Domine habitaculum}
\nocturn{3}
\respref{7}{UMEXN3R1}{}
\respref{8}{UMEXN3R2a}{}
\tedeumrubric

\feast{1124}{S. Joannis a Cruce Confessoris et Ecclesiæ Doctoris}
	{Proprium Sanctorum}{Festa Novembris}{2}{24 Novembris}
	{Duplex m.t.v.}{III. classis}{Joannis a Cruce}
	{Commune Confessoris non Pontificis, pag.\ \pageref{M-CONP}.}
	{}

\feast{1125}{S. Catharinæ Virginis et Martyris}
	{Proprium Sanctorum}{Festa Novembris}{2}{25 Novembris}
	{Duplex}{III. classis}{Catharinæ}
	{Commune aut Virginum aut non Virginum, pag.\ \pageref{M-MU}.}
	{}

\feast{1126}{S. Silvestri Abbatis}
	{Proprium Sanctorum}{Festa Novembris}{2}{26 Novembris}
	{Duplex}{III. classis}{Silvestri Abbatis}
	{Commune Abbatum, pag.\ \pageref{M-COAB}.}
	{}

\feast{1129}{In Vigilia S.\ Angreæ}
	{Proprium Sanctorum}{Festa Novembris}{2}{29 Novembris (extra Adventu)}
	{Simplex}{(Omittitur)}{Angreæ!Vigilia}
	{Officium fit de Feria, ut in Ordinario et Psalterio, et Lectiones, quæ dicuntur e Homilia in Evangelium \normaltext{Stabat Joánnes} ut in suis locis notatur, cum Responsoriis tamen de Feria currenti, ut in Proprio de Tempore. Ad Nocturnum vero in Feria \Rnum{4} tres ultimæ Antiphonæ sum suis Psalmis, et ad Laudes in qualibet Feria Antiphonæ omnes et Psalmi sumuntur de 2 loco; ad Primam additur quartus Psalmus, ut in Psalterio notatur, et ad omnes Horas dicuntur Preces feriales, ut in Ordinario.}
	{Fit commemoratio S. Saturnini Martyris, ut infra.}

\feast{1129}{S. Saturnini Martyris}
	{Proprium Sanctorum}{Festa Novembris}{2}{29 Novembris (in Adventu)}
	{(Commemoratio tantum)}{(Commemoratio tantum)}{Saturnini}
	{In Adventu de Vigilia S.\ Andreæ Apostoli nihi fit Officio, sed fit Officium de Feria cum commemoratio S. Saturnini ad Laudes tantum.}
	{Commemoratio ad Laudes tantum, etiam extra Adventu.}

\feast{1130}{S. Andreæ Apostoli}
	{Proprium Sanctorum}{Festa Novembris}{2}{30 Novembris}
	{Duplex II. classis}{II. classis}{Andreæ}
	{Omnia de Communi Apostolorum \pageref{M-APEX}, præter ea quæ hic habentur propria.}
	{}
\nocturn{1}
\gscore{1130N1A1}{A}{1}{Vidit Dominus Petrum et Andream}
\gscore{1130N1A2}{A}{2}{Venite post me dicit Dominus}
\gscore{1130N1A3}{A}{3}{Relictis retibus}
\versiculus{In omnem terram exívit sonus eórum.}{Et in fines orbis terræ verba eórum.}
\gscore{1130N1R1}{R}{1}{Cum deambularet}
\gscore{1130N1R2}{R}{2}{Mox ut vocem}
\gscore{1130N1R3}{R}{3}{Doctor bonus et amicus Dei Andreas}
\nocturn{2}
\gscore{1130N2A1}{A}{4}{Dignum sibi Dominus}
\gscore{1130N2A2}{A}{5}{Dilexit Andream}
\gscore{1130N2A3}{A}{6}{Biduo vivens}
\versiculus{Constítues eos príncipes super omnem terram.}{Mémores erunt nóminis tui, Dómine.}
\gscore{1130N2R1}{R}{4}{Homo Dei ducebatur}
\gscore{1130N2R2}{R}{5}{O Bona Crux}
\gscore{1130N2R3}{R}{6}{Expandi manus meas}
\nocturn{3}
\gscore{1130N3A1}{A}{7}{Non me permittas}
\gscore{1130N3A2}{A}{8}{Andreas vero rogabat}
\gscore{1130N3A3}{A}{9}{Accipe me ab hominibus}
\versiculus{Nimis honoráti sunt amíci tui, Deus.}{Nimis confortátus est principátus eórum.}
\gscore{1130N3R1}{R}{7}{Oravit sanctus Andreas}
\gscore{1130N3R2}{R}{8}{Videns crucem Andreas}
\tedeumrubric

\end{document}