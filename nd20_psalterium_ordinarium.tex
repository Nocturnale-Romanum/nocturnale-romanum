% !TEX TS-program = lualatex
% !TEX encoding = UTF-8

\documentclass[nocturnale-dominicis.tex]{subfiles}

\ifcsname preamble@file\endcsname
  \setcounter{page}{\getpagerefnumber{M-nd20_psalterium_ordinarium}}
\fi

\begin{document}
\feast{OR}{Ordinarium Divini Officii\\ad Matutinum}{Ordinarium}{Ordinarium}{1}{}{}{}{}{}{}
\addcontentsline{toc}{chapter}{Ordinarium Divini Officii ad Matutinum}

\intermediatetitle{Ante Divinum Officium}
\begin{multicols}{2}
\lettrine{A}{peri}, Dómine, os meum ad benedicéndum nomen sanctum tuum: munda quoque cor meum ab ómnibus vanis, pervérsis et aliénis 
cogitatiónibus; intelléctum illúmina, afféctum inflámma, ut digne, atténte ac devóte hoc officium recitáre váleam, et exaudíri mérear
ante conspéctum divinae Majestátis tuae. Per Christum Dóminum nostrum. Amen.

\lettrine{D}{ómine}, in unióne illíus divínæ intentiónis, qua ipse in terris laudes Deo persolvísti, has tibi horas \rubric{(vel} hanc tibi horam\rubric{)} persólvo.

\lettrine{P}{ater noster}, qui es in cœlis, sanctificétur nomen tuum. Advéniat regnum tuum. Fiat volúntas tua, sicut in cœlo et in terra.
Panem nostrum quotidiánum da nobis hódie. Et dimítte nobis débita nostra, sicut et nos dimíttimus debitóribus nostris. Et ne nos
indúcas in tentatiónem: sed líbera nos a malo. Amen.

\lettrine{A}{ve Maria}, grátia plena, Dóminus tecum: benedíctus fructus ventris tui Jesus. Sancta María, Mater Dei, ora pro nobis 
peccatóribus, nunc et in hora mortis nostræ. Amen.

\lettrine{C}{redo in Deum}, Patrem omnipoténtem, Creatórem cœli et terræ. Et in Jesum Christum, Fílium ejus únicum, Dóminum nostrum,
qui concéptus est de spíritu Sancto, natus ex María Virgine, passus sub Póntio Piláto, crucifixus, mórtuus et sepúltus: descéndit
ad ínferos: tértia die resurréxit a mórtuis; ascéndit ad cœlos, sedet ad déxteram Patris omnipoténtis: inde ventúrus est judicáre
vivos et mórtuos. Credo in Spíritum sanctum, sanctam Ecclésiam cathólicam, Sanctórum communiónem, remissiónem peccatórum,
carnis resurrectiónem, vitam ætérnam. Amen.
\end{multicols}

\gscore{ORIa}{T}{}{Domine labia mea!Tonus simplex}
\gscore{ORIb}{T}{}{Domine labia mea!Tonus festivus}

\rubric{Invitatorium et Hymnus ut in Psalterio, vel Proprio, vel Communis.}

\nocturn{1}

\rubric{Antiphonæ, Psalmi et Versus ut in Psalterio, vel Proprio, vel Communis.}

\gscore[n]{ORPN}{T}{}{Pater Noster}


\gscore[n]{ORLb}{T}{}{Benedictio!Tonus simplex}
\gscore[n]{ORLc}{T}{}{Benedictio!Tonus solemnis}

\rubric{Quando ab uno tantum recitatur Officium,
ante singulas Lectiones, dicitur: \normaltext{Jube, Dómine, benedícere}
et subjungitur congruens Benedictio.}

\gscore[n]{ORLd}{T}{}{In fine lectionum!Tonus simplex}
\gscore[n]{ORLe}{T}{}{In fine lectionum!Tonus solemnis}

\rubric{Benedictiones pro aliis Lectionibus:}

\rubric{\emph{Benedictio 2.}} Unigénitus \textit{Dei} \textbf{Fí}lius~\GreSpecial{*}
nos benedícere et adjuváre dignétur.
\hspace{\specialcharhsep}\rr Amen.

\rubric{\emph{Benedictio 3.}} Spíritus \textit{Sancti} \textbf{grá}tia~\GreSpecial{*}
illúminet sensus et corda nostra.
\hspace{\specialcharhsep}\rr Amen.

\nocturn{2}

\rubric{Omnia ut in \Rnum{1} Nocturno, præter Absolutio et Benedictiones, ut infra.}

\smalltitle{Absolutio}
\rubric{\emph{Absolutio 2.}}
Ipsíus píetas et misericódi\textit{a nos} \textbf{ád}juvet,~\GreSpecial{*}
qui cum Patre et Spíritu Sancto vivit et regnat in sǽcula sæculórum.
\hspace{\specialcharhsep}\rr Amen.

\smalltitle{Benedictiones}

\rubric{\emph{Benedictio 4.}} Deus Pa\textit{ter om}\textbf{ní}potens~\GreSpecial{*}
sit nobis propítius et clemens.
\hspace{\specialcharhsep}\rr Amen.

\rubric{\emph{Benedictio 5.}} Chris\textit{tus per}\textbf{pé}tuæ~\GreSpecial{*}
det nobis gaúdia vitæ.
\hspace{\specialcharhsep}\rr Amen.

\rubric{\emph{Benedictio 6.}} Ignem su\textit{i a}\textbf{mó}ris~\GreSpecial{*}
accéndat Deus in córdibus nostris.
\hspace{\specialcharhsep}\rr Amen.

\nocturn{3}

\rubric{Omnia ut in \Rnum{1} Nocturno, præter Absolutio et Benedictiones, ut infra.}

\smalltitle{Absolutio}
\rubric{\emph{Absolutio 3.}}
A vínculis peccató\textit{rum nos}\textbf{tró}rum~\GreSpecial{*}
absólvat nos omnípotens et miséricors Dóminus.
\hspace{\specialcharhsep}\rr Amen.

\smalltitle{Benedictiones}

\rubric{\emph{Benedictio 7.}}
Evangé\textit{lica} \textbf{léc}tio~\GreSpecial{*}
sit nobis salus et protéctio.
\hspace{\specialcharhsep}\rr Amen.

\rubric{In Festis Domini et in Dominicis:}

\rubric{\emph{Benedictio 8.}}
Diví\textit{num au}\textbf{xí}lium~\GreSpecial{*}
máneat semper nobíscum.
\hspace{\specialcharhsep}\rr Amen.

\rubric{In Festis beatæ Mariæ Viginis:}

\rubric{\emph{Benedictio 8.}}
Cujus \textit{festum} \textbf{có}limus,~\GreSpecial{*}
ipsa Virgo vírginum intercédat pro nobis ad Dóminum.
\hspace{\specialcharhsep}\rr Amen.

\rubric{In Festis Sanctorum:}

\rubric{\emph{Benedictio 8.}}
Cujus \rubric{(vel} Quarum\rubric{)} \textit{festum} \textbf{có}limus,~\GreSpecial{*}
ipse \rubric{(vel} ipsa \rubric{aut} ipsæ\rubric{)}
intercédat \rubric{(vel} intercédant\rubric{)} pro nobis ad Dóminum.
\hspace{\specialcharhsep}\rr Amen.

\rubric{\emph{Benedictio 9.}}
Ad societátem cívium \textit{super}\textbf{nó}rum~\GreSpecial{*}
perdúcat nos Rex Angelórum.
\hspace{\specialcharhsep}\rr Amen.

\rubric{Si autem legenda ultima Lectio sit de Homilia cum Evangelio Dominicæ,
vel Feriæ, aut Vigiliæ:}

\rubric{\emph{Benedictio 9.}}
Per evangé\textit{lica} \textbf{dic}ta~\GreSpecial{*}
deleántur nostra delícta.
\hspace{\specialcharhsep}\rr Amen.

\intermediatetitle{Te Deum}

\rubric{Post ultimam Lectionem, in omnibus Dominicis per Annum,
in Festis cujusvis ritus, excepto tamen sanctorum Innocentium Festo,
nisi hoc in Dominicam indicat, aut ritu gaudeat Duplici I classis,
dicitur Hymnus Ambrosianus.
In Adventu autem, et a Dominica Septuagesimæ usque ad Sabbatum sanctum inclusive,
non dicitur nisi in Festis;
Quando vero Hymnus prædictus omittitur,
ejus loco dicitur \Rnum{9} aut \Rnum{3} Responsorium.}

\intermediatetitle{Conclusio}

\rubric{Dicto \normaltext{Te Deum}, aut ultimo Responsorio,
statim incipiuntur Laudes a Versu \normaltext{Deus, in adjutórium}.
Si non sequent Laudes, post Hymnum \normaltext{Te Deum},
vel post ultimum Responsorium, dicitur:}

\gscore[n]{ORDV}{T}{}{Dominus vobiscum}

\rubric{Hic versus non dicitur ab eo, qui non est saltem in ordine diaconatus;
sed ejus loco substituitur:}

\versiculus{Dómine, exáudi oratiónem meam.}{Et clamor meus ad te véniat.}
\vv Orémus.
\rubric{Deinde cantatur oratio et respondetur \normaltext{Amen.}}

\versiculus{Dóminus vobíscum.}{Et cum spíritu tuo.}
\rubric{vel \normaltext{Dómine, exáudi}, etc.}
\rubric{\normaltext{Benedicámus Dómino} secundum diem, pag.\ \pageref{M-TCBD}.}
\versiculus{Fidélium ánimæ per misericórdiam Dei requiéscant in pace.}{Amen.}

\rubric{Deinde dicitur \normaltext{Pater noster} totum secreto.}

\feast{F1}{Psalterium pro Dominicis}{Psalterium}{Psalterium}{1}{}{}{}{}{}{}
\addcontentsline{toc}{chapter}{Psalterium pro Dominicis}

\feast{AF1}{Tempore Adventus}
	{Psalterium}{Dominica Tempus Adventus}{2}{}{}{}{}{}{}
\addcontentsline{toc}{section}{Tempus Adventus}
\rubric{In Dominicis \Rnum{1} et \Rnum{2} Adventus.}
\gscore{A1F1I}{I}{}{Regem venturum Dominum}
\rubric{In Dominicis \Rnum{3} et \Rnum{4} Adventus, nisi Vigilia Nativitatis venerit in Dominica.}
\gscore{A3F1I}{I}{}{Prope est jam Dominus}
\gscore{A1H}{H}{}{Verbum supernum prodiens}

\nocturn{1}
\gscore{A1F1N1A1}{A}{1}{Veniet ecce}
\psalmus{1}{1}
\gscore[n]{A1F1N1A1}{A}{1}{}
\gscore{A1F1N1A2}{A}{2}{Confortate manus}
\psalmus{2}{2}
\gscore[n]{A1F1N1A2}{A}{2}{}
\gscore{A1F1N1A3}{A}{3}{Gaudete omnes}
\psalmus{3}{3}
\gscore[n]{A1F1N1A3}{A}{3}{}

\versiculus{Ex Sion spécies decóris ejus.}{Deus noster maniféste véniet.}

\nocturn{2}
\gscore{A1F1N2A1}{A}{4}{Gaude et laetare}
\psalmus{8}{4}
\gscore[n]{A1F1N2A1}{A}{4}{}
\gscore{A1F1N2A2}{A}{5}{Rex noster adveniet}
\psalmus{9i}{5}
\gscore[n]{A1F1N2A2}{A}{5}{}
\gscore{A1F1N2A3}{A}{6}{Ecce venio cito}
\psalmus{9ii}{6}
\gscore[n]{A1F1N2A3}{A}{6}{}

\versiculus{Emítte Agnum, Dómine, Dominatórem terræ.}{De Petra desérti ad montem fíliæ Sion.}

\nocturn{3}
\gscore{A1F1N3A1}{A}{7}{Gabriel Angelus locutus est}
\psalmus{9iii}{7}
\gscore[n]{A1F1N3A1}{A}{7}{}
\gscore{A1F1N3A2}{A}{8}{Maria dixit}
\psalmus{9iv}{8}
\gscore[n]{A1F1N3A2}{A}{8}{}
\gscore{A1F1N3A3}{A}{9}{In adventu summi Regis}
\psalmus{10}{4}
\gscore[n]{A1F1N3A3}{A}{9}{}
\versiculus{Egrediétur Dóminus de locl sancto ejus.}{Véniet, ut salvet pópulum suum.}

\feast{HF1}{Extra Adventus et Tempus Paschale}
	{Psalterium}{Dominica extra Adv. et T.P.}{2}{}{}{}{}{}{}
\addcontentsline{toc}{section}{Extra Adventus et Tempus Paschale}

\rubric{In Dominicis post Epiphaniam, et Dominicis mensis Octobris et Novembris.}
\gscore{F1Iw}{I}{}{Adoremus Dominum}

\rubric{In Dominicis Septuagesimæ, Sexagesimæ et Quinquagesimæ.}
\gscore{7GF1I}{I}{}{Praeoccupemus faciem}

\rubric{In Dominicis Quadragesimæ.}
\gscore{Q1I}{I}{}{Non sit vobis vanum}

\pagebreak
\rubric{In Dominicis Tempus Passionis.}
\gscore{Q5I}{I}{}{Hodie si vocem}

\rubric{In Dominicis post Pentecosten usque ad Dominicis Septembris inclusive.}
\gscore{F1Is}{I}{}{Dominum qui fecit nos}

\rubric{In Dominicis post Epiphaniam, Septuagesimæ, Sexagesimæ et Quinquagesimæ, et Dominicis mensis Octobris et Novembris.}
\gscore{F1Hw}{H}{}{Primo die quo Trinitas}

\rubric{In Dominicis Quadragesimæ.}
\gscore{Q1H}{H}{}{Ex more docti mystico}

\rubric{In Dominicis Tempus Passionis.}
\gscore{Q5H}{H}{}{Pange lingua... Lauream}

\rubric{In Dominicis post Pentecosten usque ad Dominicis Septembris inclusive.}
\gscore{F1Hs}{H}{}{Nocte surgentes}

\nocturn{1}
\gscore{F1N1A1}{A}{1}{Beatus vir... meditatur}
\psalmus{1}{8}
\gscore[n]{F1N1A1}{A}{1}{}
\gscore{F1N1A2}{A}{2}{Servite Domino}
\psalmus{2}{7}
\gscore[n]{F1N1A2}{A}{2}{}
\gscore{F1N1A3}{A}{3}{Exsurge Domine salvum}
\psalmus{3}{6}
\gscore[n]{F1N1A3}{A}{3}{}

\rubric{Per Annum.}
\versiculus{Memor fui nocte nóminis tui, Dómine.}{Et custodívi legem tuam.}
\rubric{In Quadragesima.}
\versiculus{Ipse liberávit me de láqueo venántium.}{Et a verbo áspero.}
\rubric{Tempore Passionis.}
\versiculus{Erue a frámea, Deus, ánimam meam.}{Et de manu canis únicam meam.}

\nocturn{2}
\gscore{F1N2A1}{A}{4}{Quam admirabile}
\psalmus{8}{1}
\gscore[n]{F1N2A1}{A}{4}{}
\gscore{F1N2A2}{A}{5}{Sedisti super thronum}
\psalmus{9i}{8}
\gscore[n]{F1N2A2}{A}{5}{}
\gscore{F1N2A3}{A}{6}{Exsurge Domine non praevaleat}
\psalmus{9ii}{1}
\gscore[n]{F1N2A3}{A}{6}{}

\rubric{Per Annum.}
\versiculus{Média nocte surgébam ad confiténdum tibi.}{Super judícia justificatiónis tuæ.}
\rubric{In Quadragesima.}
\versiculus{Scápulis suis obumbrábit tibi.}{Et sub pennis ejus sperábis.}
\rubric{Tempore Passionis.}
\versiculus{De ore leónis líbera me, Dómine.}{Et a cónibus unicónium humilitátem meam.}

\nocturn{3}
\gscore{F1N3A1}{A}{7}{Ut quid Domine}
\psalmus{9iii}{2}
\gscore[n]{F1N3A1}{A}{7}{}
\gscore{F1N3A2}{A}{8}{Exsurge Domine Deus exaltetur}
\psalmus{9iv}{5}
\gscore[n]{F1N3A2}{A}{8}{}
\gscore{F1N3A3}{A}{9}{Justus Dominus... dilexit}
\psalmus{10}{1}
\gscore[n]{F1N3A3}{A}{9}{}

\rubric{Per Annum.}
\versiculus{Prævenérunt óculi mei ad te dilúculo.}{Ut meditárer elóquia tua, Dómine.}
\rubric{In Quadragesima.}
\versiculus{Scuto circúmdabit te véritas ejus.}{Non timébis a timóre noctúrno.}
\rubric{Tempore Passionis.}
\versiculus{Ne perdas cum impiis, Deus, ánimam meam.}{Et cum viris sánguinum vitam meam.}

\feast{PF1}{Tempore Paschali}
	{Psalterium}{Dominica in Tempore Paschali}{2}{}{}{}{}{}{}
\addcontentsline{toc}{section}{Tempore Paschali}

\gscore{P0I}{I}{}{Surrexit Dominus vere}
\gscore{P1H}{H}{}{Rex sempiterne coelitum}

\nocturn{1}
\gscore{P1F1N1A}{A}{1}{Alleluia lapis revolutus est}
\psalmus{1}{5}
\psalmus{2}{5}
\psalmus{3}{5}
\gscore[n]{P1F1N1A}{A}{1}{}

\versiculus{Surréxit Döminus de sepúlcro, allelúja.}{Qui pro nobis pepéndit in ligno, allelúja.}

\nocturn{2}
\gscore{P1F1N2A}{A}{2}{Alleluia quem quaeris mulier}
\psalmus{8}{5}
\psalmus{9i}{5}
\psalmus{9ii}{5}
\gscore[n]{P1F1N2A}{A}{2}{}

\versiculus{Surréxit Dóminus vere, allelúja.}{Et appáruit Simóni, allelúja.}

\nocturn{3}
\gscore{P1F1N3A}{A}{3}{Alleluia noli flere Maria}
\psalmus{9iii}{5}
\psalmus{9iv}{5}
\psalmus{10}{5}
\gscore[n]{P1F1N3A}{A}{3}{}
\versiculus{Gavísi sunt discípuli, allelúja.}{Viso Dómino, allelúja.}

\feast{TC}{Toni Communes}
	{Toni Communes}{Toni Communes}{1}{}{}{}{}{}{}
\addcontentsline{toc}{chapter}{Toni communes}

\feast{TCI}{Psalmi Toni Invitatorii}
	{Toni Communes}{Psalmi Toni Invitatorii}{2}{}{}{}{}{}{}
\smalltitle{Tonus II Modo}
\gscore{ORIP2}{P}{}{Venite exsultemus!Modo 2}
\smalltitle{Tonus III Modo}
\gscore{ORIP3a}{P}{}{Venite exsultemus!Modo 3g simplex}
\gscore{ORIP3b}{P}{}{Venite exsultemus!Modo 3f solemnis}
\smalltitle{Tonus IV Modo}
\gscore{ORIP4a}{P}{}{Venite exsultemus!Modo 4e simplex}
\gscore{ORIP4b}{P}{}{Venite exsultemus!Modo 4g festivus}
\gscore{ORIP4c}{P}{}{Venite exsultemus!Modo 4d solemnis}
\smalltitle{Tonus V Modo}
\gscore{ORIP5}{P}{}{Venite exsultemus!Modo 5g}
\smalltitle{Tonus VI Modo}
\gscore{ORIP6a}{P}{}{Venite exsultemus!Modo 6a simplex}
\gscore{ORIP6b}{P}{}{Venite exsultemus!Modo 6f festivus}
\gscore{ORIP6c}{P}{}{Venite exsultemus!Modo 6f solemnis}
\smalltitle{Tonus VII Modo}
\gscore{ORIP7a}{P}{}{Venite exsultemus!Modo 7a simplex}
\gscore{ORIP7b}{P}{}{Venite exsultemus!Modo 7a solemnis}

\feast{TCTD}{Hymnus Ambrosianus}
	{Toni Communes}{Hymnus Ambrosianus}{2}{}{}{}{}{}{}
\gscore{ORTDa}{H}{}{Te Deum laudamus!Tonus solemnis}
\gscore{ORTDb}{H}{}{Te Deum laudamus!Tonus simplex}

\newpage

\feast{TCGP}{Toni Communes ad «Gloria Patri»}
	{Toni Communes}{Toni Communes}{2}{}{}{}{}{}{}
\gscore[n]{ORGP1}{T}{}{Toni Communes Ad Gloria Patri!Modo 1}
\gscore[n]{ORGP2}{T}{}{Toni Communes Ad Gloria Patri!Modo 2}
\gscore[n]{ORGP3}{T}{}{Toni Communes Ad Gloria Patri!Modo 3}
\gscore[n]{ORGP4}{T}{}{Toni Communes Ad Gloria Patri!Modo 4}
\gscore[n]{ORGP5}{T}{}{Toni Communes Ad Gloria Patri!Modo 5}
\gscore[n]{ORGP6}{T}{}{Toni Communes Ad Gloria Patri!Modo 6}
\gscore[n]{ORGP7}{T}{}{Toni Communes Ad Gloria Patri!Modo 7}
\gscore[n]{ORGP8}{T}{}{Toni Communes Ad Gloria Patri!Modo 8}

\pagebreak

\feast{TCBD}{Toni Communes ad «Benedicamus Domino»}
	{Toni Communes}{Toni Communes}{2}{}{}{}{}{}{}
\rubric{In Festis Solemnibus}
\gscore{ORBDa}{T}{}{Benedicamus Domino!In Festis Solemnibus}
\rubric{In Festis Beatae Mariae Virginis}
\gscore{ORBDd}{T}{}{Benedicamus Domino!In Festis Beatae Mariae Virginis}
\rubric{In Dominicis per annum}
\gscore{ORBDe}{T}{}{Benedicamus Domino!In Dominicis per annum}
\pagebreak
\rubric{In Octavam Paschae}
\gscore{ORBDj}{T}{}{Benedicamus Domino!In Octavam Paschae}
\rubric{Tempore Paschali}
\gscore{ORBDk}{T}{}{Benedicamus Domino!Tempore Paschali}
\rubric{In Dominicis Adventus et Quadragesimae}
\gscore{ORBDm}{T}{}{Benedicamus Domino!In Dominicis Adventus et Quadragesimae}

\pagebreak

\feast{TCALA}{Toni Communes ad «Alleluia» in Fine Antiphonarum}
	{Toni Communes}{Toni Communes}{2}{}{}{}{}{}{}
\gscore[n]{ORAL1}{T}{}{Toni Communes Ad Alleluia in Fine Antiphonarum}

\feast{TCALR}{Toni Communes ad «Alleluia» in Fine Responsorium}
	{Toni Communes}{Toni Communes}{2}{}{}{}{}{}{}
\gscore[n]{ORAL2}{T}{}{Toni Communes Ad Alleluia In Fine Responsorium}

\pagebreak

\feast{TCPA}{Toni Communes Pneumata in fine Antiphonarum}
	{Toni Communes}{Toni Communes}{2}{}{}{}{}{}{}
\rubric{Pneumata in fine ultimæ cujuscumque Nocturni Antiphonæ in Festis majoribus ubi mos est, adjungi possunt.}
\gscore[n]{ORAL3}{T}{}{Toni Communes Pneumata In Fine Antiphonarum}

\end{document}