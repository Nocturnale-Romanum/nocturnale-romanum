% !TEX TS-program = lualatex
% !TEX encoding = UTF-8

\documentclass[nocturnale-dominicis.tex]{subfiles}

\ifcsname preamble@file\endcsname
  \setcounter{page}{\getpagerefnumber{M-nd20_psalterium_ordinarium}}
\fi

\begin{document}
\feast{OR}{Ordinarium Divini Officii\\ad Matutinum}{Ordinarium}{Ordinarium}{1}{}{}{}{}{}{}
\addcontentsline{toc}{chapter}{Ordinarium Divini Officii ad Matutinum}

\intermediatetitle{Ante Divinum Officium}
\rubric{Ante Officium, flectantes genua, ab omnes secreto recitabitur:}
\begin{multicols}{2}
Aperi, Dómine, os meum ad benedicéndum nomen sanctum tuum: munda quoque cor meum ab ómnibus vanis, pervérsis et aliénis 
cogitatiónibus; intelléctum illúmina, afféctum inflámma, ut digne, atténte ac devóte hoc officium recitáre váleam, et exaudíri mérear
ante conspéctum divinae Majestátis tuae. Per Christum Dóminum nostrum. Amen.\\

Dómine, in unióne illíus divínæ intentiónis, qua ipse in terris laudes Deo persolvísti, has tibi horas \rubric{(vel} hanc tibi horam\rubric{)} persólvo.
\end{multicols}
\rubric{Posteaquam universi locum sibi proprium occupaverint, omnes assurgent et, si fit consuetudinem, recitabunt totum secreto:}
\begin{multicols}{2}
Pater noster, qui es in cœlis, sanctificétur nomen tuum. Advéniat regnum tuum. Fiat volúntas tua, sicut in cœlo et in terra.
Panem nostrum quotidiánum da nobis hódie. Et dimítte nobis débita nostra, sicut et nos dimíttimus debitóribus nostris. Et ne nos
indúcas in tentatiónem: sed líbera nos a malo. Amen.

Ave Maria, grátia plena, Dóminus tecum: benedíctus fructus ventris tui Jesus. Sancta María, Mater Dei, ora pro nobis 
peccatóribus, nunc et in hora mortis nostræ. Amen.

Credo in Deum, Patrem omnipoténtem, Creatórem cœli et terræ. Et in Jesum Christum, Fílium ejus únicum, Dóminum nostrum,
qui concéptus est de spíritu Sancto, natus ex María Virgine, passus sub Póntio Piláto, crucifixus, mórtuus et sepúltus: descéndit
ad ínferos: tértia die resurréxit a mórtuis; ascéndit ad cœlos, sedet ad déxteram Patris omnipoténtis: inde ventúrus est judicáre
vivos et mórtuos. Credo in Spíritum sanctum, sanctam Ecclésiam cathólicam, Sanctórum communiónem, remissiónem peccatórum,
carnis resurrectiónem, vitam ætérnam. Amen.
\end{multicols}

\rubric{Tum Hebdomadarius, omnes se in ore pollice dextero signans, cantabit \normaltext{Dómine, lábia mea} ac responso a choro \normaltext{Et os meum}, se manu signans continuabit Hebdomadarius \normaltext{Deus in adjutórium}. Dum cantatur \normaltext{Glória Patri}, omnes caput profunde inclinabunt. Hæc in Matutinis Defunctorum aut in triduo ante Pascham omittuntur, in quo absolute cum prima antiphona Officium incipiatur.}

\gscore{ORIa}{T}{}{Domine labia mea!Tonus simplex}
\gscore{ORIb}{T}{}{Domine labia mea!Tonus festivus}

\rubric{Postea cantatur invitatorium, ut in Psalterio in Dominicis, aut in Proprio in Festis. Antiphonam ante psalmum \normaltext{Veníte} Cantores cantant, et omnes repetant. Post singulos ejusdem psalmi versus, a Cantores cantatum, antiphonam vel integram vel ab asterisco \GreSpecial{*} alternatim omnes cantatur. Expleto psalmo, inchoatur hymnus ab Hebdomadario, et a duobus choris alternatim cantatur. Quum in ultimo versu nominabitur SS. Trinitas, caput ab omnibus inclinabitur.}

\nocturn{1}
\rubric{Deinde cantantur antiphonæ convenientes, quæ in Officiis ritus duplicis ante et post psalmos integræ cantantur; in Officiis autem ritus semiduplicis initio psalmi inchoantur usque ad asteriscum \GreSpecial{*}, atque in fine integræ cantantur. Incepto primo versu primi psalmi, donec antiphona ultimi psalmi cujuslibet Nocturni repetita sit, sedent omnes. Antiphonæ inchoantur in primis ab Hebdomadario, deinde e dignissimo clero ad minus dignos fideles, secundum consuetudinem. Post ultimam antiphonam cujuslibet Nocturni, omnes stantes, cantat Cantor versum ut in Psalterio aut in Proprio, et respondent omnes.}
\rubric{Post versum cujuslibet Nocturni inchoat Hebdomadarius \normaltext{Pater noster}, deinde secreto usque ad :}
\versiculus{Et ne nos indúcas in tentatiónem.}{Sed líbera nos a malo.}
\smalltitle{Absolutio}
\rubric{Hebdomadarius cantat:}
\gscore[n]{ORA}{T}{}{Absolutio}
\newpage
\smalltitle{Benedictiones et Lectiones}
\rubric{Lector, qui canturus sit lectionem primam, junctis manibus et ad Hebdomadarium conversus, cantat:}
\gscore[n]{ORLa}{T}{}{}
\rubric{Si in choro non est diaconos vel sacerdos, ante singulas Lectiones Matutini, cantat Lector \normaltext{Jube, Dómine, benedícere}, ad Crucem conversus, et ipse Lector cantat congruens Benedictio. Digniores sive seniores ultimas Lectiones, minus antiqui primas cantent oportet.}
\rubric{Hebdomadarius respondet:}
\gscore[n]{ORLb}{T}{}{Benedictio!Tonus simplex}
\rubric{Tonus solemnior:}
\gscore[n]{ORLc}{T}{}{Benedictio!Tonus solemnior}
\rubric{Post \normaltext{Amen} a choro responsum, sedent omnes. Lectio peracta, fit Lector reverentiam ad Crucem, si Canonicus, secus genuflexionem, et cantat:}
\gscore[n]{ORLd}{T}{}{In fine lectionum!Tonus simplex}
\pagebreak
\rubric{Tonus solemnior:}
\gscore[n]{ORLe}{T}{}{In fine lectionum!Tonus solemnior}
\rubric{Post quamlibet vero Lectionem, quae Hymnum \normaltext{Te Deum} immediate non præcedat, congruens dicitur Responsorium, et in fine ultimi Responsorii cujuslibet Nocturni additur Versus:  \normaltext{Glória Patri, et Fílio, et Spirítui Sancto,} et Responsorium a Signo\/ {\upshape \dag,} et quidem a secundo\/ {\upshape \ddag} aut tertio\/  {\upshape\P,} si duo vel tria fuerint, repetitur.}
\rubric{Pro Lectionibus secunda, tertia, quinta, sexta et octava, unus Hebdomadarius ad benedictionem assurget. Alia ut pro Lectione prima, præter benedictio, ut infra.}

\versiculus{\rubric{2.} Unigénitus \textit{Dei} \textbf{Fí}lius \GreSpecial{*} nos benedícere et adjuváre dignétur.}{Amen.}
\versiculus{\rubric{3.} Spíritus \textit{Sancti} \textbf{grá}tia \GreSpecial{*} illúminet sensus et corda nostra.}{Amen.}

\nocturn{2}

\rubric{Omnia ut in \Rnum{1} Nocturno, præter Absolutio et Benedictiones, ut infra.}

\smalltitle{Absolutio}
\versiculus{Ipsíus píetas et misericódi\textit{a nos} \textbf{ád}juvet, \GreSpecial{*} qui cum Patre et Spíritu Sancto vivit et regnat in sǽcula sæculórum.}{Amen.}

\smalltitle{Benedictiones}

\versiculus{\rubric{4.} Deus Pa\textit{ter om}\textbf{ní}potens \GreSpecial{*} sit nobis propítius et clemens.}{Amen.}
\versiculus{\rubric{5.} Chris\textit{tus per}\textbf{pé}tuæ \GreSpecial{*} det nobis gaúdia vitæ.}{Amen.}
\versiculus{\rubric{6.} Ignem su\textit{i a}\textbf{mó}ris \GreSpecial{*} accéndat Deus in córdibus nostris.}{Amen.}

\nocturn{3}

\rubric{Omnia ut in \Rnum{1} Nocturno, præter Absolutio et Benedictiones, ut infra.}

\smalltitle{Absolutio}
\versiculus{A vínculis peccató\textit{rum nos}\textbf{tró}rum \GreSpecial{*} absólvat nos omnípotens et miséricors Dóminus.}{Amen.}

\smalltitle{Benedictiones}

\versiculus{\rubric{7.} Evangé\textit{lica} \textbf{léc}tio \GreSpecial{*} sit nobis salus et protéctio.}{Amen.}
\rubric{In Evangelii lectione stant omnes.}

\rubric{In Festis Domini et in Dominicis :}
\versiculus{\rubric{8.} Diví\textit{num au}\textbf{xí}lium \GreSpecial{*} máneat semper nobíscum.}{Amen.}

\rubric{In Festis beatæ Mariæ Viginis :}
\versiculus{\rubric{8.} Cujus \textit{festum} \textbf{có}limus, \GreSpecial{*} ipsa Virgo vírginum intercédat pro nobis ad Dóminum.}{Amen.}

\rubric{In Festis Sanctorum :}
\versiculus{\rubric{8.} Cujus \rubric{(vel} Quarum\rubric{)} \textit{festum} \textbf{có}limus, \GreSpecial{*} ipse \rubric{(vel} ipsa \rubric{aut} ipsæ\rubric{)} intercédat \rubric{(vel} intercédant\rubric{)} pro nobis ad Dóminum.}{Amen.}

\rubric{Hebdomadarius, conversus ad digniorem de choro, benedictionem ab illo petet, et dignior respondebit:}
\versiculus{\rubric{9.} Ad societátem cívium \textit{super}\textbf{nó}rum \GreSpecial{*} perdúcat nos Rex Angelórum.}{Amen.}
\rubric{Ab Epsicopo autem, ultimam Lectionem cantaturo, item cantatur: \normaltext{Jube, Dómine, benedícere}, et respondetur a Choro \normaltext{Amen}, Benedictio omissa, et omnes stant quando fit Lectio.}

\rubric{Si autem legenda ultima Lectio sit de Homilia cum Evangelio Dominicæ, vel Feriæ, aut Vigilia :}
\versiculus{\rubric{9.} Per evangé\textit{lica} \textbf{dic}ta \GreSpecial{*} deleántur nostra delícta.}{Amen.}

\rubric{Si \normaltext{Te Deum} cantatur, omnes stantes, inchoatur a Hebdomadario. Ad versus \normaltext{Te ergo}, omnes genuflectent.}
\rubric{Deinde cantantur Laudes matutinas. Si Laudes non sequent, cantat Hebdomadarius:}
\gscore[n]{ORDV}{T}{}{Dominus vobiscum}
\rubric{Si Hebdomadarius non est diaconus, cantatur similiter:}
\versiculus{Dómine, exáudi oratiónem meam.}{Et clamor meus ad te véniat.}
\vv Orémus.
\rubric{Deinde cantatur oratio ut ad Laudes, sed semper sine commemorationes. Respondetur \normaltext{Amen}. Item cantat Hebdomadarius:}
\versiculus{Dóminus vobíscum.}{Et cum spíritu tuo.}
\rubric{vel \normaltext{Dómine, exáudi}, etc., et cantant Cantores \normaltext{Benedicámus Dómino} secundum diem, pag.\ \pageref{M-TCBD}. Omnes respondent \normaltext{Deo grátias}. Item cantat Hebdomadarius, voce recta et depressa, omnes se manu signans:}
\versiculus{Fidélium ánimæ per misericórdiam Dei requiéscant in pace.}{Amen.}
\rubric{Et omnes, genua flectantes, secreto dicent \normaltext{Pater noster}, si fit consuetudinem.}
\newpage

\feast{F1}{Psalterium pro Dominicis}{Psalterium}{Psalterium}{1}{}{}{}{}{}{}
\addcontentsline{toc}{chapter}{Psalterium pro Dominicis}

\feast{AF1}{Tempore Adventus}
	{Psalterium}{Dominica Tempus Adventus}{2}{}{}{}{}{}{}
\rubric{In Dominicis \Rnum{1} et \Rnum{2} Adventus.}
\gscore{A1F1I}{I}{}{Regem venturum Dominum}
\rubric{In Dominicis \Rnum{3} et \Rnum{4} Adventus, nisi Vigilia Nativitatis venerit in Dominica.}
\gscore{A3F1I}{I}{}{Prope est jam Dominus}
\gscore{A1H}{H}{}{Verbum supernam prodiens}

\nocturn{1}
\gscore{A1F1N1A1}{A}{1}{Veniet ecce}
\psalmus{1}{1}
\gscore[n]{A1F1N1A1}{A}{1}{}
\gscore{A1F1N1A2}{A}{2}{Confortate manus}
\psalmus{2}{2}
\gscore[n]{A1F1N1A2}{A}{2}{}
\gscore{A1F1N1A3}{A}{3}{Gaudete omnes}
\psalmus{3}{3}
\gscore[n]{A1F1N1A3}{A}{3}{}
\rubric{Si Officium fit unius Nocturni, hic Versus omittitur.}
\versiculus{Ex Sion spécies decóris ejus.}{Deus noster maniféste véniet.}

\nocturn{2}
\gscore{A1F1N2A1}{A}{4}{Gaude et laetare}
\psalmus{8}{4}
\gscore[n]{A1F1N2A1}{A}{4}{}
\gscore{A1F1N2A2}{A}{5}{Rex noster adveniet}
\psalmus{9i}{5}
\gscore[n]{A1F1N2A2}{A}{5}{}
\gscore{A1F1N2A3}{A}{6}{Ecce venio cito}
\psalmus{9ii}{6}
\gscore[n]{A1F1N2A3}{A}{6}{}
\rubric{Si Officium fit unius Nocturni, hic Versus omittitur.}
\versiculus{Emítte Agnum, Dómine, Dominatórem terræ.}{De Petra desérti ad montem fíliæ Sion.}

\nocturn{3}
\gscore{A1F1N3A1}{A}{7}{Gabriel Angelus locutus est}
\psalmus{9iii}{7}
\gscore[n]{A1F1N3A1}{A}{7}{}
\gscore{A1F1N3A2}{A}{8}{Maria dixit}
\psalmus{9iv}{8}
\gscore[n]{A1F1N3A2}{A}{8}{}
\gscore{A1F1N3A3}{A}{9}{In adventu summi Regis}
\psalmus{10}{4}
\gscore[n]{A1F1N3A3}{A}{9}{}
\versiculus{Egrediétur Dóminus de locl sancto ejus.}{Véniet, ut salvet pópulum suum.}
\rubric{Si Officium fit unius Nocturni :}
\versiculus{Ex Sion spécies decóris ejus.}{Deus noster maniféste véniet.}

\newpage
\feast{HF1}{Extra Adventus et Tempus Paschale}
	{Psalterium}{Dominica extra Adv. et T.P.}{2}{}{}{}{}{}{}

\rubric{In Dominicis post Epiphaniam, et Dominicis mensis Octobris et Novembris.}
\gscore{F1Iw}{I}{}{Adoremus Dominum}

\rubric{In Dominicis Septuagesimæ, Sexagesimæ et Quinquagesimæ.}
\gscore{7GF1I}{I}{}{Praeoccupemus faciem}

\rubric{In Dominicis \Rnum{1}, \Rnum{2}, \Rnum{3} et \Rnum{4} Quadragesimæ.}
\gscore{Q1I}{I}{}{Non sit vobis vanum}

\pagebreak
\rubric{In Dominicis Passionis et in Palmis.}
\gscore{Q5I}{I}{}{Hodie si vocem}

\rubric{In Dominicis post Pentecosten usque ad Dominicis Septembris inclusive.}
\gscore{F1Is}{I}{}{Dominum qui fecit nos}

\rubric{In Dominicis post Epiphaniam, Septuagesimæ, Sexagesimæ et Quinquagesimæ, et Dominicis mensis Octobris et Novembris.}
\gscore{F1Hw}{H}{}{Primo die quo Trinitas}

\rubric{In Dominicis \Rnum{1}, \Rnum{2}, \Rnum{3} et \Rnum{4} Quadragesimæ.}
\gscore{Q1H}{H}{}{Ex more docti mystico}

\rubric{In Dominicis Passionis et in Palmis.}
\gscore{Q5H}{H}{}{Pange lingua... Lauream}

\rubric{In Dominicis post Pentecosten usque ad Dominicis Septembris inclusive.}
\gscore{F1Hs}{H}{}{Nocte surgentes}

\nocturn{1}
\gscore{F1N1A1}{A}{1}{Beatus vir... meditatur}
\psalmus{1}{8}
\gscore[n]{F1N1A1}{A}{1}{}
\gscore{F1N1A2}{A}{2}{Servite Domino}
\psalmus{2}{7}
\gscore[n]{F1N1A2}{A}{2}{}
\gscore{F1N1A3}{A}{3}{Exsurge Domine salvum}
\psalmus{3}{6}
\gscore[n]{F1N1A3}{A}{3}{}
\rubric{Si Officium fit unius Nocturni, hic Versus omittitur.}\\
\rubric{Per Annum.}
\versiculus{Memor fui nocte nóminis tui, Dómine.}{Et custodívi legem tuam.}
\rubric{In Quadragesima.}
\versiculus{Ipse liberávit me de láqueo venántium.}{Et a verbo áspero.}
\rubric{Tempore Passionis.}
\versiculus{Erue a frámea, Deus, ánimam meam.}{Et de manu canis únicam meam.}

\nocturn{2}
\gscore{F1N2A1}{A}{4}{Quam admirabile}
\psalmus{8}{1}
\gscore[n]{F1N2A1}{A}{4}{}
\gscore{F1N2A2}{A}{5}{Sedisti super thronum}
\psalmus{9i}{8}
\gscore[n]{F1N2A2}{A}{5}{}
\gscore{F1N2A3}{A}{6}{Exsurge Domine non praevaleat}
\psalmus{9ii}{1}
\gscore[n]{F1N2A3}{A}{6}{}
\rubric{Si Officium fit unius Nocturni, hic Versus omittitur.}\\
\rubric{Per Annum.}
\versiculus{Média nocte surgébam ad confiténdum tibi.}{Super judícia justificatiónis tuæ.}
\rubric{In Quadragesima.}
\versiculus{Scápulis suis obumbrábit tibi.}{Et sub pennis ejus sperábis.}
\rubric{Tempore Passionis.}
\versiculus{De ore leónis líbera me, Dómine.}{Et a cónibus unicónium humilitátem meam.}

\nocturn{3}
\gscore{F1N3A1}{A}{7}{Ut quid Domine}
\psalmus{9iii}{2}
\gscore[n]{F1N3A1}{A}{7}{}
\gscore{F1N3A2}{A}{8}{Exsurge Domine Deus exaltetur}
\psalmus{9iv}{5}
\gscore[n]{F1N3A2}{A}{8}{}
\gscore{F1N3A3}{A}{9}{Justus Dominus... dilexit}
\psalmus{10}{1}
\gscore[n]{F1N3A3}{A}{9}{}
\rubric{Si Officium fit trium Nocturnorum :}\\
\rubric{Per Annum.}
\versiculus{Prævenérunt óculi mei ad te dilúculo.}{Ut meditárer elóquia tua, Dómine.}
\rubric{In Quadragesima.}
\versiculus{Scuto circúmdabit te véritas ejus.}{Non timébis a timóre noctúrno.}
\rubric{Tempore Passionis.}
\versiculus{Ne perdas cum impiis, Deus, ánimam meam.}{Et cum viris sánguinum vitam meam.}
\rubric{Si Officium fit unius Nocturni :}\\
\rubric{Per Annum.}
\versiculus{Prævenérunt óculi mei ad te dilúculo.}{Ut meditárer elóquia tua, Dómine.}
\rubric{In Quadragesima.}
\versiculus{Ipse liberávit me de láqueo venántium.}{Et a verbo áspero.}
\rubric{Tempore Passionis.}
\versiculus{Erue a frámea, Deus, ánimam meam.}{Et de manu canis únicam meam.}

\feast{PF1}{Tempore Paschali}{Psalterium}{Dominica in Tempore Paschali}{2}{}{}{}{}{}{}
\gscore{P1I}{I}{}{Surrexit Dominus vere}
\gscore{P1H}{H}{}{Rex sempiterne coelitum}

\nocturn{1}
\gscore{P1F1N1A}{A}{1}{Alleluia lapis revolutus est}
\psalmus{1}{5}
\psalmus{2}{5}
\psalmus{3}{5}
\gscore[n]{P1F1N1A}{A}{1}{}
\rubric{Si Officium fit unius Nocturni, non repetitur Antiphona, neque Versus cantatur, sed Psalmus 8 et sequenti cantatur.}
\versiculus{Surréxit Döminus de sepúlcro, allelúja.}{Qui pro nobis pepéndit in ligno, allelúja.}

\nocturn{2}
\gscore{P1F1N2A}{A}{2}{Alleluia quem quaeris mulier}
\psalmus{8}{5}
\psalmus{9i}{5}
\psalmus{9ii}{5}
\gscore[n]{P1F1N2A}{A}{2}{}
\rubric{Si Officium fit unius Nocturni, non repetitur Antiphona, neque Versus cantatur, sed Psalmus 9iii et sequenti cantatur.}
\versiculus{Surréxit Dóminus vere, allelúja.}{Et appáruit Simóni, allelúja.}

\nocturn{3}
\gscore{P1F1N3A}{A}{3}{Alleluia noli flere Maria}
\psalmus{9iii}{5}
\psalmus{9iv}{5}
\psalmus{10}{5}
\gscore[n]{P1F1N3A}{A}{3}{}
\versiculus{Gavísi sunt discípuli, allelúja.}{Viso Dómino, allelúja.}
\rubric{Si Officium fit unius Nocturni, prima Antiphona repetitur, et hic Versus cantatur.}
\gscore[n]{P1F1N1A}{A}{1}{}
\versiculus{Surréxit Döminus de sepúlcro, allelúja.}{Qui pro nobis pepéndit in ligno, allelúja.}

\feast{TC}{Toni Communes}
	{Toni Communes}{Toni Communes}{1}{}{}{}{}{}{}
\addcontentsline{toc}{chapter}{Toni communes}

\feast{TCI}{Psalmi Toni Invitatorii}
	{Toni Communes}{Psalmi Toni Invitatorii}{2}{}{}{}{}{}{}
\smalltitle{Tonus II Modo}
\gscore{ORIP2}{P}{}{Venite exsultemus!Modo 2}
\smalltitle{Tonus III Modo}
\gscore{ORIP3a}{P}{}{Venite exsultemus!Modo 3g simplex}
\gscore{ORIP3b}{P}{}{Venite exsultemus!Modo 3f solemnis}
\smalltitle{Tonus IV Modo}
\gscore{ORIP4a}{P}{}{Venite exsultemus!Modo 4e simplex}
\gscore{ORIP4b}{P}{}{Venite exsultemus!Modo 4g festivus}
\gscore{ORIP4c}{P}{}{Venite exsultemus!Modo 4d solemnis}
\smalltitle{Tonus V Modo}
\gscore{ORIP5}{P}{}{Venite exsultemus!Modo 5g}
\smalltitle{Tonus VI Modo}
\gscore{ORIP6a}{P}{}{Venite exsultemus!Modo 6a simplex}
\gscore{ORIP6b}{P}{}{Venite exsultemus!Modo 6f festivus}
\gscore{ORIP6c}{P}{}{Venite exsultemus!Modo 6f solemnis}
\smalltitle{Tonus VII Modo}
\gscore{ORIP7a}{P}{}{Venite exsultemus!Modo 7a simplex}
\gscore{ORIP7b}{P}{}{Venite exsultemus!Modo 7a solemnis}

\feast{TCTD}{Hymnus Ambrosianus}
	{Toni Communes}{Hymnus Ambrosianus}{2}{}{}{}{}{}{}
\gscore{ORTDa}{H}{}{Te Deum laudamus!Tonus solemnis}
\gscore{ORTDb}{H}{}{Te Deum laudamus!Tonus simplex}

\newpage

\feast{TCGP}{Toni Communes ad \og Gloria Patri\fg}
	{Toni Communes}{Toni Communes}{2}{}{}{}{}{}{}
\gscore[n]{ORGP1}{T}{}{Toni Communes Ad Gloria Patri!Modo 1}
\gscore[n]{ORGP2}{T}{}{Toni Communes Ad Gloria Patri!Modo 2}
\gscore[n]{ORGP3}{T}{}{Toni Communes Ad Gloria Patri!Modo 3}
\gscore[n]{ORGP4}{T}{}{Toni Communes Ad Gloria Patri!Modo 4}
\gscore[n]{ORGP5}{T}{}{Toni Communes Ad Gloria Patri!Modo 5}
\gscore[n]{ORGP6}{T}{}{Toni Communes Ad Gloria Patri!Modo 6}
\gscore[n]{ORGP7}{T}{}{Toni Communes Ad Gloria Patri!Modo 7}
\gscore[n]{ORGP8}{T}{}{Toni Communes Ad Gloria Patri!Modo 8}

\pagebreak

\feast{TCBD}{Toni Communes ad \og Benedicamus Domino\fg}
	{Toni Communes}{Toni Communes}{2}{}{}{}{}{}{}
\rubric{In Festis Solemnibus}
\gscore{ORBDa}{T}{}{Benedicamus Domino!In Festis Solemnibus}
\rubric{In Festis Beatae Mariae Virginis}
\gscore{ORBDd}{T}{}{Benedicamus Domino!In Festis Beatae Mariae Virginis}
\rubric{In Dominicis per annum}
\gscore{ORBDe}{T}{}{Benedicamus Domino!In Dominicis per annum}
\pagebreak
\rubric{In Octavam Paschae}
\gscore{ORBDj}{T}{}{Benedicamus Domino!In Octavam Paschae}
\rubric{Tempore Paschali}
\gscore{ORBDk}{T}{}{Benedicamus Domino!Tempore Paschali}
\rubric{In Dominicis Adventus et Quadragesimae}
\gscore{ORBDm}{T}{}{Benedicamus Domino!In Dominicis Adventus et Quadragesimae}

\pagebreak

\feast{TCAA}{Toni Communes ad \og Alleluia\fg\ in fine Antiphonarum}
	{Toni Communes}{Toni Communes}{2}{}{}{}{}{}{}
\gscore[n]{ORAL1}{T}{}{Toni Communes Ad Alleluia in Fine Antiphonarum}

\feast{TCAR}{Toni Communes ad \og Alleluia\fg\ in fine Responsorium}
	{Toni Communes}{Toni Communes}{2}{}{}{}{}{}{}
\gscore[n]{ORAL2}{T}{}{Toni Communes Ad Alleluia In Fine Responsorium}

\pagebreak

\feast{TCPA}{Toni Communes Pneumata in fine Antiphonarum}
	{Toni Communes}{Toni Communes}{2}{}{}{}{}{}{}
\rubric{Pneumata in fine ultimæ cujuscumque Nocturni Antiphonæ in Festis majoribus ubi mos est, adjungi possunt.}
\gscore[n]{ORAL3}{T}{}{Toni Communes Pneumata In Fine Antiphonarum}

\end{document}