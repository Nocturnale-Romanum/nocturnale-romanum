\item Ut quid, Deus, repulísti in finem:~* irátus est furor tuus super oves páscuæ tuæ?

\item Memor esto congregatiónis tuæ:~* quam possedísti ab inítio.

\item Redemísti virgam hereditátis tuæ:~* mons Sion, in quo habitásti in eo.

\item Leva manus tuas in supérbias eórum in finem:~* quanta malignátus est inimícus in sancto!

\item Et gloriáti sunt qui odérunt te:~* in médio solemnitátis tuæ.

\item Posuérunt signa sua, signa:~* et non cognovérunt sicut in éxitu super summum.

\item Quasi in silva lignórum secúribus excidérunt jánuas ejus in idípsum:~* in secúri et áscia dejecérunt eam.

\item Incendérunt igni sanctuárium tuum:~* in terra polluérunt tabernáculum nóminis tui.

\item Dixérunt in corde suo cognátio eórum simul:~* Quiéscere faciámus omnes dies festos Dei a terra.

\item Signa nostra non vídimus, jam non est prophéta:~* et nos non cognóscet ámplius.

\item Usquequo, Deus, improperábit inimícus:~* irrítat adversárius nomen tuum in finem?

\item Ut quid avértis manum tuam, et déxteram tuam,~* de médio sinu tuo in finem?

\item Deus autem Rex noster ante sǽcula:~* operátus est salútem in médio terræ.

\item Tu confirmásti in virtúte tua mare:~* contribulásti cápita dracónum in aquis.

\item Tu confregísti cápita dracónis:~* dedísti eum escam pópulis Æthíopum.

\item Tu dirupísti fontes, et torréntes~* tu siccásti flúvios Ethan.

\item Tuus est dies, et tua est nox:~* tu fabricátus es auróram et solem.

\item Tu fecísti omnes términos terræ:~* æstátem et ver tu plasmásti ea.

\item Memor esto hujus, inimícus improperávit Dómino:~* et pópulus insípiens incitávit nomen tuum.

\item Ne tradas béstiis ánimas confiténtes tibi,~* et ánimas páuperum tuórum ne obliviscáris in finem.

\item Réspice in testaméntum tuum:~* quia repléti sunt, qui obscuráti sunt terræ dómibus iniquitátum.

\item Ne avertátur húmilis factus confúsus:~* pauper et inops laudábunt nomen tuum.

\item Exsúrge, Deus, júdica causam tuam:~* memor esto improperiórum tuórum, eórum quæ ab insipiénte sunt tota die.

\item Ne obliviscáris voces inimicórum tuórum:~* supérbia eórum, qui te odérunt, ascéndit semper.

