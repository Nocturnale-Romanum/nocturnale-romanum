% !TEX TS-program = lualatex
% !TEX encoding = UTF-8

\documentclass[nocturnale-romanum.tex]{subfiles}

\ifcsname preamble@file\endcsname
  \setcounter{page}{\getpagerefnumber{M-nr44_sanctorale_maii}}
\fi

\begin{document}

\feast{0500}{Festa Maii}{Proprium Sanctorum}{Festa Maii}{1}{}{}{}{}{}

\feast{0501}{Ss. Philippi et Jacobi Apostolorum}
	{Proprium Sanctorum}{Festa Maii}{2}{1 Maii (rubricæ Pii X)}
	{Duplex II. classis}{(Omittitur : cf. 11 Maii)}
	{Omnia de Communi Apostolorum tempore paschali, \pageref{M-APTP}.}
	{}

\feast{0501b}{In Festo Sancti Joseph Opificis}
	{Proprium Sanctorum}{Festa Maii}{2}{1 Maii (rubricæ Joannis XXIII)}
	{(Omittitur)}{I. classis}
	{}
	{Psalmi trium nocturnorum de Communi Confessoris non Pontificis \pageref{M-CONP}, reliqua ut infra notantur.}
\gscore{0501I}{I}{}{Regem regum Dominum qui putari}
\gscore{0501H}{H}{}{Te pater Joseph opifex}
\nocturn{1}
\gscore{0501N1A}{A}{1}{Exit homo ad opus}
\versiculus{Glória et exémplar opíficum, sancte Joseph, allelúja.}{Cui obœdíre vóluit Fílius Dei, allelúja.}
\gscore{0501N1R1}{R}{1}{Sex diebus operaberis V Sex enim diebus}
\gscore{0501N1R2}{R}{2}{Posuit Dominux V Haec erat conditio}
\gscore{0501N1R3}{R}{3}{Post peccatum V Et in sudore V Gloria}
\nocturn{2}
\gscore{0501N2A}{A}{2}{Jesus cum esset triginta}
\versiculus{O magnam diginátem laboris, allelúja.}{Quem Christus sanctificávit, allelúja.}
\gscore{0501N2R1}{R}{4}{Non facies V Dignus est enim operarius}
\gscore{0501N2R2}{R}{5}{Dedisti mihi protectionem salutis V Ego protector tuus}
\gscore{0501N2R3}{R}{6}{Miser ego sum et pauper V Laborem mannum tuarum V Gloria}
\nocturn{3}
\gscore{0501N3A}{A}{3}{Nonne hic est fabri filius}
\versiculus{Verbum Dei, per quod facta sunt ómnia, allelúja.}{Dignátus est operári mánibus suis, allelúja.}
\gscore{0501N3R1}{R}{7}{Jesus erat in cipiens V Faber autem}
\gscore{0501N3R2}{R}{8}{Unde huic sapientia V Ita dicebant V Gloria}
\tedeumrubric

\feast{0502}{S. Athanasii Episcopi Confessoris et Ecclesiæ Doctoris}
	{Proprium Sanctorum}{Festa Maii}{2}{2 Maii}
	{Duplex}{III. classis}
	{Commune Doctorum, \pageref{M-CODO}.}
	{}

\feast{0503}{In Inventione Sanctæ Crucis}
	{Proprium Sanctorum}{Festa Maii}{2}{3 Maii}
	{Duplex II. classis}{(Omittitur)}
	{}
	{}
\gscore{0503I}{I}{}{Christum Regem crucifixum venite}
\gscore{0503H}{H}{}{Pange lingua... Lauream}
\nocturn{1}
\rubric{Tres Psalmi ut in \Rnum{1} Nocturno Communis unius Martyris extra tempus paschale, pagina \pageref{M-UMEX} sub hac sola Antiphona cantatur.}
% TODO : point towards the right psalm and not to the beginning of the common
\gscore{0503N1A}{A}{1}{Inventae Crucis festa recolimus}
\versiculus{Hoc signum Crucis erit in cœlo, allelúja.}{Cum Dóminus ad judicándum vénerit, allelúja.}
\gscore{0503N1R1}{R}{1}{Gloriosum diem V In ligno pendes}
\gscore{0503N1R2}{R}{2}{Crux fidelis inter omnes V Super omnia ligna}
\gscore{0503N1R3}{R}{3}{Haec est arbor dignissima V Crux praecellenti V Gloria}
\nocturn{2}
\rubric{Tres Psalmi ut in \Rnum{2} Nocturno Communis unius Martyris extra tempus paschale, pagina \pageref{M-UMEX} sub hac sola Antiphona cantatur.}
% TODO : point towards the right psalm and not to the beginning of the common
\gscore{0503N2A}{A}{2}{Felix ille triumphus}
\versiculus{Adorámus te, Christe, et benedícimus tibi, allelúja.}{Quia per Crucem tuam redemísti mundum, allelúja.}
\gscore{0503N2R1}{R}{4}{Nos autem gloriari V Tuam Crucem adoramus}
\gscore{0503N2R2}{R}{5}{Dum sacrum pignus V Ad crucis contactum}
\gscore{0503N2R3}{R}{6}{Hoc signum Crucis V Cum sederit Filius V Gloria}
\nocturn{3}
\rubric{Tres Psalmi ut in \Rnum{3} Nocturno Communis Virginum, pagina \pageref{M-MU} sub hac sola Antiphona cantatur.}
% TODO : point towards the right psalm and not to the beginning of the common
\gscore{0503N3A}{A}{3}{Adoramus te Christe... quia per Crucem}
\versiculus{Omnis terra adóret te, et psallat tibi, allelúja.}{Psalmum dicat nómini tuo, allelúja.}
\gscore{0503N3R1}{R}{7}{Dulce lignum V Hoc signum Crucis erit}
\gscore{0503N3R2}{R}{8}{Sicut Moyses exaltavit V Non misit Deus V Gloria}
\tedeumrubric

\feast{0504}{S. Monicæ Viduæ}
	{Proprium Sanctorum}{Festa Maii}{2}{4 Maii}
	{Duplex}{III. classis}
	{Commune aut Virginum aut non Virginum, \pageref{M-MU}.}
	{}

\feast{0505}{S. Pii V Papæ et Confessoris}
	{Proprium Sanctorum}{Festa Maii}{2}{5 Maii}
	{Duplex m.t.v.}{III. classis}
	{Commune Confessoris Pontificis, \pageref{M-COPO}.}
	{}

\feast{0506}{S. Joannis Apostoli ante Portam Latinam}
	{Proprium Sanctorum}{Festa Maii}{2}{6 Maii}
	{Duplex majus}{(Omittitur)}
	{Omnia de Communi Apostolorum Tempore Paschali, \pageref{M-APTP}.}
	{}

\feast{0507}{S. Stanislai Episcopi et Martyris}
	{Proprium Sanctorum}{Festa Maii}{2}{7 Maii}
	{Duplex}{III. classis}
	{Commune unius aut plurim.\ Mart.\ t.p., \pageref{M-MRTP}.}
	{}

\feast{0508}{In Apparitione S. Michaëlis Archangeli}
	{Proprium Sanctorum}{Festa Maii}{2}{8 Maii}
	{Duplex majus}{(Omittitur)}
	{Invitatorium \scorename{0324I} pag.\ \pageref{M-0324I}, psalmi trium Nocturnorum ut in Festo Gabrielis Archangeli, pag.\ \pageref{M-0324}, reliqua ut infra.}
	{}
\gscore{0508H}{H}{}{Te splendor et virtus Patris}
\nocturn{1}
\gscore{0508N1A}{A}{1}{Concussum est mare... alleluia}
\versiculus{Stetit Angelus juxta aram templi, allelúja.}{Habens thuríbulum áureum in manu sua, allelúja.}
\gscore{0508N1R1}{R}{1}{Factum est silentium in coelo V Millia millium}
\gscore{0508N1R2}{R}{2}{Stetit Angelus super aram V In conspectu Angelorum}
\gscore{0508N1R3}{R}{3}{In conspectu Angelorum V Super misericordia V Gloria}
\nocturn{2}
\gscore{0508N2A}{A}{2}{Michael Archangele venit in adjutorium... alleluia}
\versiculus{Ascéndit fumus aromátum in conspéctu Dómini, allelúja.}{De manu Angeli, allelúja.}
\gscore{0508N2R1}{R}{4}{Hic est Michael Archangelus V Archangelus Michael praepositus}
\gscore{0508N2R2}{R}{5}{Venit Michael Archangelus V Emitte Domine Spiritum}
\gscore{0508N2R3}{R}{6}{In tempore illo consurget V In tempore illo salvabitur V Gloria}
\nocturn{3}
\gscore{0508N3A}{A}{3}{Angelus Archangelus Michael Dei nuntius... alleluia alleluia}
\versiculus{In conspéctu Angelórum psallam tibi, Deus meus, allelúja.}{Adorábo ad templum sanctum tuum, et confitébor nómini tuo, allelúja.}
\gscore{0508N3R1}{R}{7}{In conspectu Gentium V Stetit Angelus juxta}
\gscore{0508N3R2}{R}{8}{Michael Archangelus venit V Stetit Angelus juxta aram V Gloria}
\tedeumrubric

\feast{0509}{S. Gregorii Nazianzeni\\Episcopi Confessoris et Ecclesiæ Doctoris}
	{Proprium Sanctorum}{Festa Maii}{2}{9 Maii}
	{Duplex}{III. classis}
	{Commune Doctorum, \pageref{M-CODO}.}
	{}

\feast{0510}{S. Antonini Episcopi et Confessoris}
	{Proprium Sanctorum}{Festa Maii}{2}{10 Maii}
	{Duplex m.t.v.}{III. classis}
	{Commune Conf. Pont., \pageref{M-COPO}.}
	{}

\feast{0511}{Ss. Philippi et Jacobi Apostolorum}
	{Proprium Sanctorum}{Festa Maii}{2}{11 Maii}
	{(Omittitur : cf. 1 Maii)}{II. classis}
	{}
	{}

\feast{0512}{Ss. Nerei, Achillei et Domitillæ Virginis\\atque Pancratii Martyrum}
	{Proprium Sanctorum}{Festa Maii}{2}{12 Maii}
	{Semiduplex}{III. classis}
	{Commune unius aut plurimorum Mm.\ tempore paschali, \pageref{M-MRTP}.}
	{}

\feast{0513}{S. Roberti Bellarmino\\Episcopi Confessoris et Ecclesiæ Doctoris}
	{Proprium Sanctorum}{Festa Maii}{2}{13 Maii}
	{Duplex m.t.v.}{III. classis}
	{Commune Doctorum, \pageref{M-CODO}.}
	{}

\feast{0514}{S. Bonifatii Martyris}
	{Proprium Sanctorum}{Festa Maii}{2}{14 Maii}
	{Simplex}{(Commemoratio tantum)}
	{Commune unius aut plurimorum Mm.\ tempore paschali, \pageref{M-MRTP}.}
	{}

\feast{0515}{S. Joannis Baptistæ de la Salle Confessoris}
	{Proprium Sanctorum}{Festa Maii}{2}{15 Maii}
	{Duplex m.t.v.}{III. classis}
	{Commune Conf. non Pont., \pageref{M-CONP}.}
	{}

\feast{0516}{S. Ubaldi Episcopi et Confessoris}
	{Proprium Sanctorum}{Festa Maii}{2}{16 Maii}
	{Semiduplex}{III. classis}
	{Commune Conf. Pont., \pageref{M-COPO}.}
	{}

\feast{0517}{S. Paschalis Baylon Confessoris}
	{Proprium Sanctorum}{Festa Maii}{2}{17 Maii}
	{Duplex}{III. classis}
	{Commune Conf. non Pont., \pageref{M-CONP}.}
	{}

\feast{0518}{S. Venantii Martyris}
	{Proprium Sanctorum}{Festa Maii}{2}{18 Maii}
	{Duplex}{III. classis}
	{Commune unius aut plurimorum Mm.\ tempore paschali, \pageref{M-MRTP}, præter hymnus, ut infra.}
	{}
\gscore{0518H}{H}{}{Athleta Christi nobilis}

\feast{0519}{S. Petri Celestini Papæ et Confessoris}
	{Proprium Sanctorum}{Festa Maii}{2}{19 Maii}
	{Duplex}{III. classis}
	{Commune Conf. Pont., \pageref{M-COPO}.}
	{}

\feast{0520}{S. Bernardini Senensis Confessoris}
	{Proprium Sanctorum}{Festa Maii}{2}{20 Maii}
	{Semiduplex}{III. classis}
	{Commune Conf. non Pont., \pageref{M-CONP}.}
	{}

\feast{0525}{S. Gregorii VII Papæ et Confessoris}
	{Proprium Sanctorum}{Festa Maii}{2}{25 Maii}
	{Duplex}{III. classis}
	{Commune Conf. Pont., \pageref{M-COPO}.}
	{}

\feast{0526}{S. Philippi Neri Confessoris}
	{Proprium Sanctorum}{Festa Maii}{2}{26 Maii}
	{Duplex m.t.v.}{III. classis}
	{Commune Conf. non Pont., \pageref{M-CONP}.}
	{}

\feast{0527}{S. Bedæ Venerabilis Confessoris et Ecclesiæ Doctoris}
	{Proprium Sanctorum}{Festa Maii}{2}{27 Maii}
	{Duplex m.t.v.}{III. classis}
	{Commune Doctorum, \pageref{M-CODO}.}
	{}

\feast{0528}{S. Augustini Episcopi et Confessoris}
	{Proprium Sanctorum}{Festa Maii}{2}{28 Maii}
	{Duplex m.t.v.}{III. classis}
	{Commune Conf. Pont., \pageref{M-COPO}.}
	{}

\feast{0529}{S. Mariæ Magdalenæ de Pazzis Virginis}
	{Proprium Sanctorum}{Festa Maii}{2}{29 Maii}
	{Semiduplex}{III. classis}
	{Commune aut Virginum aut non Virginum, \pageref{M-MU}.}
	{}

\feast{0530}{S. Felicis I Papæ et Martyris}
	{Proprium Sanctorum}{Festa Maii}{2}{30 Maii}
	{Simplex}{(Commemoratio tantum)}
	{Commune unius aut plur. Mm.\ T.P. \pageref{M-MRTP}, aut Commune unius Mart., \pageref{M-UMEX}.}
	{}

\feast{0531}{S. Angelæ Mericiæ Virginis}
	{Proprium Sanctorum}{Festa Maii}{2}{31 Maii (rubricæ Pii X)}
	{Duplex}{(Omittitur : cf. 1 Junii)}
	{Commune aut Virginum aut non Virginum, \pageref{M-MU}.}
	{}

\feast{0531b}{In Festo Beatæ Mariæ Virginis Reginæ}
	{Proprium Sanctorum}{Festa Maii}{2}{31 Maii (rubricæ Joannis XXIII)}
	{(Omittitur)}{II. classis}
	{}
	{Omnia ut in Communi B.M.V., pag.\pageref{M-CBMV}, præter ea quæ hic habentur propria.}
\gscore{0531I}{I}{}{Christum regem qui suam coronavit Matrem}
\gscore{0531H}{H}{}{Rerum suprem in vertice}
\nocturn{1}
\versiculus{Salve, Regina misericórdiæ, allelúja.}{Ex qua natus est Christus, Rex noster, allelúja.}
\respref{1}{0815N3R2}{sed sine \normaltext{Allelúja} in fine}
\gscore{0531N1R2}{R}{2}{Regalem dignitatem V Gloriam Reginae}
\gscore{0531N1R3}{R}{3}{Elegit eam Deus V Et in tabernaculo V Gloria}
\nocturn{2}
\versiculus{Stabat juxta crucem Jesu Mater ejus, allelúja.}{In passióne sócia, totíus mundi Regina, allelúja.}
\gscore{0531N2R1}{R}{4}{Suscipe verbum Virgo Maria V Et regina}
\gscore{0531N2R2}{R}{5}{Ecce positus est hic V Ave Christi Mater}
\gscore{0531N2R3}{R}{6}{Signum magnum...alleluia V Cujus filius regnat V Gloria}
\nocturn{3}
\respref{7}{0325N2R1}{}
\gscore{0531N3R2}{R}{8}{Exaltata es sancta Dei Genitrix V Intercede pro nobis V Gloria}
\tedeumrubric
\end{document}