% !TEX TS-program = lualatex
% !TEX encoding = UTF-8

\documentclass[nocturnale-romanum.tex]{subfiles}

\ifcsname preamble@file\endcsname
  \setcounter{page}{\getpagerefnumber{M-nr36_officium_defunctorum}}
\fi

\begin{document}
\feast{ODEF}{Officium Defunctorum}
	{Officium Defunctorum}{Officium Defunctorum}{2}{}
	{}{}{Defunctorum!Officium}
	{}{}
\addcontentsline{toc}{section}{Officium Defunctorum}
\rubric{Quoties Matutinum vel delationem cadaveris ad Ecclesiam ac Responsorium \normaltext{Subveníte}, vel Matutinum et Laudes diei currentis immediate non sequatur, dicitur secrete \normaltext{Pater}, \normaltext{Ave} et \normaltext{Credo}; secus absolute incipitur ab Invitatorio.}

\rubric{Sequens Invitatorium dicitur semper in Officio Defunctorum, quando persolvitur cum tribus Nocturnis, etiam sub ritu semiduplici, aut cum unico quidem Nocturno, sed sub ritu duplici. In reliquis vero casibus omittitur.}

\rubric{Nocturni vero inferius positi, omnes dici possunt vel etiam unus tantum, ita tamen, ut extra diem depositionis, in qua semper dicitur primus Nocturnus, Dominica, Feria \Rnum{2} et \Rnum{5}, dicatur primus, Feria \Rnum{3} et \Rnum{6} secundus, et Feria \Rnum{4} et Sabbato tertius.}

\gscore{ODEFIb}{I}{}{Regem cui omnia vivunt}
\gscore{ODEFIP}{P}{}{Venite exsultemus (defunctorum)}
\nocturn{1}
\smalltitle{Pro Dominica, Feria \Rnum{2} et \Rnum{5}}
\gscore{ODEFN1A1}{A}{1}{Dirige Domine deus meus in conspectu tuo viam meam}
\psalmus{5}{def}
\gscore{ODEFN1A2}{A}{2}{Convertere Domine et eripe animam}
\psalmus{6}{def}
\gscore{ODEFN1A3}{A}{3}{Nequando rapiat ut leo}
\psalmus{7}{def}
\versiculus{A porta ínferi.}{Erue, Dómine, ánimas eórum.}

Pater noster. \begin{rubric}totum secreto.\end{rubric}

\rubric{Non fit Absolutio nec Benedictio.}

\gscore{ODEFN1R1}{R}{1}{Credo quod redemptor}
\gscore{ODEFN1R2}{R}{2}{Qui Lazarum}
\gscore{ODEFN1R3}{R}{3}{Domine quando veneris}

\nocturn{2}
\smalltitle{Pro Dominica, Feria \Rnum{2} et \Rnum{5}}
\gscore{ODEFN2A1}{A}{4}{In loco pascuae}
\psalmus{22}{def}
\gscore{ODEFN2A2}{A}{5}{Delicta juventutis meae}
\psalmus{24}{def}
\gscore{Q6F7N2A2}{A}{6}{Credo videre}
\psalmus{26}{def}
\versiculus{Cóllocet eos Dóminus cum princíbus.}{Cum princípibus pópuli sui.}

Pater noster. \begin{rubric}totum secreto.\end{rubric}

\rubric{Non fit Absolutio nec Benedictio.}

\gscore{ODEFN2R1}{R}{4}{Memento mei Deus}
\gscore{ODEFN2R2}{R}{5}{Heu mihi Domine}
\gscore{ODEFN2R3}{R}{6}{Ne recorderis peccata}

\nocturn{3}
\smalltitle{Pro Dominica, Feria \Rnum{2} et \Rnum{5}}
\gscore{ODEFN3A1}{A}{7}{Complaceat tibi}
\psalmus{39}{def}
\gscore{ODEFN3A2}{A}{8}{Sana Domine animam}
\psalmus{40}{def}
\gscore{ODEFN3A3}{A}{9}{Sitivit anima mea ad Deum vivum}
\psalmus{41}{def}
\versiculus{Ne tradas béstiis ánimas confiténtes tibi.}{Et ánimas páuperum tuórum ne obliviscáris in finem.}

Pater noster. \begin{rubric}totum secreto.\end{rubric}

\rubric{Non fit Absolutio nec Benedictio.}

\gscore{ODEFN3R1}{R}{7}{Peccantem me quotidie}
\gscore{ODEFN3R2}{R}{8}{Domine secundum actum meum}
\rubric{Sequens Responsorium tunc ponitur, quando tertius tantum Nocturnus dictus fuerit pro Defunctis.}
\gscore{ODEFN3R3a}{R}{9}{Libera me Domine de viis}
\rubric{Sequens Responsorium ponitur loco præcedentis, quando dicti fuerint pro Defunctis tres Nocturni.}
\gscore{ODEFN3R3b}{R}{9}{Libera me Domine de morte}

\rubric{Oratione dicta (seu Orationibus dictis) additur sequentia :}

\versiculus{Réquiem ætérnam dona eis, Dómine.}{Et lux perpétua lúceat eis.}
\versiculus{Réquiescant in pace.}{Amen.}
\end{document}
