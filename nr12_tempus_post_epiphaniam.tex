% !TEX TS-program = lualatex
% !TEX encoding = UTF-8

\documentclass[nocturnale-romanum.tex]{subfiles}

\ifcsname preamble@file\endcsname
  \setcounter{page}{\getpagerefnumber{M-nr12_tempus_post_epiphaniam}}
\fi

\begin{document}

\feast{E1F1}{In Festo sanctæ Familiæ\\Jesu, Mariæ, Joseph}
	{Proprium de Tempore}{Sanctæ Familiæ}{2}{Dominica infra Octavam Epiphaniæ}
	{Duplex majus}{II. classis}{Jesu Christi, Domini nostri!Familia Sancta}
	{Quando die Octava Epiphaniæ in Dominicam inciderit, Sabbato præcedenti fit Officium de Sancta Familia. Sicubi tamen hoc Sabbato occurrat Festum Duplex \Rnum{1} classis, Officium de Sancta Familía cum Commemoratione ipsius Dominicæ anticipatur in proximiorem Feriam infra Octavam.}
	{}
\gscore{E1I}{I}{}{Christum Dei Filium Mariae}
\gscore{E1H}{H}{}{Sacra jam splendent decorata}
\nocturn{1}
\rubric{Psalmi trium Nocturnorum ut in Communi Festorum B.M.V., pag.\ \pageref{M-CBMV}.}
\gscore{E1N1A1}{A}{1}{Cum inducerent puerum}
\gscore{E1N1A2}{A}{2}{Ut perfecerunt omnia}
\gscore{E1N1A3}{A}{3}{Puer autem crescebat}
\versiculus{Propter nos egénus factus est cum esset dives.}{Ut illíus inópia nos dívites essémus.}
\gscore{E1N1R1}{R}{1}{Deus noster in terris}
\gscore{E1N1R2}{R}{2}{Beati qui habitant}
\gscore{E1N1R3}{R}{3}{Debuit per omnia}
\nocturn{2}
\gscore{E1N2A1}{A}{4}{Consurgens Joseph... secessit}
\gscore{E1N2A2}{A}{5}{Angelus Domini apparuit in somnis... vade}
\gscore{E1N2A3}{A}{6}{Et veniens habitavit}
\versiculus{Dóminus vias suas docébis nos.}{Et ambulábimus in sémitis ejus.}
\gscore{E1N2R1}{R}{4}{Ego autem mendicus}
\gscore{E1N2R2}{R}{5}{Vulpes foveas}
\gscore{E1N2R3}{R}{6}{Cum in forma Dei}
\nocturn{3}
\gscore{E1N3A1}{A}{7}{Ibant parentes}
\gscore{E1N3A2}{A}{8}{Cum redirent}
\gscore{E1N3A3}{A}{9}{Non invenientes Jesum}
\versiculus{Pauper sum ego, et in labóribus a juventúte mea.}{Exaltátus autem humiliátus sum, et conturbátus.}
\gscore{E1N3R1}{R}{7}{Vere tu es Rex}
\gscore{E1N3R2}{R}{8}{Sicut per inoboedientiam}
\tedeumrubric

\feast{E0B}{Infra Octavam Epiphaniæ}
	{Proprium de Tempore}{Infra Octavam Epiphaniæ}{2}{7 usque ad 12 Januarii}
	{Semiduplex}{IV. classis}{}
	{Omnia ut suo loco in Festo notatur.}
	{}

\feast{0113}{In Octava Epiphaniæ,\\ et Baptismate Domini}
	{Proprium de Tempore}{In Octava Epiphaniæ}{2}{13 Januarii}
	{Duplex majus}{II. classis}{Jesu Christi, Domini nostri!Epiphania, octava}
	{Omnia ut suo loco in Festo notatur.}
	{}

\feast{E1F1B}{Dominica I post Epiphaniam}
	{Proprium de Tempore}{Dominica I post Epiphaniam}{2}{}
	{Semiduplex Dominica minor}{(Omittitur)}{}
	{}
	{}
\rubric{Officium fit de S.\ Familia, et supra notatur; sed si Dominica in diem 13.\ Januarii inciderit, agitur Officium de die Octavæ Epiphaniæ, cum \rr 1. \scorename{0106N1R1}.}

\feast{E1F2}{Feria II}
	{Proprium de Tempore}{Hebdomada I post Epiphaniam}{3}{}
	{Feria}{IV. classis}{}
	{}
	{}
\rubric{In hac et sequentibus Feriis, si infra Octavam Epiphaniæ occurrant, Officium fit de Octava et de ea dicuntur Responsoria, ut supra: si vero extra eam incidant, Officium fit integrum ut in Ordinario et Psalterio, et sumuntur Responsoria singulis Feriis assignata, ut infra: quæ etiam dicuntur in Festis tam trium quam novem Lectionum in quibus sumantur Lectiones de Scriptura occurrenti.
Prima tamen die, qua hujusmodi Lectiones dicuntur, omissis aliis Responsoriis ea die secus recitandis, sumuntur ea quæ huic Feria \Rnum{2} sunt assignata.}
\gscore{E1F2R1}{R}{1}{Domine ne in ira}
\gscore{E1F2R2}{R}{2}{Deus qui sedes}
\gscore{E1F2R3}{R}{3}{A dextris est mihi}

\feast{E1F3}{Feria III}
	{Proprium de Tempore}{Hebdomada I post Epiphaniam}{3}{}
	{Feria}{IV. classis}{}
	{}
	{}
\gscore{E1F3R1}{R}{1}{Auribus percipe Domine}
\gscore{E1F3R2}{R}{2}{Statuit Dominus supra}
\gscore{E1F3R3}{R}{3}{Ego dixi Domine}

\feast{E1F4}{Feria IV}
	{Proprium de Tempore}{Hebdomada I post Epiphaniam}{3}{}
	{Feria}{IV. classis}{}
	{}
	{}
\gscore{E1F4R1}{R}{1}{Ne perdideris me}
\gscore{E1F4R2}{R}{2}{Paratum cor meum}
\gscore{E1F4R3}{R}{3}{Adjutor meus tibi psallam}

\feast{E1F5}{Feria V}
	{Proprium de Tempore}{Hebdomada I post Epiphaniam}{3}{}
	{Feria}{IV. classis}{}
	{}
	{}
\gscore{E1F5R1}{R}{1}{Deus in te speravi}
\gscore{E1F5R2}{R}{2}{Repleatur os meum laude}
\gscore{E1F5R3}{R}{3}{Gaudebunt labia mea}

\feast{E1F6}{Feria VI}
	{Proprium de Tempore}{Hebdomada I post Epiphaniam}{3}{}
	{Feria}{IV. classis}{}
	{}
	{}
\gscore{E1F6R1}{R}{1}{Confitebor tibi Domine Deus}
\gscore{E1F6R2}{R}{2}{Misericordia tua Domine}
\gscore{E1F6R3}{R}{3}{Factus est mihi Dominus}

\feast{E1F7}{Sabbato}
	{Proprium de Tempore}{Hebdomada I post Epiphaniam}{3}{}
	{Feria}{IV. classis}{}
	{}
	{}
\gscore{E1F7R1}{R}{1}{Misericordiam et judicium}
\gscore{E1F7R2}{R}{2}{Domine exaudi orationem}
\gscore{E1F7R3}{R}{3}{Velociter exaudi me}

\feast{E2F1}{Dominicæ II usque IV post Epiphaniam}
	{Proprium de Tempore}{Dominicæ II usque IV post Epiphaniam}{2}{}
	{Semiduplicia}{II. classis}{}
	{}
	{}
\rubric{Invitatorium \scorename{F1Iw} et Hymnus \scorename{F1Hw}, quæ dicuntur etiam in reliquis Dominicis post Epiphaniam. In \Rnum{1} Nocturno Responsoria ut in Feria \Rnum{2} præcedentis Hebdomadæ, pag.\pageref{M-E1F2}.}
\nocturn{2}
\gscore{E2N2R1}{R}{4}{Notas mihi fecisti}
\gscore{E2N2R2}{R}{5}{Diligam te Domine virtus}
\gscore{E2N2R3}{R}{6}{Domini est terra et plenitudo}
\nocturn{3}
\gscore{E2N3R1}{R}{7}{Ad te Domine levavi}
\gscore{E2N3R2}{R}{8}{Duo Seraphim}
\tedeumrubric

\feast{E2F2}{Feria II}
	{Proprium de Tempore}{Hebdomadæ II usque IV post Epiphaniam}{3}{}
	{Feria}{IV. classis}{}
	{}
	{}
\gscore{E2F2R1}{R}{1}{Quam magna multitudo}
\gscore{E2F2R2}{R}{2}{Adjutor meus esto Deus}
\gscore{E2F2R3}{R}{3}{Benedicam Dominum in omni}

\feast{E2F3}{Feria III usque ad Sabbato}
	{Proprium de Tempore}{Hebdomadæ II usque IV post Epiphaniam}{3}{}
	{Feria}{IV. classis}{}
	{}
	{}
\rubric{Omnia Responsoria ut in Feriis eisdem præcedentis hebdomadæ.}

\feast{7GF1}{Dominica in Septuagesima}
	{Proprium de Tempore}{Dominica in Septuagesima}{2}{}
	{II. classis - Semiduplex}{II. classis}{Septuagesima}
	{}
	{}
\rubric{Invitatorium \scorename{7GF1I}, et Hymnus \scorename{F1Hw}, ut in Psalterio, pag.\ \pageref{M-F1}  quæ etiam dicuntur in duabus Dominicis sequentibus. Quando sequentes Lectiones \Rnum{1} Nocturni reponuntur infra Hebdomadam juxta Rubricas, dicuntur cum suis Responsoriis hic assigantis, omissis aliis secus recitandis. Quod item servatur quoties Lectiones \Rnum{1} Nocturni alicujus Dominicæ infra Hebdomadam sint reponendæ, etiam si cum Lectionibus infra Hebdomadam positis conjugantur.}
\nocturn{1}
\gscore{7GN1R1}{R}{1}{In principio fecit Deus}
\gscore{7GN1R2}{R}{2}{In principio Deus creavit}
\gscore{7GN1R3}{R}{3}{Formavit igitur}
\nocturn{2}
\gscore{7GN2R1}{R}{4}{Tulit ergo}
\gscore{7GN2R2}{R}{5}{Dixit Dominus Deus non est}
\gscore{7GN2R3}{R}{6}{Immisit Dominus soporem}
\nocturn{3}
\gscore{7GN3R1}{R}{7}{Plantaverat autem}
\gscore{7GN3R2}{R}{8}{Ecce Adam}
\gscore{7GN3R3}{R}{9}{Ubi est Abel}

\feast{7GF2}{Feria II}
	{Proprium de Tempore}{Hebdomada Septuagesimæ}{3}{}
	{Simplex}{IV. classis}{}
	{}
	{}
\gscore{7GF2R1}{R}{1}{Dum deambularet}
\gscore{7GF2R2}{R}{2}{In sudore}
\respref{3}{7GN1R3}{}

\feast{7GF3}{Feria III}
	{Proprium de Tempore}{Hebdomada Septuagesimæ}{3}{}
	{Simplex}{IV. classis}{}
	{}
	{}
\rubric{Tria Responsoria ut in \Rnum{2} Nocturno Dominicæ præcedentis, pag.\ \pageref{M-7GN2R1}.}

\feast{7GF4}{Feria IV}
	{Proprium de Tempore}{Hebdomada Septuagesimæ}{3}{}
	{Simplex}{IV. classis}{}
	{}
	{}
\rubric{Tria Responsoria ut in \Rnum{3} Nocturno Dominicæ præcedentis, pag.\ \pageref{M-7GN3R1}.}

\feast{7GF5}{Feria V}
	{Proprium de Tempore}{Hebdomada Septuagesimæ}{3}{}
	{Simplex}{IV. classis}{}
	{}
	{}
\rubric{Tria Responsoria ut in \Rnum{1} Nocturno Dominicæ præcedentis, pag.\ \pageref{M-7GN1R1}.}

\feast{7GF6}{Feria VI}{Proprium de Tempore}{Hebdomada Septuagesimæ}{3}{}
	{Simplex}{IV. classis}{}
	{}
	{}
\rubric{Tria Responsoria ut in \Rnum{2} Nocturno Dominicæ præcedentis, pag.\ \pageref{M-7GN2R1}.}

\feast{7GF7}{Sabbato}
	{Proprium de Tempore}{Hebdomada Septuagesimæ}{3}{}
	{Simplex}{IV. classis}{}
	{}
	{}
\rubric{Tria Responsoria ut in \Rnum{3} Nocturno Dominicæ præcedentis, pag.\ \pageref{M-7GN3R1}.}

\feast{6GF1}{Dominica in Sexagesima}
	{Proprium de Tempore}{Dominica in Sexagesima}{2}{}
	{II. classis - Semiduplex}{II. classis}{Sexagesima}
	{}
	{}
\rubric{Sequentes Lectiones \Rnum{1} Nocturni, si hac nocte dici non potuerint, ponuntur cum suis Responsoriis prima die infra Hebdomadam, in qua dicendæ sint Lectiones de Scriptura occurenti ; quod servatur de Lectionibus et Responsoriis \Rnum{1} Nocturni sequentis Dominicæ Quinquagesimæ.}
\nocturn{1}
\gscore{6GN1R1}{R}{}{Dixit Dominus ad Noe}
\gscore{6GN1R2}{R}{}{Noe vir justus}
\gscore{6GN1R3}{R}{}{Quadraginta dies}
\nocturn{2}
\gscore{6GN2R1}{R}{}{Aedificavit Noe}
\gscore{6GN2R2}{R}{}{Ponam arcum meum}
\gscore{6GN2R3}{R}{}{Per memetipsum juravi}
\nocturn{3}
\gscore{6GN3R1}{R}{}{Benedixit Deus Noe}
\gscore{6GN3R2}{R}{}{Ecce ego statuam pactum meum}
\gscore{6GN3R3}{R}{}{Cum turba plurima}

\feast{6GF2}{Feria II}
	{Proprium de Tempore}{Hebdomada Sexagesimæ}{3}{}
	{Simplex}{IV. classis}{}
	{}
	{}
\gscore{6GF2R1}{R}{1}{In articulo die illius}
\gscore{6GF2R2}{R}{2}{Recordatus Dominus Noe}
\respref{3}{6GN1R3}{}

\feast{6GF3}{Feria III}
	{Proprium de Tempore}{Hebdomada Sexagesimæ}{3}{}
	{Simplex}{IV. classis}{}
	{}
	{}
\rubric{Tria Responsoria ut in \Rnum{2} Nocturno Dominicæ præcedentis, pag.\ \pageref{M-6GN2R1}.}

\feast{6GF4}{Feria IV}
	{Proprium de Tempore}{Hebdomada Sexagesimæ}{3}{}
	{Simplex}{IV. classis}{}
	{}
	{}
\respref{1}{6GN3R1}{}
\respref{2}{6GN3R2}{}
\resprefcumgp{3}{6GF2R1}

\feast{6GF5}{Feria V}
	{Proprium de Tempore}{Hebdomada Sexagesimæ}{3}{}
	{Simplex}{IV. classis}{}
	{}
	{}
\rubric{Tria Responsoria ut in \Rnum{1} Nocturno Dominicæ præcedentis, pag.\ \pageref{M-6GN1R1}.}

\feast{6GF6}{Feria VI}
	{Proprium de Tempore}{Hebdomada Sexagesimæ}{3}{}
	{Simplex}{IV. classis}{}
	{}
	{}
\rubric{Tria Responsoria ut in \Rnum{2} Nocturno Dominicæ præcedentis, pag.\ \pageref{M-6GN2R1}.}

\feast{6GF7}{Sabbato}
	{Proprium de Tempore}{Hebdomada Sexagesimæ}{3}{}
	{Simplex}{IV. classis}{}
	{}
	{}
\respref{1}{6GN3R1}{}
\respref{2}{6GN3R2}{}
\resprefcumgp{3}{6GF2R1}

\feast{5GF1}{Dominica in Quinquagesima}
	{Proprium de Tempore}{Dominica in Quinquagesima}{2}{}
	{II. classis - Semiduplex}{II. classis}{Quinquagesima}
	{}
	{}
\nocturn{1}
\gscore{5GN1R1}{R}{1}{Locutus est Dominus ad Abraham}
\gscore{5GN1R2}{R}{2}{Dum staret Abraham}
\gscore{5GN1R3}{R}{3}{Tentavit Deus Abraham}
\nocturn{2}
\gscore{5GN2R1}{R}{4}{Angelus Domini vocavit Abraham}
\gscore{5GN2R2}{R}{5}{Vocavit Angelus Domini Abraham}
\gscore{5GN2R3}{R}{6}{Deus domini mei Abraham}
\nocturn{3}
\gscore{5GN3R1}{R}{7}{Veni hodie ad fontem}
\gscore{5GN3R2}{R}{8}{Factus est sermo Domini}
\gscore{5GN3R3}{R}{9}{Caecus sedebat}

\feast{5GF2}{Feria II}
	{Proprium de Tempore}{Hebdomada Quinquagesimæ}{3}{}
	{Simplex}{IV. classis}{}
	{}
	{}
\gscore{5GF2R1}{R}{1}{Movens igitur Abraham}
\gscore{5GF2R2}{R}{2}{Credidit Abraham Deo}
\respref{3}{5GN1R3}{}

\feast{5GF3}{Feria III}
	{Proprium de Tempore}{Hebdomada Quinquagesimæ}{3}{}
	{Simplex}{IV. classis}{}
	{}
	{}
\rubric{Tria Responsoria ut in \Rnum{2} Nocturno Dominicæ præcedentis, pag.\ \pageref{M-5GN2R1}.}

\rubric{Sequens Feria \Rnum{4} Cinerum est major privilegata, reliquæ usque ad Sabbatum post Dominicam Passionis inclusive, sunt majores non priviligeatæ.}

\feast{5GF4}{Feria IV Cinerum}
	{Proprium de Tempore}{Hebdomada Quinquagesimæ}{2}{}
	{Feria privilegiata}{I. classis}{Cineres}
	{}
	{}
\rubric{Ab hac die cessant omnes Octavæ usque ad Sabbatum sanctum.}

\rubric{In hac et aliis Feriis usque ad Nonam Sabbati sequentis inclusive, omnia dicuntur ut in præcedentibus Feriis post Septuagesimam, exceptis iis, quæ hic habentur propria.}

\respref{1}{5GN3R1}{}
\respref{2}{5GN3R2}{}
\resprefcumgp{3}{5GF2R1}

\feast{5GF5}{Feria V post Cineres}
	{Proprium de Tempore}{Hebdomada Quinquagesimæ}{3}{}
	{Simplex}{III. classis}{}
	{}
	{}
\gscore{5GF5R1}{R}{1}{Domine puer meus}
\respref{2}{5GN1R2}{}
\respref{3}{5GN1R3}{}

\feast{5GF6}{Feria VI post Cineres}
	{Proprium de Tempore}{Hebdomada Quinquagesimæ}{3}{}
	{Simplex}{III. classis}{}
	{}
	{}
\rubric{Tria Responsoria ut in \Rnum{2} Nocturno Dominicæ præcedentis, pag.\ \pageref{M-5GN2R1}.}

\feast{5GF7}{Sabbato post Cineres}
	{Proprium de Tempore}{Hebdomada Quinquagesimæ}{3}{}
	{Simplex}{III. classis}{}
	{}
	{}
\respref{1}{5GN3R1}{}
\respref{2}{5GN3R2}{}
\resprefcumgp{3}{5GF2R1}

\end{document}