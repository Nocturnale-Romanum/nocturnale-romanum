% !TEX TS-program = lualatex
% !TEX encoding = UTF-8

\documentclass[nocturnale-romanum.tex]{subfiles}

\ifcsname preamble@file\endcsname
  \setcounter{page}{\getpagerefnumber{M-nr40_sanctorale_januarii}}
\fi

\begin{document}
\feast{PRSA}{Proprium Sanctorum}{Proprium Sanctorum}{Proprium Sanctorum}{1}{}{}{}{}{}

\rubric{In omnibus Festis novem Lectionum Domini, beatæ Mariæ Virginis, Angelorum, sancti Joannis Baptistæ, sancti Joseph, Apostolorum, Evangelistarum, necnon in omnibus Duplicibus \Rnum{1} et \Rnum{2} classis, integrum Officium dicitur propriæ vel specialiter assignatæ habeantur, summuntur de Communi 1 loco.

In reliquis vero Festis novem Lectionum, nisi propria suis locis assignentur, Antiphonæ et Psalmi ad omnes Horas, et ad Matutinum Versus Nocturnorum, dicuntur de occurente hebdomadæ die, ad Laudes quidem Feria \Rnum{4} etiam in \Rnum{3} Nocturno, 1 loco; Lectiones \Rnum{1} Nocturni de Scriptura occurenti, vel resumendæ aut aunticipatæ juxta Rubricas, cum suis Responsoriis de Tempore; quæ si omnio desint, Lectiones et Responsoria de Communi 1 loco, nisi aliter propriis notetur; reliqua omnia item de Communi, nempe præter Orationem:

Ad Matutinum Invitatorium, Hymnus, Lectiones \Rnum{2} et \Rnum{3} Nocturni cum suis Responsoriis.

Per Octavas autem communes, similiter Antiphonæ et Psalmi dicuntur de occurenti hebdomadæ die, reliqua, nisi aliter notetur, ut in Festo: sed Lectiones et Responsoria \Rnum{1} Nocturni, nisi propria habeantur, sumuntur de Scriptura occurenti, ut supra, e iis deficientibus, letiones in diebus infra Octavam dicuntur de Communi, in die Octava, et pro Festis Domini Commune non habentibus etiam infra Octavam quam in die Octava, et pro Festis Domini ut supra, ut in Festo.  Et de Festo etiam infra Octavam sumuntur Lectiones \Rnum{3} Nocturni, si alia non habeatur Homilia in Evangelium Festi.

In Festis autem et diebus Octavis simplicibus omnia pariter dicuntur de occurenti hebdomadæ die, et de Festo vel de Communi, ut supra; sed ad Matutinum omissis Versibus \rnum{1} et \rnum{2}, novem Psalmi sub suis Antiphonis, vel Tempore Paschali sub unica Antiphona, dicuntur continuatim in unico Nocturno, et in eo sumuntur \rnum{1} et \rnum{2} Lection de Scriptura occurenti cum suis Responsoriis de Tempore, additio \rnum{2} Responsorio \emph{Glória Patri}, ut notatur; Lection vero \rnum{3} de Festo, ut in Proprio vel Communi.}

\feast{0100}{Festa Januarii}{Proprium Sanctorum}{Festa Januarii}{1}{}{}{}{}{}

\feast{0114}{S. Hilarii Episcopi et Ecclesiæ Doctoris}
	{Proprium Sanctorum}{Festa Januarii}{2}{14 Januarii}
	{Duplex m.t.v.}{III. classis}
	{Commune Doctorum Ecclesiæ, \pageref{M-CODO}.}
	{}
	
\feast{0115}{S. Pauli Primi Eremitæ et Confessoris}
	{Proprium Sanctorum}{Festa Januarii}{2}{15 Januarii}
	{Duplex m.t.v.}{III. classis}
	{Commune Confessoris non Pontificis, \pageref{M-CONP}.}
	{}
	
\feast{0116}{S. Marcelli Papæ et Confessoris}
	{Proprium Sanctorum}{Festa Januarii}{2}{16 Januarii}
	{Semiduplex}{III. classis}
	{Commune Confessoris Pontificis, \pageref{M-COPO}.}
	{}
	
\feast{0117}{S. Antonii Abbatis}
	{Proprium Sanctorum}{Festa Januarii}{2}{17 Januarii}
	{Duplex}{III. classis}
	{Commune Abbatum, \pageref{M-COAB}.}
	{}

\feast{0118}{In Cathedra Sancti Petri Apostoli Romæ}
	{Proprium Sanctorum}{Festa Januarii}{2}{18 Januarii}
	{Duplex majus}{(Omittitur)}
	{Omnia ut in Communi Confessoris Pontificis, \pageref{M-COPO}, præter ea quæ hic habentur propria.}
	{}
\gscore{0118I}{I}{}{Tu es pastor ovium}
\gscore{0118H}{H}{}{Quodcumque in orbe}
\nocturn{1}
\gscore{0118N1R1}{R}{1}{Simon Petre antequam V Quodcumque ligaveris}
\gscore{0118N1R2}{R}{2}{Si diligis me Simon V Si oportuerit me}
\gscore{0118N1R3}{R}{3}{Tu es Petrus V Quodcumque ligaveris V Gloria}
\nocturn{2}
\gscore{0118N2R1}{R}{4}{Tu es pastor ovium V Quodcumque ligaveris}
\gscore{0118N2R2}{R}{5}{Ego pro te rogavi V Caro et sanguis non revelavit}
\gscore{0118N2R3}{R}{6}{Petre amas me V Simon Joannis diligis V Gloria}
\nocturn{3}
\gscore{0118N3R1}{R}{7}{Quem dicunt homines V Beatus es Simon}
\gscore{0118N3R2}{R}{8}{Elegit te Dominus V Immola V Gloria}

\feast{0119}{Ss. Marii, Marthæ, Audifacis et Abachum Martyrum}
	{Proprium Sanctorum}{Festa Januarii}{2}{19 Januarii}
	{Simplex}{(Commemoratio tantum)}
	{Commune plurimorum Martyrum \pageref{M-PMEX}.}
	{}

\feast{0120}{Ss. Fabiani et Sebastiani Martyrum}
	{Proprium Sanctorum}{Festa Januarii}{2}{20 Januarii}
	{Duplex}{III. classis}
	{Commune plurimorum Martyrum \pageref{M-PMEX}.}
	{}

\feast{0121}{S. Agnetis Virginis et Martyris}
	{Proprium Sanctorum}{Festa Januarii}{2}{21 Januarii}
	{Duplex}{III. classis}
	{Invitatorium, et Hymnus pro Martyre, de Communi Virginum aut non Virginum \pageref{M-CommuneVirginum}, Psalmi ut in Com.\ unius Mart.}
	{}
\nocturn{1}
\gscore{0121N1A1}{A}{1}{Discede a me pabulum}
\gscore{0121N1A2}{A}{2}{Dexteram meam et collum}
\gscore{0121N1A3}{A}{3}{Posuit signum in faciem}
\versiculus{Spécie tua et pulchritúdine tua.}{Inténde, próspere procéde, et regna.} 
\gscore{0121N1R1}{R}{1}{Diem festum... Agnes V Infantia}
\gscore{0121N1R2}{R}{2}{Dexteram meam et collum V Posuit signum}
\gscore{0121N1R3}{R}{3}{Amo Christum V Anulo fidei V Gloria}
\nocturn{2}
\gscore{0121N2A1}{A}{4}{Induit me Dominus}
\gscore{0121N2A2}{A}{5}{Mel et lac ex ejus}
\gscore{0121N2A3}{A}{6}{Ipsi soli servo fidem}
\versiculus{Adjuvábit eam Deus vultu suo.}{Deus in médio ejus, non commovébitur.}
\gscore{0121N2R1}{R}{4}{Induit me Dominus V Tradidit auribus}
\gscore{0121N2R2}{R}{5}{Mel et lac ex ejus V Ostendit mihi}
\gscore{0121N2R3}{R}{6}{Jam corpus ejus corpori V Ipsi sum desponsata V Gloria}
\nocturn{3}
\gscore{0121N3A1}{A}{7}{Cujus pulchritudinem}
\gscore{0121N3A2}{A}{8}{Christus circumdederit me vernantibus}
\gscore{0121N3A3}{A}{9}{Ipsi sum desponsata}
\versiculus{Elégit eam Deus, et prælégit eam.}{In tabernáculo suo habitáre facit eam.}
\gscore{0121N3R1}{R}{7}{Ipsi sum desponsata V Dexteram meam et collum}
\gscore{0121N3R2}{R}{8}{Omnipotens adorande V Te confiteor labiis V Gloria}
\tedeumrubric

\feast{0122}{Ss. Vincentii et Anastasii Martyrum}
	{Proprium Sanctorum}{Festa Januarii}{2}{22 Januarii}
	{Semiduplex}{III. classis}
	{Commune plurimorum Martyrum \pageref{M-PMEX}.}
	{}

\feast{0123}{S. Raymundi de Peñafort Confessoris}
	{Proprium Sanctorum}{Festa Januarii}{2}{23 Januarii}
	{Semiduplex}{III. classis}
	{Commune Confessoris non Pontificis \pageref{M-CONP}.}
	{}

\feast{0124}{S. Timothei Episcopis et Martyris}
	{Proprium Sanctorum}{Festa Januarii}{2}{24 Januarii}
	{Duplex}{III. classis}
	{Commune unius Martyris \pageref{M-UMEX}.}
	{}

\feast{0125}{In Conversione Sancti Pauli Apostoli}
	{Proprium Sanctorum}{Festa Januarii}{2}{25 Januarii}
	{Duplex majus}{III. classis}
	{Omnia de Communi Apostolorum \pageref{M-APEX}, præter ea quæ hic habentur propria.}
	{}
\gscore{0125I}{I}{}{Laudemus Deum nostrum in convesione}
\gscore{0125H}{H}{}{Egregie Doctor Paule}
\nocturn{1}
\gscore{0125N1A1}{A}{1}{Qui operatus est Petro}
\gscore{0125N1A2}{A}{2}{Scio cui credidi}
\gscore{0125N1A3}{A}{3}{Mihi vivere Christus est}
\versiculus{In omnem terram exívit sonus eórum.}{Et in fines orbis terræ verba eórum.}
\gscore{0125N1R1}{R}{1}{Qui operatus est Petro V Gratia Dei in me}
\gscore{0125N1R2}{R}{2}{Bonum certamen certavi V Scio cui credidi}
\gscore{0125N1R3}{R}{3}{Reposita est mihi corona V Scio cui credidi V Gloria}
\nocturn{2}
\gscore{0125N2A1}{A}{4}{Tu es vas electionis}
\gscore{0125N2A2}{A}{5}{Magnus sanctus Paulus}
\gscore{0125N2A3}{A}{6}{Bonum certamen certavi}
\versiculus{Constítues eos príncipes super omnem terram.}{Mémores erunt nóminis tui, Dómine.}
\gscore{0125N2R1}{R}{4}{Tu es vas electionis V Intercede pro nobis}
\gscore{0125N2R2}{R}{5}{Gratia Dei sum id quod V Qui operatus est}
\gscore{0125N2R3}{R}{6}{Saulus qui et Paulus magnus V Ostendens quia V Gloria}
\nocturn{3}
\gscore{0125N3A1}{A}{7}{Saulus qui et Paulus magnus}
\gscore{0125N3A2}{A}{8}{Ne magnitudo revelationum}
\gscore{0125N3A3}{A}{9}{Reposita est mihi corona}
\versiculus{Nimis honoráti sunt amíci tui, Deus.}{Nimis confortátus est principátus eórum.}
\gscore{0125N3R1}{R}{7}{Sancte Paule Apostole V Tu es vas electionis}
\gscore{0125N3R2}{R}{8}{Damasci praepositus gentis V Deus et Pater V Gloria}

\feast{0126}{S. Polycarpi Episcopi et Martyris}
	{Proprium Sanctorum}{Festa Januarii}{2}{26 Januarii}
	{Duplex}{III. classis}
	{Commune unius Martyris \pageref{M-UMEX}.}
	{}

\feast{0127}{S. Joannis Chrysostomi\\Episcopi Confessoris et Ecclesiæ Doctoris}
	{Proprium Sanctorum}{Festa Januarii}{2}{27 Januarii}
	{Duplex m.t.v.}{III. classis}
	{Commune Doctorum Ecclesiae \pageref{M-CODO}.}
	{}

\feast{0128}{S. Petri Nolasci Confessoris}
	{Proprium Sanctorum}{Festa Januarii}{2}{28 Januarii}
	{Duplex m.t.v.}{III. classis}
	{Commune Confessoris non Pontificis \pageref{M-CONP}.}
	{}

\feast{0129}{S. Francisci Salesii\\Episcopi Confessoris et Ecclesiæ Doctoris}
	{Proprium Sanctorum}{Festa Januarii}{2}{29 Januarii}
	{Duplex}{III. classis}
	{Commune Doctorum Ecclesiae \pageref{M-CODO}.}
	{}

\feast{0130}{S. Martinæ Virginis et Martyris}
	{Proprium Sanctorum}{Festa Januarii}{2}{30 Januarii}
	{Duplex}{III. classis}
	{Commune Virginum aut non Virginum, \pageref{M-MU}, excepto Hymno qui proprius est.}
	{}
\gscore{0130H}{H}{}{Martinae celebri plaudite nomini}

\feast{0131}{S. Joannis Bosco Confessoris}
	{Proprium Sanctorum}{Festa Januarii}{2}{31 Januarii}
	{Duplex}{III. classis}
	{Commune Confessoris non Pontificis, \pageref{M-CONP}.}
	{}

\end{document}