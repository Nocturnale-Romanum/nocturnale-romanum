\documentclass[11pt, twoside, french]{book}
%%%%%%%%%%%%%%% INDICES %%%%%%%%%%%%%%%

\usepackage{imakeidx}

\indexsetup{level=\section*,toclevel=section,noclearpage,othercode=\footnotesize\thispagestyle{empty}}
\makeindex[name=I,title=Index Invitatoriorum, columns=2,columnseprule]
\makeindex[name=H,title=Index Hymnorum, columns=2,columnseprule]
\makeindex[name=A,title=Index Antiphonarum, columns=2,columnseprule]
\makeindex[name=R,title=Index Responsoriorum, columns=2,columnseprule]
\makeindex[name=P,title=Index Psalmorum, columns=2,columnseprule]
\makeindex[name=T,title=Toni Communes, columns=2,columnseprule]
\makeindex[name=F,title=Index Festorum, columns=2,columnseprule]

%%%%%%%%%%%%%%% STANDARD PACKAGES %%%%%%%%%%%%%%%

%% This is the format of the recent Solesmes books.
\usepackage[paperwidth=135mm, paperheight=205mm]{geometry}

%% This is the format of the 1912 Antiphonale Romanum
%\usepackage[paperwidth=160mm, paperheight=240mm]{geometry}

\usepackage{fontspec}
\usepackage[nolocalmarks]{polyglossia}
\usepackage[table]{xcolor}
\usepackage{fancyhdr}
\usepackage{titlesec}
\usepackage{setspace}
\usepackage{expl3}
\usepackage{hyperref}
\usepackage{refcount}
\usepackage{needspace}
\usepackage{etoolbox}
\usepackage{enumitem}
\usepackage{lettrine}

%%%%%%%%%%%%%%% HYPHENATION AND TYPOGRAPHICAL CONVENTIONS %%%%%%%%%%%%%%

%\setdefaultlanguage[variant=ecclesiastic, hyphenation=liturgical, usej=true, babelshorthands=false]{latin}
\setdefaultlanguage[variant=french, frenchitemlabels=false]{french}

%%%%%%%%%%%%%%% GEOMETRY %%%%%%%%%%%%%%%

%% This should mimic the layout of the recent Solesmes books.
\geometry{
inner=15mm,
outer=10mm,
top=12mm,
bottom=15mm,
headsep=3mm,
}

%% General scale of all graphical elements.
%% Values different from 1 are largely untested.
%% Used in those commands (e.g. everything FontSpec) that use a scale parameter.
\newcommand{\customscale}{1}

%% Provide the command \fpevalc as a copy of the code-level \fp_eval:n.
%% \fpevalc allows to evaluate floating point calculation for scaled parameters, e.g. \setSomeStretchFactor{\fpevalc{0,6 * \customscale}}
\ExplSyntaxOn
\cs_new_eq:NN \fpevalc \fp_eval:n
\ExplSyntaxOff

%% No indentation of paragraphs
\setlength{\parindent}{0mm}

%% We want to allow large inter-words space 
%% to avoid overfull boxes in two-columns rubrics.
\sloppy

%%%%%%%%%%%%%%% GREGORIO CONFIG %%%%%%%%%%%%%%%

\usepackage[autocompile]{gregoriotex}

%% text above lines shall be of color gregoriocolor
\grechangestyle{abovelinestext}{\color{gregoriocolor}\footnotesize\itshape}
%% fine-tuning of space beween the staff and the text above lines
\newcommand{\altraise}{-0.4mm} %% default is -0.1cm
\grechangedim{abovelinestextraise}{\altraise}{scalable}
\grechangedim{abovelinestextheight}{6mm}{scalable}

%% fine-tuning of space between the staff and the lyrics
\newcommand{\textraise}{2.8ex} %% default is 3.48471 ex
\grechangedim{spacelinestext}{\textraise}{scalable}

%% fine-tuning of space between the initial and the annotations
\newcommand{\annraise}{0mm} %% default is -0.2mm
\grechangedim{annotationraise}{\annraise}{scalable}

%% \officepartannotation converts a letter (IHARPT) into the office part to be printed as annotation,
%% storing the result into \result.

\newcommand{\result}{}
\newcommand{\lookup}[3]{%
  \IfSubStr{#2}{#1}{ \renewcommand{\result}{#3} }{}%
}%
\newcommand{\officepartannotation}[1]{%
  \renewcommand{\result}{#1}%
  \lookup{#1}{T}{}%
  \lookup{#1}{H}{Hymn.}%
  \lookup{#1}{A}{Ant.}%
  \lookup{#1}{P}{}%
  \lookup{#1}{R}{Resp.}%
  \lookup{#1}{I}{Invit.}%
  \result%
}%

%% header capture setup for the mode
\newcommand{\defaultannotationshift}{-2mm}
\newcommand{\modeannotation}[1]{\greannotation{\hspace{\defaultannotationshift}#1}}
\gresetheadercapture{mode}{modeannotation}{string}

%% outputs a score with no label, indexing, initials or annotations
%% for 
\newcommand{\unindexedscore}[1]{
  \gresetinitiallines{0}
  %% the use of a directory called "gabc" is linked
  %% to the management of gabc files by the website: do not change 
  %% without also changing the website static files structure
  \gregorioscore{\subfix{gabc/#1}}
  \gresetinitiallines{1}
}

%% outputs a score with label, indexing, and annotations. no initials if [n] is passed
\makeatletter
\newcommand{\gscore}[5][y]{
  %% #1 (passed as option) : y = initial, n = no initial
  %% #2 : name of the score file, should be a code, e.g. Q4F4A3 or 1225N1R1
  %% #3 : office-part among the values: T, H, A, P, R, I (toni communes, hy., ant., psalmus, resp., invit.)
  %% #4 : if applicable, a number between 1 and 9 (rank of the ant./resp.) - else: empty
  %% #5 : the indexed name of the piece
  
  %% this prevents page breaks between the phantom section and its label, and the actual score.
  \needspace{4\baselineskip} 
  \protected@edef\@currentlabelname{#5}
  \phantomsection
  \label{#2}
  %% we add the office part, and number of that ant. or resp. in the current office, if applicable
  %% todo : the negative hspace is here because somehow the initial and annotation (first line only) are misaligned by 1mm _with this initial font size_.
  %% this should probably be fixed in a more elegant way.
  \greannotation[c]{\hspace{-1.4mm}\hspace{\defaultannotationshift}\officepartannotation{#3}#4}
  %% if #5 (indexed name) is blank, nothing is indexed.
  %% this is for pieces that are repetitions of another piece (antiphons after psalms)
  \ifblank{#5}{}{\index[#3]{#5}}
  %% if optional arg #1 has been passed as 'n', set no initial
  \ifx n#1\gresetinitiallines{0}\fi
  %% the use of a directory called "gabc" is linked
  %% to the management of gabc files by the website: do not change 
  %% without also changing the website static files structure
  \gregorioscore{\subfix{gabc/#2}}
  %% if optional arg #1 has been passed as 'n', unset no initial
  \ifx n#1\gresetinitiallines{1}\fi
  \vspace{1mm}
}
\makeatother

%%%%%%%%%%%%%%% FONTS %%%%%%%%%%%%%%%

%%%%%%%%%%%%%%% Main font
\setmainfont[Ligatures=TeX, Scale=\customscale]{Charis SIL}
%\setmainfont[Ligatures=TeX, Scale=\customscale]{TeXGyreBonum-Regular}
\setstretch{\fpevalc{1.05 * \customscale}}

%%%%%%%%%%%%%%% Score initials
%% \initialsize resizes the initials, with one argument (size in points)
\newcommand{\initialsize}[1]{
    \grechangestyle{initial}{\fontspec{Zallman Caps}\fontsize{#1}{#1}\selectfont}
}
%% default initial size is 32 points
\newcommand{\defaultinitialsize}{28}
\initialsize{\defaultinitialsize}

%% spacing before and after initials to kern the Zallman Caps.
%% this should be changed if we move away from Zallman Caps.
\newcommand{\initialspace}[2]{
  \grechangedim{afterinitialshift}{#2}{scalable}
  \grechangedim{beforeinitialshift}{#1}{scalable}
}
%% default space before and after initials is 0mm to the left and 2mm to the right.
\newcommand{\defaultinitialspace}{\initialspace{0mm}{-\defaultannotationshift}}
\defaultinitialspace{}

%%%%%%%%%%%%%%% Score annotations
\grechangestyle{annotation}{\small}

%%%%%%%%%%%%%%% GRAPHICAL ELEMENTS %%%%%%%%%%%%%%%

%% V/, R/, A/ and + signs for in-line use (\vv \rr \aa \cc)
\newcommand{\specialcharhsep}{3mm} % space after invoking R/ or V/ or A/ outside rubrics
\newcommand{\vv}{\textcolor{gregoriocolor}{\fontspec[Scale=\customscale]{Charis SIL}℣.\nolinebreak[4]\hspace{\specialcharhsep}\nolinebreak[4]}}
\newcommand{\rr}{\textcolor{gregoriocolor}{\fontspec[Scale=\customscale]{Charis SIL}℟.\nolinebreak[4]\hspace{\specialcharhsep}\nolinebreak[4]}}
\renewcommand{\aa}{\textcolor{gregoriocolor}{\fontspec[Scale=\customscale]{Charis SIL}\Abar.\nolinebreak[4]\hspace{\specialcharhsep}\nolinebreak[4]}}
\newcommand{\cc}{\textcolor{gregoriocolor}{\fontspec[Scale=\customscale]{FreeSerif}\symbol{"2720}~}}
%% Same special characters, for in-score use (<sp>V/ R/ A/ +</sp>)
\gresetspecial{V/}{\textcolor{gregoriocolor}{\fontspec[Scale=\customscale]{Charis SIL}℣.~}}
\gresetspecial{R/}{\textcolor{gregoriocolor}{\fontspec[Scale=\customscale]{Charis SIL}℟.~}}
\gresetspecial{A/}{\textcolor{gregoriocolor}{\fontspec[Scale=\customscale]{Charis SIL}\Abar.~}}
\gresetspecial{+}{{\fontspec[Scale=\customscale]{FreeSerif}†~}}
\gresetspecial{*}{\gresixstar}
\gresetspecial{cross}{\textcolor{gregoriocolor}{\fontspec[Scale=\customscale]{FreeSerif}\symbol{"2720}}}
\gresetspecial{labiacross}{\textcolor{gregoriocolor}{+}}
%% Same special characters, for use in rubrics (no space, and no red command since it will be reddified with the rest)
\newcommand{\vvrub}{{\fontspec[Scale=\customscale]{Charis SIL}℣.~}}
\newcommand{\rrrub}{{\fontspec[Scale=\customscale]{Charis SIL}℟.~}}
\newcommand{\aarub}{{\fontspec[Scale=\customscale]{Charis SIL}\Abar.~}}

%% the asterisk as found in the mediants of text-only psalms
\newcommand{\psstar}{\GreSpecial{*}}
\newcommand{\pscross}{\GreSpecial{+}}
%% also, most psalms do not call those but use † and * - todo

%% Roman Numerals
\usepackage{modroman}
\newcommand{\Rnum}[1]{\nbRoman{#1}}
\newcommand{\rnum}[1]{\nbshortroman{#1}}

%% Macro to print versicles
\newcommand{\versiculus}[2]{\par\vv #1 \\ \rr #2\par}

\newcommand{\versiculustpall}[2]
	{\versiculus{#1 \rubric{(T.P.} Allelúja. \rubric{)}}{#2 \rubric{(T.P.} Allelúja. \rubric{)}}}

%% Macro to print rubrics
\newcommand{\rubric}[1]{\textcolor{gregoriocolor}{\emph{#1}}}

%% Macro to print the name of a score in normal characters inside a \rubric
\newcommand{\normaltext}[1]{{\normalfont\normalcolor #1}}
\newcommand{\scorename}[1]{\normaltext{\nameref{M-#1}}}

%% Macro to print a full reference to a responsory
%% #1 is the R/ number in the feast, #2 is the R/ code, #3 is an optional additional text, like "sine Gloria Patri".
\newcommand{\respref}[3]{\rubric{%
\rrrub #1 \scorename{#2}, pag.\ \pageref{M-#2}%
\if\relax\detokenize{#3}\relax%
.%
\else%
, #3.%
\fi%
}}

\newcommand{\resprefcumgp}[2]
	{\respref{#1}{#2}{sed cum \normaltext{Glória Patri} in fine}}
	
\newcommand{\resprefsinegp}[2]
	{\respref{#1}{#2}{sine \normaltext{Glória Patri}}}

%% Macro to print the common rubric that signals the Te Deum
\newcommand{\tedeumrubric}{\rubric{Lectione ultima peracta Hymnus \normaltext{Te Deum} cantatur.}}

%% Macro to print a separator

\newcommand{\sep}{{\centering\greseparator{3}{20}\par}}

%%%%%%%%%%%%%%% COLUMN MANAGEMENT %%%%%%%%%%%%%%%

\usepackage{multicol}
\usepackage{parcolumns}
\setlength\columnseprule{0.4pt}
\setlength{\multicolsep}{6pt plus 2pt minus 1.5pt}

%% Macro to print a psalm on two columns.
\newcommand{\psalmus}[2]{
	\vspace{0.5\baselineskip}
	\needspace{5\baselineskip}
	\phantomsection
	\label{Psalm#1_#2}
	\index[P]{#1 (mode #2)}
	\smalltitle{Psalmus #1}
	\begin{multicols}{2}
	\begin{itemize}[
		label=\null, 
		leftmargin=0pt, 
		itemindent=0pt, 
		labelsep=0pt, 
		labelwidth=0pt, 
		rightmargin=0pt, 
		parsep=0pt, 
		itemsep=0pt]
	\input{psalmi/#1_#2.tex}
	\end{itemize}
	\end{multicols}
}

\newcommand{\twocolrubric}[2]{
	\begin{parcolumns}[rulebetween]{2}
	\colchunk{%
      \rubric{#1}
	}
	\colchunk{%
	  \rubric{#2} 
    }
	\end{parcolumns}
}

%%%%%%%%%%%%%%% HEADER STYLES %%%%%%%%%%%%%%%

\pagestyle{fancy}
\fancyhead{}
\fancyfoot{}
\renewcommand{\headrulewidth}{0pt}
\setlength{\headheight}{20pt}
\fancyhead[RO]{\small\rightmark\hspace{1cm}\thepage}
\fancyhead[LE]{\small\thepage\hspace{1cm}\leftmark}

% this command is called every time the left and right header texts are set (e.g. by calling \feast)
% the hyphenpenalty override is neede only on older versions of gregorio which do not reset it correctly after typesetting the score.
% see https://tex.stackexchange.com/questions/581013/lualatex-hyphenation-issue-in-fancyhdr-with-gregoriotex-and-multicols-latin-te
\newcommand{\setheaders}[2]{
	\renewcommand{\rightmark}{\hyphenpenalty=50{\sc#2}}
	\renewcommand{\leftmark}{\hyphenpenalty=50{\sc#1}}
}
\setheaders{}{}

%%%%%%%%%%%%%%% TITLE STYLES %%%%%%%%%%%%%%%

%% Titles are centered and small-caps
\titleformat{\chapter}[block]{\Large\filcenter\sc}{}{}{}
\titleformat{\section}[block]{\large\filcenter\sc}{}{}{}
\titleformat{\subsection}[block]{\filcenter\sc}{}{}{}
\setcounter{secnumdepth}{0}
%% Fine-tuning of space around titles
\titlespacing*{\paragraph}{0pt}{1ex}{.6ex}

%% Typesets all titles throughout the NR except Nocturn titles and a few special titles.
%% Using a continuation is necessary because there are 11 arguments.
%% Only \feast, \nocturn, \intermediatetitle and \smalltitle should ever be used.
\newcommand{\feast}[6]{
  %% #1: feast code, e.g. 1225 or A1F1
  %% #2: feast title
  %% #3: left header title
  %% #4: right header title
  %% #5: title level
    %% title level 1 : full page width, a few major feasts + titles of temporale, sanctorale, etc.
	%% title level 2 : all feasts, sundays and major ferias
	%% title level 3 : ferias
  %% #6: incipit date (goes above feast title)
  %% cont'd #1: 1954 rank
  %% cont'd #2: 1960 rank
  %% cont'd #3: name of the feast as it shows up in the index
  %% cont'd #4: 1945 feast-wide rubrics
  %% cont'd #5: 1960 feast-wide rubrics

  %% needspace: should be barely more than the vertical space for the titles, rubrics excluded.
  %% this is to ensure that the page does not get cut after the title or the phantomsection.
  \needspace{8\baselineskip}
  %% phantomsection is to allow the label to attach to the title and not the previous counter object.
  \phantomsection
  \label{#1}
  \begin{center}
  %% we typeset a line for the date if the date is not blank
  \ifblank{#6}{}{
    {\large #6}\\%
  }%
  %% the actual title
  \ifx 1#5{\setstretch{1.2}\sc\huge #2\par}\fi
  \ifx 2#5{\Large #2\par}\fi
  \ifx 3#5{\large #2\par}\fi
  \end{center}
  %% If this is a level 1 title, empty pagestyle
  \ifx 1#5\thispagestyle{empty}\fi
  
  %% we define the header titles manually
  \setheaders{#3}{#4}
  
  %% moving on to a continuation macro to unpack the last 5 arguments
  \feastcontinued
}
\newcommand{\feastcontinued}[5]
{
  %% we name the last 5 arguments
  \def\oldrank{#1}%
  \def\newrank{#2}%
  \def\indexfeastname{#3}%
  \def\oldrubric{#4}%
  \def\newrubric{#5}%
  %% we index the feast if the indexing name is given
  \ifblank{#3}{}{
	\index[F]{\indexfeastname}
  }
  %% we typeset a two-column rank & rubrics block if one rank is filled in
  \ifblank{#1}{}{%
    \begin{parcolumns}[rulebetween]{2}
	\colchunk{%
      {\centering\oldrank\par}%
	  \ifblank{#4}{}{\rubric{\oldrubric}}
	}
	\colchunk{%
	  {\centering\newrank\par}%
	  \ifblank{#5}{}{\rubric{\newrubric}}
    }
	\end{parcolumns}
	\vspace{2mm}
  }
}

\newcommand{\smalltitle}[1]{
  \needspace{5\baselineskip}
  \par{\centering\textbf{#1}\par}
}

\newcommand{\intermediatetitle}[1]{
  \needspace{8\baselineskip}
  \begin{center}
  {\large #1}
  \end{center}
 }

\newcommand{\nocturn}[1]{
  \intermediatetitle{In \Rnum{#1} Nocturno}
}

%% command to wrap printindex and set the headers for indices
\newcommand{\cprintindex}[2]{
	\setheaders{Indices}{#2}
	\pagestyle{fancy}
	\thispagestyle{empty}
	\printindex[#1]
}

%%%%%%%%%%%%%%% SUBFILES %%%%%%%%%%%%%%%

\usepackage{xr}
\usepackage{subfiles}

%% When we start a new subfile (new chapter), 
%% we start on a new page (with blank filling on the previous page) and create a corresponding label.
\newcommand{\customsubfile}[1]{\newpage\label{#1}\thispagestyle{empty}\subfile{#1}}


\raggedbottom

\usepackage{ragged2e}
\usepackage{geometry} 
\usepackage[parfill]{parskip}

\usepackage{longtable}
\usepackage{multirow,makecell}

\usepackage{layout}
\pagestyle{plain}
\setlength{\columnsep}{4mm}
\setlength{\parindent}{0mm}
\setlength{\marginparwidth}{7mm}
\setlength{\marginparsep}{3mm}
\setlength{\headsep}{10pt}

\usepackage{microtype}
\usepackage[defaultlines=2,all]{nowidow}
\usepackage[hyphenation,lastparline,nosingleletter]{impnattypo}
\usepackage{epsfig}
\spaceskip=1.0\fontdimen2\font plus 3\fontdimen3\font minus 0.8\fontdimen4\font

\begin{document}

\lineskiplimit = 0pt
\markboth{}{}
\thispagestyle{empty}
\centering {\large EPHEMERIDÆ}\label{ephemeridae} \par
\setlength\LTleft{0pt}
\setlength\LTright{0pt}
\setlength{\tabcolsep}{0pt}
\renewcommand{\arraystretch}{1.25}
\newcommand{\boldhline}{\noalign{\global\arrayrulewidth1.5pt}\hline\noalign{\global\arrayrulewidth1pt}}
\newcommand{\thinhline}{\noalign{\global\arrayrulewidth0.5pt}\hline\noalign{\global\arrayrulewidth1pt}}
\newcommand{\whiteline}{\noalign{\global\arrayrulewidth4pt}\hline\noalign{\global\arrayrulewidth1pt}}
\newcommand{\STAB}[1]{\begin{tabular}{@{}c@{}}#1\end{tabular}}
\fontsize{7.5}{7.5}\selectfont
\begin{longtable}{|>{\centering}m{0.12\textwidth}|>{\centering}m{0.08\textwidth}|>{\centering}m{0.15\textwidth}|>{\centering}m{0.15\textwidth}|>{\centering}m{0.16\textwidth}|>{\centering}m{0.16\textwidth}|>{\centering\arraybackslash}m{0.15\textwidth}|}
\arrayrulecolor{black}\boldhline
\textcolor{red}{Anno} & \textcolor{red}{Litt.\\dom.} & \textcolor{red}{Feria IV\\Cinerum} & \textcolor{red}{Pascha} & \textcolor{red}{Ascensio} & \textcolor{red}{Pentecostes} & \textcolor{red}{Adventus} \\
\boldhline
\endfirsthead
\endhead
\endfoot
\endlastfoot
\arrayrulecolor{black} 
\rule{0pt}{3.5mm}2022 & 	\rule{0pt}{3.5mm}b & 	\rule{0pt}{3.5mm}2 mar. &		\rule{0pt}{3.5mm}17 apr. & 	\rule{0pt}{3.5mm}26 maii & 		\rule{0pt}{3.5mm}5 junii & 		\rule{0pt}{3.5mm}27 nov.\\
2023 & 				\textcolor{red}{A} & 		22 febr. & 					9 apr. & 					18 maii & 					28 maii &					3 dec.\\
2024 & 				g f & 					14 febr. & 					31 mar. & 					9 maii & 					19 maii & 					1 dec.\\
2025 &				e & 					5 mar. & 					20 apr. & 					29 maii & 					8 junii & 					30 nov.\\[0.5mm]
\arrayrulecolor{red} \thinhline \arrayrulecolor{black}
\rule{0pt}{3.5mm}2026 & 	\rule{0pt}{3.5mm}d & 	\rule{0pt}{3.5mm}18 febr. & 	\rule{0pt}{3.5mm}5 apr. & 		\rule{0pt}{3.5mm}14 maii & 		\rule{0pt}{3.5mm}24 maii & 		\rule{0pt}{3.5mm}29 nov.\\
2027 &				c & 					10 febr. & 					28 mar. & 					6 maii & 					16 maii & 					28 nov.\\
2028 &				b \textcolor{red}{A} & 	1 mar. & 					16 apr. & 					25 maii & 					4 junii & 					3 dec.\\
2029 &				g & 					14 febr. & 					1 apr. & 					10 maii & 					20 maii &					2 dec.\\[0.5mm]
\arrayrulecolor{red} \thinhline \arrayrulecolor{black}
\rule{0pt}{3.5mm}2030 & 	\rule{0pt}{3.5mm}f & 		\rule{0pt}{3.5mm}6 mar. & 		\rule{0pt}{3.5mm}21 apr. & 	\rule{0pt}{3.5mm}30 maii & 		\rule{0pt}{3.5mm}9 junii & 		\rule{0pt}{3.5mm}1 dec.\\
2031 &				e & 					26 febr. & 					13 apr. 					& 22 maii 					& 1 junii & 					30 nov.\\
2032 &				d c & 				11 febr. & 					28 mar. 					& 6 maii 					& 16 maii & 				28 nov.\\
2033 &				b & 					2 mar. & 					17 apr. 					& 26 maii 					& 5 junii & 					27 nov.\\[0.5mm]
\arrayrulecolor{red} \thinhline \arrayrulecolor{black}
\rule{0pt}{3.5mm}2034 & 	\rule{0pt}{3.5mm}\textcolor{red}{A} & \rule{0pt}{3.5mm}22 febr. & \rule{0pt}{3.5mm}9 apr. & \rule{0pt}{3.5mm}18 maii & 		\rule{0pt}{3.5mm}28 maii & 		\rule{0pt}{3.5mm}3 dec.\\
2035 &				g & 					7 febr. & 					25 mar. & 					3 maii & 					13 maii  & 					2 dec.\\
2036 &				f e & 					27 febr. & 					13 apr. & 					22 maii & 					1 junii & 					30 nov.\\
2037 &				d & 					18 febr. & 					5 apr. & 					14 maii & 					24 maii & 					29 nov.\\[0.5mm]
\arrayrulecolor{red} \thinhline \arrayrulecolor{black}
\rule{0pt}{3.5mm}2038 & 	\rule{0pt}{3.5mm}c & 	\rule{0pt}{3.5mm}10 mar. & 	\rule{0pt}{3.5mm}25 apr. & 	\rule{0pt}{3.5mm}3 junii & 		\rule{0pt}{3.5mm}13 junii & 		\rule{0pt}{3.5mm}28 nov.\\ 		
2039 &				b & 					23 febr. & 					10 apr. & 					19 maii & 					29 maii  & 					27 nov.\\
2040 &				\textcolor{red}{A} g & 	15 febr. & 					1 apr. & 					10 maii & 					20 maii  & 					2 dec.\\
2041 &				f & 					6 mar. & 					21 apr. & 					30 maii & 					9 junii  & 					1 dec.\\[0.5mm]
\arrayrulecolor{red} \thinhline \arrayrulecolor{black}
\rule{0pt}{3.5mm}2042 & 	\rule{0pt}{3.5mm}e & 	\rule{0pt}{3.5mm}19 febr. & 	\rule{0pt}{3.5mm}6 apr. & 		\rule{0pt}{3.5mm}15 maii & 		\rule{0pt}{3.5mm}25 maii  & 	\rule{0pt}{3.5mm}30 nov.\\
2043 &				d & 					11 febr. & 					29 mar. & 					7 maii & 					17 maii  & 					29 nov.\\
2044 &				c b & 				2 mar. & 					17 apr. & 					26 maii & 					5 junii & 					27 nov.\\
2045 &				\textcolor{red}{A} & 		22 febr. & 					9 apr. & 					18 maii & 					28 maii & 					3 dec.\\[0.5mm]
\arrayrulecolor{red} \thinhline \arrayrulecolor{black}
\rule{0pt}{3.5mm}2046 & 	\rule{0pt}{3.5mm}g & 	\rule{0pt}{3.5mm}7 febr. & 		\rule{0pt}{3.5mm}25 mar. & 	\rule{0pt}{3.5mm}3 maii &		\rule{0pt}{3.5mm}13 maii & 		\rule{0pt}{3.5mm}2 dec.\\
2047 &				f & 					27 febr. & 					14 apr. & 					23 maii & 					2 junii  & 					1 dec.\\
2048 &				e d & 				19 febr. & 					5 apr. & 					14 maii & 					24 maii & 					29 nov.\\
2049 &				c & 					3 mar. & 					18 apr. & 					27 maii & 					6 junii & 					28 nov.\\[0.5mm]
\arrayrulecolor{red} \thinhline \arrayrulecolor{black}
\rule{0pt}{3.5mm}2050 & 	\rule{0pt}{3.5mm}b & 	\rule{0pt}{3.5mm}23 febr. & 	\rule{0pt}{3.5mm}10 apr. & 	\rule{0pt}{3.5mm}19 maii & 		\rule{0pt}{3.5mm}29 maii & 		\rule{0pt}{3.5mm}27 nov.\\
2051 &				\textcolor{red}{A} & 		15 febr. & 					2 apr. & 					11 maii & 					21 maii & 					3 dec.\\
2052 &				g f & 					6 mar. & 					21 apr. & 					30 maii & 					9 junii & 					1 dec.\\
2053 &				e & 					19 febr. & 					6 apr. & 					15 maii & 					25 maii & 					30 nov.\\[0.5mm]
\arrayrulecolor{red} \thinhline \arrayrulecolor{black}
\rule{0pt}{3.5mm}2054 & 	\rule{0pt}{3.5mm}d & 	\rule{0pt}{3.5mm}11 febr. & 	\rule{0pt}{3.5mm}29 mar. & 	\rule{0pt}{3.5mm}7 maii & 		\rule{0pt}{3.5mm}17 maii & 		\rule{0pt}{3.5mm}29 nov.\\
2055 &				c & 					3 mar. & 					18 apr. & 					27 maii & 					6 junii & 					28 nov.\\
2056 &				b \textcolor{red}{A} & 	16 febr. & 					2 apr. & 					11 maii & 					21 maii & 					3 dec.\\
2057 &				g & 					7 mar. & 					22 apr. & 					31 maii & 					10 junii & 					2 dec.\\[0.5mm]
\arrayrulecolor{red} \thinhline \arrayrulecolor{black}
\rule{0pt}{3.5mm}2058 & 	\rule{0pt}{3.5mm}f & 		\rule{0pt}{3.5mm}27 febr. & 	\rule{0pt}{3.5mm}14 apr. & 	\rule{0pt}{3.5mm}23 maii & 		\rule{0pt}{3.5mm}2 junii & 		\rule{0pt}{3.5mm}1 dec.\\
2059 &				e & 					12 febr. & 					30 mar. & 					8 maii & 					18 maii & 				30 nov.\\
2060 &				d c & 				3 mar. & 					18 apr. & 					27 maii & 					6 junii & 					28 nov.\\
2061 &				b & 					23 febr. & 					10 apr. & 					19 maii & 					29 maii & 					27 nov.\\[0.5mm]
\thinhline
\end{longtable}

\pagebreak

\normalsize

% TABLE DU CALENDRIER
\centering {\large KALENDARIUM} \thispagestyle{empty} \markboth{Calendrier}{Calendrier} \nopagebreak \par \nopagebreak\vspace{5mm}\label{calendrier}
\setlength\LTleft{0pt}
\setlength\LTright{0pt}
\setlength{\tabcolsep}{5pt}
\renewcommand{\arraystretch}{1.4}
\fontsize{8}{8}\selectfont
\begin{longtable}{>{\centering}p{0.025\textwidth}|>{\raggedleft}p{0.025\textwidth}|>{\raggedright\arraybackslash}p{0.85\textwidth}}
\boldhline
\multirow{1.5}{*}{\STAB{\rotatebox[origin=c]{90}{{\footnotesize \textcolor{red}{L.D.}}}}} & \multirow{1.5}{*}{\STAB{\rotatebox[origin=c]{90}{{\footnotesize \textcolor{red}{Jour}}}}} &  \multicolumn{1}{c}{\multirow{1.75}{*}{{\footnotesize \textcolor{red}{Mois}}}} \\[8.5pt]
\boldhline
\null & \null & \null\\[2pt]
\endfirsthead
\boldhline
\multirow{1.5}{*}{\STAB{\rotatebox[origin=c]{90}{{\footnotesize \textcolor{red}{L.D.}}}}} & \multirow{1.5}{*}{\STAB{\rotatebox[origin=c]{90}{{\footnotesize \textcolor{red}{Jour}}}}} &  \multicolumn{1}{c}{\multirow{1.75}{*}{{\footnotesize \textcolor{red}{Mois}}}} \\[8.5pt]
\boldhline
\null & \null & \null\\[2pt]
\endhead
\endfoot
\endlastfoot
% DEBUT CALENDRIER
\null & \null & \null\\[1pt] \null & \null & \multicolumn{1}{c}{{\normalsize \textcolor{red}{Janvier}}}\\[5pt]\textcolor{red}{A} & 1 & \setlength{\hangindent}{10pt}OCTAVE DE LA NATIVITÉ : SAINTE MARIE MÈRE DE DIEU, \textcolor{red}{Solennité.}\\
b & 2 & \setlength{\hangindent}{10pt}Ss. Basile le Grand et Grégoire de Nazianze, évêques et docteurs de l'Église, \textcolor{red}{Mémoire~obligatoire.}\\
c & 3 & \setlength{\hangindent}{10pt}\textit{Saint Nom de Jésus}, \textcolor{red}{Mémoire~facultative.}\\
\null & \null & \textcolor{red}{En France :} \setlength{\hangindent}{10pt}\textit{Ste Geneviève, vierge}, \textcolor{red}{Mémoire~facultative.}\\
\null & \null & \textcolor{red}{Au Luxembourg :} \setlength{\hangindent}{10pt}\textit{Ste Irmine, religieuse}, \textcolor{red}{Mémoire~facultative.}\\
\null & \null & \textcolor{red}{En Afrique du Nord :} \setlength{\hangindent}{10pt}\textit{S. Fulgence, évêque}, \textcolor{red}{Mémoire~facultative.}\\
d & 4 & \null\\
e & 5 & \textcolor{red}{En Afrique du Nord :} \setlength{\hangindent}{10pt}\textit{Ss. Longin, Eugène et Vindémial,  évêques}, \textcolor{red}{Mémoire~facultative.}\\
f & 6 & \setlength{\hangindent}{10pt}ÉPIPHANIE DU SEIGNEUR, \textcolor{red}{Solennité.}\\
g & 7 & \setlength{\hangindent}{10pt}\textit{S. Raymond de Penyafort, prêtre}, \textcolor{red}{Mémoire~facultative.}\\
\null & \null & \textcolor{red}{Au Canada :} \setlength{\hangindent}{10pt}S. André Bessette, religieux, \textcolor{red}{Mémoire~obligatoire.}\\
\textcolor{red}{A} & 8 & \textcolor{red}{En Afrique du Nord :} \setlength{\hangindent}{10pt}\textit{Ss. Quodvultdeus et Deogratias, évêques}, \textcolor{red}{Mémoire~facultative.}\\
b & 9 & \null\\
c & 10 & \null\\
d & 11 & \textcolor{red}{En Afrique du Nord :} \setlength{\hangindent}{10pt}Ss. Victor I\textsuperscript{er}, Miltiade et Gélase I\textsuperscript{er},  papes, \textcolor{red}{Mémoire~obligatoire.}\\
e & 12 & \textcolor{red}{Au Canada :} \setlength{\hangindent}{10pt}Ste Marguerite Bourgeoys, vierge, \textcolor{red}{Mémoire~obligatoire.}\\
f & 13 & \setlength{\hangindent}{10pt}\textit{S. Hilaire, évêque et docteur de l'Église}, \textcolor{red}{Mémoire~facultative.}\\
g & 14 & \null\\
\textcolor{red}{A} & 15 & \textcolor{red}{En France :} \setlength{\hangindent}{10pt}\textit{S. Remi, évêque}, \textcolor{red}{Mémoire~facultative.}\\
c & 17 & \setlength{\hangindent}{10pt}S. Antoine, abbé, \textcolor{red}{Mémoire~obligatoire.}\\
d & 18 & \null\\
e & 19 & \null\\
f & 20 & \setlength{\hangindent}{10pt}\textit{S. Fabien, pape et martyr}, \textcolor{red}{Mémoire~facultative.}\\
\null & \null & \setlength{\hangindent}{10pt}\textit{S. Sébastien, martyr}, \textcolor{red}{Mémoire~facultative.}\\
g & 21 & \setlength{\hangindent}{10pt}Ste Agnès, vierge et martyre, \textcolor{red}{Mémoire~obligatoire.}\\
\textcolor{red}{A} & 22 & \setlength{\hangindent}{10pt}\textit{S. Vincent, diacre et martyr}, \textcolor{red}{Mémoire~facultative.}\\
b & 23 & \null\\
c & 24 & \setlength{\hangindent}{10pt}S. François de Sales, évêque et docteur de l'Église, \textcolor{red}{Mémoire~obligatoire.}\\
d & 25 & \setlength{\hangindent}{10pt}CONVERSION DE S. PAUL, APÔTRE, \textcolor{red}{Fête.}\\
e & 26 & \setlength{\hangindent}{10pt}Ss. Timothée et Tite, évêques, \textcolor{red}{Mémoire~obligatoire.}\\
f & 27 & \setlength{\hangindent}{10pt}\textit{Ste Angèle Merici, vierge}, \textcolor{red}{Mémoire~facultative.}\\
g & 28 & \setlength{\hangindent}{10pt}S. Thomas d'Aquin, prêtre et docteur  de l'Église, \textcolor{red}{Mémoire~obligatoire.}\\
b & 30 & \null\\
c & 31 & \setlength{\hangindent}{10pt}S. Jean Bosco, prêtre, \textcolor{red}{Mémoire~obligatoire.}\\

\null & \null & \null\\[1pt] \null & \null & \multicolumn{1}{c}{{\normalsize \textcolor{red}{Février}}}\\[5pt]d & 1 & \null\\
e & 2 & \setlength{\hangindent}{10pt}PRÉSENTATION DU SEIGNEUR AU TEMPLE, \textcolor{red}{Fête.}\\
f & 3 & \setlength{\hangindent}{10pt}\textit{S. Blaise, évêque et martyr}, \textcolor{red}{Mémoire~facultative.}\\
\null & \null & \setlength{\hangindent}{10pt}\textit{S. Anschaire, évêque}, \textcolor{red}{Mémoire~facultative.}\\
g & 4 & \textcolor{red}{En Afrique du Nord :} \setlength{\hangindent}{10pt}\textit{Ste Célérina et ses compagnons, martyrs}, \textcolor{red}{Mémoire~facultative.}\\
\textcolor{red}{A} & 5 & \setlength{\hangindent}{10pt}Ste Agathe, vierge et martyre, \textcolor{red}{Mémoire~obligatoire.}\\
b & 6 & \setlength{\hangindent}{10pt}S. Paul Miki et ses compagnons, martyrs, \textcolor{red}{Mémoire~obligatoire.}\\
\null & \null & \textcolor{red}{En Belgique :} \setlength{\hangindent}{10pt}S. Amand, évêque, \textcolor{red}{Mémoire~obligatoire.}\\
c & 7 & \textcolor{red}{En Belgique :} \setlength{\hangindent}{10pt}S. Paul Miki et ses compagnons, martyrs, \textcolor{red}{Mémoire~obligatoire.}\\
d & 8 & \setlength{\hangindent}{10pt}\textit{S. Jérôme Émilien}, \textcolor{red}{Mémoire~facultative.}\\
\null & \null & \setlength{\hangindent}{10pt}\textit{Ste Joséphine Bakhita, vierge}, \textcolor{red}{Mémoire~facultative.}\\
e & 9 & \null\\
f & 10 & \setlength{\hangindent}{10pt}Ste Scholastique, vierge, \textcolor{red}{Mémoire~obligatoire.}\\
g & 11 & \setlength{\hangindent}{10pt}\textit{Bienheureuse Vierge Marie de Lourdes}, \textcolor{red}{Mémoire~facultative.}\\
\textcolor{red}{A} & 12 & \null\\
b & 13 & \null\\
c & 14 & \setlength{\hangindent}{10pt}Ss. Cyrille, moine, et Méthode, évêque, \textcolor{red}{Mémoire~obligatoire.}\\
d & 15 & \null\\
e & 16 & \null\\
f & 17 & \setlength{\hangindent}{10pt}\textit{Les sept saints fondateurs de l'Ordre des Servites de Marie}, \textcolor{red}{Mémoire~facultative.}\\
g & 18 & \textcolor{red}{En France :} \setlength{\hangindent}{10pt}\textit{Ste Bernadette Soubirous, vierge}, \textcolor{red}{Mémoire~facultative.}\\
\textcolor{red}{A} & 19 & \null\\
b & 20 & \null\\
c & 21 & \setlength{\hangindent}{10pt}\textit{S. Pierre Damien, évêque et docteur de l'Église}, \textcolor{red}{Mémoire~facultative.}\\
d & 22 & \setlength{\hangindent}{10pt}CHAIRE DE S. PIERRE, APÔTRE, \textcolor{red}{Fête.}\\
e & 23 & \setlength{\hangindent}{10pt}S. Polycarpe, évêque et martyr, \textcolor{red}{Mémoire~obligatoire.}\\
f & 24 & \null\\
g & 25 & \null\\
\textcolor{red}{A} & 26 & \null\\
b & 27 & \setlength{\hangindent}{10pt}\textit{S. Grégoire de Narek, abbé et docteur de l'Église}, \textcolor{red}{Mémoire~facultative.}\\
c & 28 & \null\\
\null & \null & \null\\[1pt] \null & \null & \multicolumn{1}{c}{{\normalsize \textcolor{red}{Mars}}}\\[5pt]d & 1 & \null\\
e & 2 & \null\\
f & 3 & \null\\
g & 4 & \setlength{\hangindent}{10pt}\textit{S. Casimir}, \textcolor{red}{Mémoire~facultative.}\\
\textcolor{red}{A} & 5 & \null\\
b & 6 & \null\\
c & 7 & \setlength{\hangindent}{10pt}Stes Perpétue et Félicité, martyres, \textcolor{red}{Mémoire~obligatoire.}\\
\null & \null & \textcolor{red}{En Afrique du Nord :} \setlength{\hangindent}{10pt}STES PERPÉTUE ET FÉLICITÉ ET LEURS COMPAGNONS, MARTYRS, \textcolor{red}{Fête.}\\
d & 8 & \setlength{\hangindent}{10pt}\textit{S. Jean de Dieu, religieux}, \textcolor{red}{Mémoire~facultative.}\\
e & 9 & \setlength{\hangindent}{10pt}\textit{Ste Françoise Romaine, religieuse}, \textcolor{red}{Mémoire~facultative.}\\
f & 10 & \null\\
g & 11 & \null\\
\textcolor{red}{A} & 12 & \null\\
b & 13 & \null\\
c & 14 & \null\\
d & 15 & \null\\
e & 16 & \null\\
f & 17 & \setlength{\hangindent}{10pt}\textit{S. Patrick, évêque}, \textcolor{red}{Mémoire~facultative.}\\
g & 18 & \setlength{\hangindent}{10pt}\textit{S. Cyrille de Jérusalem, évêque et docteur  de l'Église}, \textcolor{red}{Mémoire~facultative.}\\
\textcolor{red}{A} & 19 & \setlength{\hangindent}{10pt}S. JOSEPH, ÉPOUX DE LA BIENHEUREUSE VIERGE MARIE, \textcolor{red}{Solennité.}\\
b & 20 & \null\\
c & 21 & \null\\
d & 22 & \null\\
e & 23 & \setlength{\hangindent}{10pt}\textit{S. Turibio de Mogrovejo, évêque}, \textcolor{red}{Mémoire~facultative.}\\
f & 24 & \null\\
g & 25 & \setlength{\hangindent}{10pt}ANNONCIATION DU SEIGNEUR, \textcolor{red}{Solennité.}\\
\textcolor{red}{A} & 26 & \null\\
b & 27 & \null\\
c & 28 & \null\\
d & 29 & \null\\
e & 30 & \null\\
f & 31 & \null\\
\null & \null & \null\\[1pt] \null & \null & \multicolumn{1}{c}{{\normalsize \textcolor{red}{Avril}}}\\[5pt]g & 1 & \null\\
\textcolor{red}{A} & 2 & \setlength{\hangindent}{10pt}\textit{S. François de Paule, ermite}, \textcolor{red}{Mémoire~facultative.}\\
b & 3 & \null\\
c & 4 & \setlength{\hangindent}{10pt}\textit{S. Isidore, évêque et docteur de l'Église}, \textcolor{red}{Mémoire~facultative.}\\
d & 5 & \setlength{\hangindent}{10pt}\textit{S. Vincent Ferrier, prêtre}, \textcolor{red}{Mémoire~facultative.}\\
e & 6 & \null\\
f & 7 & \setlength{\hangindent}{10pt}S. Jean-Baptiste de La Salle, prêtre, \textcolor{red}{Mémoire~obligatoire.}\\
g & 8 & \null\\
\textcolor{red}{A} & 9 & \null\\
b & 10 & \null\\
c & 11 & \setlength{\hangindent}{10pt}S. Stanislas, évêque et martyr, \textcolor{red}{Mémoire~obligatoire.}\\
d & 12 & \null\\
e & 13 & \setlength{\hangindent}{10pt}\textit{S. Martin I\textsuperscript{er}, pape et martyr}, \textcolor{red}{Mémoire~facultative.}\\
f & 14 & \null\\
g & 15 & \null\\
\textcolor{red}{A} & 16 & \null\\
b & 17 & \textcolor{red}{Au Canada :} \setlength{\hangindent}{10pt}Ste Kateri Tekakwitha, vierge, \textcolor{red}{Mémoire~obligatoire.}\\
c & 18 & \textcolor{red}{Au Canada :} \setlength{\hangindent}{10pt}\textit{Bienheureuse Marie-Anne Blondin, vierge}, \textcolor{red}{Mémoire~facultative.}\\
d & 19 & \null\\
e & 20 & \null\\
f & 21 & \setlength{\hangindent}{10pt}\textit{S. Anselme, évêque et docteur de l'Église}, \textcolor{red}{Mémoire~facultative.}\\
g & 22 & \null\\
\textcolor{red}{A} & 23 & \setlength{\hangindent}{10pt}\textit{S. Georges, martyr}, \textcolor{red}{Mémoire~facultative.}\\
\null & \null & \setlength{\hangindent}{10pt}\textit{S. Adalbert, évêque et martyr}, \textcolor{red}{Mémoire~facultative.}\\
b & 24 & \setlength{\hangindent}{10pt}\textit{S. Fidèle de Sigmaringen, prêtre et martyr}, \textcolor{red}{Mémoire~facultative.}\\
c & 25 & \setlength{\hangindent}{10pt}S. MARC, ÉVANGÉLISTE, \textcolor{red}{Fête.}\\
d & 26 & \textcolor{red}{Au Canada :} \setlength{\hangindent}{10pt}\textit{Bienheureuse Vierge Marie du Bon Conseil}, \textcolor{red}{Mémoire~facultative.}\\
f & 28 & \setlength{\hangindent}{10pt}\textit{S. Pierre Chanel, prêtre et martyr}, \textcolor{red}{Mémoire~facultative.}\\
\null & \null & \setlength{\hangindent}{10pt}\textit{S. Louis-Marie Grignion de Montfort, prêtre}, \textcolor{red}{Mémoire~facultative.}\\
g & 29 & \setlength{\hangindent}{10pt}Ste Catherine de Sienne, vierge et docteur de l'Église, \textcolor{red}{Mémoire~obligatoire.}\\
\textcolor{red}{A} & 30 & \setlength{\hangindent}{10pt}\textit{S. Pie V, pape}, \textcolor{red}{Mémoire~facultative.}\\
\null & \null & \textcolor{red}{En Afrique du Nord :} \setlength{\hangindent}{10pt}BIENHEUREUSE VIERGE MARIE, NOTRE-DAME D'AFRIQUE, \textcolor{red}{Solennité.}\\
\null & \null & \textcolor{red}{Au Canada :} \setlength{\hangindent}{10pt}Ste Marie de l'Incarnation, \textcolor{red}{Mémoire~obligatoire.}\\
\null & \null & \setlength{\hangindent}{10pt}Samedi après le quatrième dimanche de Pâques : \textcolor{red}{Au Luxembourg :} BIENHEUREUSE VIERGE MARIE, CONSOLATRICE DES AFFLIGÉS, \textcolor{red}{Solennité.}\\
\null & \null & \null\\[1pt] \null & \null & \multicolumn{1}{c}{{\normalsize \textcolor{red}{Mai}}}\\[5pt]b & 1 & \setlength{\hangindent}{10pt}\textit{S. Joseph travailleur}, \textcolor{red}{Mémoire~facultative.}\\
\null & \null & \textcolor{red}{En Afrique du Nord, au Canada :} \setlength{\hangindent}{10pt}\textit{S. Pie V, pape}, \textcolor{red}{Mémoire~facultative.}\\
c & 2 & \setlength{\hangindent}{10pt}S. Athanase, évêque et docteur de l'Église, \textcolor{red}{Mémoire~obligatoire.}\\
d & 3 & \setlength{\hangindent}{10pt}SS. PHILIPPE ET JACQUES, APÔTRES, \textcolor{red}{Fête.}\\
e & 4 & \textcolor{red}{Au Canada :} \setlength{\hangindent}{10pt}\textit{Bienheureuse Marie-Léonie Paradis, vierge}, \textcolor{red}{Mémoire~facultative.}\\
f & 5 & \null\\
g & 6 & \textcolor{red}{En Afrique du Nord :} \setlength{\hangindent}{10pt}\textit{Ss. Jacques, diacre, Marien, lecteur, et leurs compagnons, martyrs}, \textcolor{red}{Mémoire~facultative.}\\
\null & \null & \textcolor{red}{Au Canada :} \setlength{\hangindent}{10pt}S. François de Laval, évêque, \textcolor{red}{Mémoire~obligatoire.}\\
\textcolor{red}{A} & 7 & \null\\
b & 8 & \textcolor{red}{En Afrique du Nord :} \setlength{\hangindent}{10pt}\textit{Bx Pierre Claverie, évêque, et ses compagnons, religieux et martyrs}, \textcolor{red}{Mémoire~facultative.}\\
\null & \null & \textcolor{red}{Au Canada :} \setlength{\hangindent}{10pt}\textit{Bienheureuse Catherine de Saint-Augustin, vierge}, \textcolor{red}{Mémoire~facultative.}\\
c & 9 & \textcolor{red}{En France :} \setlength{\hangindent}{10pt}\textit{Ste Louise de Marillac}, \textcolor{red}{Mémoire~facultative.}\\
d & 10 & \setlength{\hangindent}{10pt}\textit{S. Jean d'Avila, prêtre et docteur de l'Église}, \textcolor{red}{Mémoire~facultative.}\\
\null & \null & \textcolor{red}{En Belgique :} \setlength{\hangindent}{10pt}S. Damien De Veuster, prêtre, \textcolor{red}{Mémoire~obligatoire.}\\
e & 11 & \null\\
f & 12 & \setlength{\hangindent}{10pt}\textit{Ss. Nérée et Achille, martyrs}, \textcolor{red}{Mémoire~facultative.}\\
\null & \null & \setlength{\hangindent}{10pt}\textit{S. Pancrace, martyr}, \textcolor{red}{Mémoire~facultative.}\\
g & 13 & \setlength{\hangindent}{10pt}\textit{Bienheureuse Vierge Marie de Fatima}, \textcolor{red}{Mémoire~facultative.}\\
\textcolor{red}{A} & 14 & \setlength{\hangindent}{10pt}S. MATTHIAS, APÔTRE, \textcolor{red}{Fête.}\\
c & 16 & \null\\
d & 17 & \null\\
e & 18 & \setlength{\hangindent}{10pt}\textit{S. Jean I\textsuperscript{er}, pape et martyr}, \textcolor{red}{Mémoire~facultative.}\\
f & 19 & \textcolor{red}{En France :} \setlength{\hangindent}{10pt}\textit{S. Yves, prêtre}, \textcolor{red}{Mémoire~facultative.}\\
g & 20 & \setlength{\hangindent}{10pt}\textit{S. Bernardin de Sienne, prêtre}, \textcolor{red}{Mémoire~facultative.}\\
\textcolor{red}{A} & 21 & \setlength{\hangindent}{10pt}\textit{S. Christophe Magallanès, prêtre, et ses compagnons, martyrs}, \textcolor{red}{Mémoire~facultative.}\\
\null & \null & \textcolor{red}{Au Canada :} \setlength{\hangindent}{10pt}\textit{S. Eugène de Mazenod, évêque}, \textcolor{red}{Mémoire~facultative.}\\
b & 22 & \setlength{\hangindent}{10pt}\textit{Ste Rita de Cascia, religieuse}, \textcolor{red}{Mémoire~facultative.}\\
c & 23 & \null\\
d & 24 & \textcolor{red}{Au Canada :} \setlength{\hangindent}{10pt}\textit{Bienheureux Louis-Zéphirin Moreau, évêque}, \textcolor{red}{Mémoire~facultative.}\\
e & 25 & \setlength{\hangindent}{10pt}\textit{S. Bède le Vénérable, prêtre et docteur de l'Église}, \textcolor{red}{Mémoire~facultative.}\\
\null & \null & \setlength{\hangindent}{10pt}\textit{S. Grégoire VII, pape}, \textcolor{red}{Mémoire~facultative.}\\
\null & \null & \setlength{\hangindent}{10pt}\textit{Ste Marie-Madeleine de Pazzi, vierge}, \textcolor{red}{Mémoire~facultative.}\\
f & 26 & \setlength{\hangindent}{10pt}S. Philippe Néri, prêtre, \textcolor{red}{Mémoire~obligatoire.}\\
g & 27 & \setlength{\hangindent}{10pt}\textit{S. Augustin de Cantorbéry, évêque}, \textcolor{red}{Mémoire~facultative.}\\
\textcolor{red}{A} & 28 & \null\\
b & 29 & \setlength{\hangindent}{10pt}\textit{S. Paul VI, pape}, \textcolor{red}{Mémoire~facultative.}\\
c & 30 & \textcolor{red}{En France :} \setlength{\hangindent}{10pt}Ste Jeanne d'Arc, vierge, \textcolor{red}{Mémoire~obligatoire.}\\
d & 31 & \setlength{\hangindent}{10pt}VISITATION DE LA BIENHEUREUSE VIERGE MARIE, \textcolor{red}{Fête.}\\
\null & \null & \setlength{\hangindent}{10pt}Jeudi après la Pentecôte : \textcolor{red}{Au Luxembourg :} NOTRE SEIGNEUR JÉSUS CHRIST, ÉTERNEL ET SOUVERAIN PRÊTRE, \textcolor{red}{Fête.}\\
%\null & \null & \null\\[1pt]
\null & \null & \multicolumn{1}{c}{{\normalsize \textcolor{red}{Juin}}}\\[5pt]e & 1 & \setlength{\hangindent}{10pt}S. Justin, martyr, \textcolor{red}{Mémoire~obligatoire.}\\
f & 2 & \setlength{\hangindent}{10pt}\textit{Ss. Marcellin et Pierre, martyrs}, \textcolor{red}{Mémoire~facultative.}\\
\null & \null & \textcolor{red}{En France :} \setlength{\hangindent}{10pt}\textit{Ss. Pothin, évêque, Blandine, vierge, et leurs compagnons, martyrs}, \textcolor{red}{Mémoire~facultative.}\\
g & 3 & \setlength{\hangindent}{10pt}S. Charles Lwanga et ses compagnons, martyrs, \textcolor{red}{Mémoire~obligatoire.}\\
\textcolor{red}{A} & 4 & \textcolor{red}{En France :} \setlength{\hangindent}{10pt}\textit{Ste Clotilde}, \textcolor{red}{Mémoire~facultative.}\\
\null & \null & \textcolor{red}{En Afrique du Nord :} \setlength{\hangindent}{10pt}S. Optat, évêque, \textcolor{red}{Mémoire~obligatoire.}\\
b & 5 & \setlength{\hangindent}{10pt}S. Boniface, évêque et martyr, \textcolor{red}{Mémoire~obligatoire.}\\
c & 6 & \setlength{\hangindent}{10pt}\textit{S. Norbert, évêque}, \textcolor{red}{Mémoire~facultative.}\\
d & 7 & \null\\
e & 8 & \null\\
f & 9 & \setlength{\hangindent}{10pt}\textit{S. Éphrem, diacre et docteur de l'Église}, \textcolor{red}{Mémoire~facultative.}\\
g & 10 & \textcolor{red}{En Belgique :} \setlength{\hangindent}{10pt}\textit{Bx Édouard Poppe, prêtre}, \textcolor{red}{Mémoire~facultative.}\\
\textcolor{red}{A} & 11 & \setlength{\hangindent}{10pt}S. Barnabé, apôtre, \textcolor{red}{Mémoire~obligatoire.}\\
c & 13 & \setlength{\hangindent}{10pt}S. Antoine de Padoue, prêtre et docteur  de l'Église, \textcolor{red}{Mémoire~obligatoire.}\\
d & 14 & \null\\
e & 15 & \null\\
f & 16 & \null\\
g & 17 & \null\\
\textcolor{red}{A} & 18 & \null\\
b & 19 & \setlength{\hangindent}{10pt}\textit{S. Romuald, abbé}, \textcolor{red}{Mémoire~facultative.}\\
c & 20 & \null\\
d & 21 & \setlength{\hangindent}{10pt}S. Louis de Gonzague, religieux, \textcolor{red}{Mémoire~obligatoire.}\\
e & 22 & \setlength{\hangindent}{10pt}\textit{S. Paulin de Nole, évêque}, \textcolor{red}{Mémoire~facultative.}\\
\null & \null & \setlength{\hangindent}{10pt}\textit{Ss. Jean Fisher, évêque, et Thomas More, martyrs}, \textcolor{red}{Mémoire~facultative.}\\
f & 23 & \null\\
g & 24 & \setlength{\hangindent}{10pt}NATIVITÉ DE SAINT JEAN BAPTISTE, \textcolor{red}{Solennité.}\\
\textcolor{red}{A} & 25 & \null\\
b & 26 & \null\\
c & 27 & \setlength{\hangindent}{10pt}\textit{S. Cyrille d'Alexandrie, évêque et docteur  de l'Église}, \textcolor{red}{Mémoire~facultative.}\\
\null & \null & \textcolor{red}{Au Canada :} \setlength{\hangindent}{10pt}\textit{Bienheureux Nykyta Budka et Vasyl Velychkovsky, évêques et martyrs}, \textcolor{red}{Mémoire~facultative.}\\
d & 28 & \setlength{\hangindent}{10pt}S. Irénée, évêque et martyr, \textcolor{red}{Mémoire~obligatoire.}\\
e & 29 & \setlength{\hangindent}{10pt}SS. PIERRE ET PAUL, APÔTRES, \textcolor{red}{Solennité.}\\
f & 30 & \setlength{\hangindent}{10pt}\textit{Les saints premiers martyrs de l'Église de Rome}, \textcolor{red}{Mémoire~facultative.}\\


\null & \null & \null\\[1pt]
\null & \null & \null\\[1pt]
\null & \null & \null\\[1pt]
\null & \null & \null\\[1pt]
\pagebreak \null & \null & \multicolumn{1}{c}{{\normalsize \textcolor{red}{Juillet}}}\\[5pt]g & 1 & \textcolor{red}{Au Canada :} \setlength{\hangindent}{10pt}\textit{Fête du Canada}, \textcolor{red}{Mémoire~facultative.}\\
\textcolor{red}{A} & 2 & \null\\
b & 3 & \setlength{\hangindent}{10pt}S. THOMAS, APÔTRE, \textcolor{red}{Fête.}\\
c & 4 & \setlength{\hangindent}{10pt}\textit{Ste Élisabeth de Portugal}, \textcolor{red}{Mémoire~facultative.}\\
d & 5 & \setlength{\hangindent}{10pt}\textit{S. Antoine-Marie Zaccaria, prêtre}, \textcolor{red}{Mémoire~facultative.}\\
e & 6 & \setlength{\hangindent}{10pt}\textit{Ste Maria Goretti, vierge et martyre}, \textcolor{red}{Mémoire~facultative.}\\
f & 7 & \null\\
g & 8 & \null\\
\textcolor{red}{A} & 9 & \setlength{\hangindent}{10pt}\textit{S. Augustin Zhao Rong, prêtre, et ses compagnons, martyrs}, \textcolor{red}{Mémoire~facultative.}\\
b & 10 & \textcolor{red}{En Afrique du Nord :} \setlength{\hangindent}{10pt}\textit{Ste Marcienne, vierge et martyre}, \textcolor{red}{Mémoire~facultative.}\\
c & 11 & \setlength{\hangindent}{10pt}S. Benoît, abbé, \textcolor{red}{Mémoire~obligatoire.}\\
d & 12 & \null\\
e & 13 & \setlength{\hangindent}{10pt}\textit{S. Henri}, \textcolor{red}{Mémoire~facultative.}\\
\null & \null & \textcolor{red}{Au Luxembourg :} \setlength{\hangindent}{10pt}S. Henri et Ste Cunégonde, \textcolor{red}{Mémoire~obligatoire.}\\
f & 14 & \setlength{\hangindent}{10pt}\textit{S. Camille de Lellis, prêtre}, \textcolor{red}{Mémoire~facultative.}\\
g & 15 & \setlength{\hangindent}{10pt}S. Bonaventure, évêque et docteur  de l'Église, \textcolor{red}{Mémoire~obligatoire.}\\
\textcolor{red}{A} & 16 & \setlength{\hangindent}{10pt}\textit{Bienheureuse Vierge Marie du Mont Carmel}, \textcolor{red}{Mémoire~facultative.}\\
b & 17 & \textcolor{red}{En Afrique du Nord :} \setlength{\hangindent}{10pt}S. Spérat et ses compagnons, martyrs, \textcolor{red}{Mémoire~obligatoire.}\\
c & 18 & \null\\
d & 19 & \null\\
e & 20 & \setlength{\hangindent}{10pt}\textit{S. Apollinaire, évêque et martyr}, \textcolor{red}{Mémoire~facultative.}\\
f & 21 & \setlength{\hangindent}{10pt}\textit{S. Laurent de Brindisi, prêtre et docteur  de l'Église}, \textcolor{red}{Mémoire~facultative.}\\
g & 22 & \setlength{\hangindent}{10pt}STE MARIE MADELEINE, \textcolor{red}{Fête.}\\
\textcolor{red}{A} & 23 & \setlength{\hangindent}{10pt}\textit{Ste Brigitte, religieuse}, \textcolor{red}{Mémoire~facultative.}\\
b & 24 & \setlength{\hangindent}{10pt}\textit{S. Charbel Makhlouf, prêtre}, \textcolor{red}{Mémoire~facultative.}\\
c & 25 & \setlength{\hangindent}{10pt}S. JACQUES, APÔTRE, \textcolor{red}{Fête.}\\
d & 26 & \setlength{\hangindent}{10pt}Ss. Joachim et Anne, parents de la bienheureuse Vierge Marie, \textcolor{red}{Mémoire~obligatoire.}\\
e & 27 & \null\\
f & 28 & \null\\
g & 29 & \setlength{\hangindent}{10pt}Stes Marthe, Marie et S. Lazare, \textcolor{red}{Mémoire~obligatoire.}\\
\textcolor{red}{A} & 30 & \setlength{\hangindent}{10pt}\textit{S. Pierre Chrysologue, évêque et docteur de l'Église}, \textcolor{red}{Mémoire~facultative.}\\
b & 31 & \setlength{\hangindent}{10pt}S. Ignace de Loyola, prêtre, \textcolor{red}{Mémoire~obligatoire.}\\
\null & \null & \null\\[1pt] \null & \null & \multicolumn{1}{c}{\normalsize {\textcolor{red}{Août}}}\\[5pt]c & 1 & \setlength{\hangindent}{10pt}S. Alphonse-Marie de Liguori, évêque et docteur de l'Église, \textcolor{red}{Mémoire~obligatoire.}\\
d & 2 & \setlength{\hangindent}{10pt}\textit{S. Eusèbe de Verceil, éveque}, \textcolor{red}{Mémoire~facultative.}\\
\null & \null & \setlength{\hangindent}{10pt}\textit{S. Pierre-Julien Eymard, prêtre}, \textcolor{red}{Mémoire~facultative.}\\
f & 4 & \setlength{\hangindent}{10pt}S. Jean-Marie Vianney, prêtre, \textcolor{red}{Mémoire~obligatoire.}\\
g & 5 & \setlength{\hangindent}{10pt}\textit{Dédicace de la basilique Sainte-Marie Majeure}, \textcolor{red}{Mémoire~facultative.}\\
\null & \null & \textcolor{red}{Au Canada :} \setlength{\hangindent}{10pt}\textit{Bienheureux Frédéric Janssoone, prêtre}, \textcolor{red}{Mémoire~facultative.}\\
\textcolor{red}{A} & 6 & \setlength{\hangindent}{10pt}TRANSFIGURATION DU SEIGNEUR, \textcolor{red}{Fête.}\\
b & 7 & \setlength{\hangindent}{10pt}\textit{S. Sixte II, pape, et ses compagnons, martyrs}, \textcolor{red}{Mémoire~facultative.}\\
\null & \null & \setlength{\hangindent}{10pt}\textit{S. Gaétan, prêtre}, \textcolor{red}{Mémoire~facultative.}\\
\null & \null & \textcolor{red}{En Belgique :} \setlength{\hangindent}{10pt}\textit{Ste Julienne du Mont-Cornillon, vierge}, \textcolor{red}{Mémoire~facultative.}\\
c & 8 & \setlength{\hangindent}{10pt}S. Dominique, prêtre, \textcolor{red}{Mémoire~obligatoire.}\\
d & 9 & \setlength{\hangindent}{10pt}Ste Thérèse-Bénédicte de la Croix, vierge et martyre, \textcolor{red}{Mémoire~obligatoire.}\\
e & 10 & \setlength{\hangindent}{10pt}S. LAURENT, DIACRE ET MARTYR, \textcolor{red}{Fête.}\\
f & 11 & \setlength{\hangindent}{10pt}Ste Claire, vierge, \textcolor{red}{Mémoire~obligatoire.}\\
\null & \null & \textcolor{red}{Au Luxembourg :} \setlength{\hangindent}{10pt}\textit{B. Schecelin ou Schetzel, ermite}, \textcolor{red}{Mémoire~facultative.}\\
g & 12 & \setlength{\hangindent}{10pt}\textit{Ste Jeanne-Françoise de Chantal, religieuse}, \textcolor{red}{Mémoire~facultative.}\\
\textcolor{red}{A} & 13 & \setlength{\hangindent}{10pt}\textit{Ss. Pontien, pape, et Hippolyte, prêtre, martyrs}, \textcolor{red}{Mémoire~facultative.}\\
b & 14 & \setlength{\hangindent}{10pt}S. Maximilien-Marie Kolbe, prêtre et martyr, \textcolor{red}{Mémoire~obligatoire.}\\
c & 15 & \setlength{\hangindent}{10pt}ASSOMPTION DE LA BIENHEUREUSE VIERGE MARIE, \textcolor{red}{Solennité.}\\
d & 16 & \setlength{\hangindent}{10pt}\textit{S. Étienne de Hongrie}, \textcolor{red}{Mémoire~facultative.}\\
f & 18 & \null\\
g & 19 & \setlength{\hangindent}{10pt}\textit{S. Jean Eudes, prêtre}, \textcolor{red}{Mémoire~facultative.}\\
\textcolor{red}{A} & 20 & \setlength{\hangindent}{10pt}S. Bernard, abbé et docteur de l'Église, \textcolor{red}{Mémoire~obligatoire.}\\
b & 21 & \setlength{\hangindent}{10pt}S. Pie X, pape, \textcolor{red}{Mémoire~obligatoire.}\\
c & 22 & \setlength{\hangindent}{10pt}Bienheureuse Vierge Marie Reine, \textcolor{red}{Mémoire~obligatoire.}\\
d & 23 & \setlength{\hangindent}{10pt}\textit{Ste Rose de Lima, vierge}, \textcolor{red}{Mémoire~facultative.}\\
\null & \null & \textcolor{red}{En Afrique du Nord :} \setlength{\hangindent}{10pt}\textit{Ste Émilie de Vialar, vierge}, \textcolor{red}{Mémoire~facultative.}\\
e & 24 & \setlength{\hangindent}{10pt}S. BARTHÉLÉMY, APÔTRE, \textcolor{red}{Fête.}\\
f & 25 & \setlength{\hangindent}{10pt}\textit{S. Louis}, \textcolor{red}{Mémoire~facultative.}\\
\null & \null & \setlength{\hangindent}{10pt}\textit{S. Joseph de Calasanz, prêtre}, \textcolor{red}{Mémoire~facultative.}\\
g & 26 & \textcolor{red}{En France :} \setlength{\hangindent}{10pt}\textit{S. Césaire d'Arles, évêque}, \textcolor{red}{Mémoire~facultative.}\\
\null & \null & \textcolor{red}{En France :} \setlength{\hangindent}{10pt}\textit{S. Joseph de Calasanz, prêtre}, \textcolor{red}{Mémoire~facultative.}\\
\textcolor{red}{A} & 27 & \setlength{\hangindent}{10pt}Ste Monique, mère de famille, \textcolor{red}{Mémoire~obligatoire.}\\
b & 28 & \setlength{\hangindent}{10pt}S. Augustin, évêque et docteur de l'Église, \textcolor{red}{Mémoire~obligatoire.}\\
c & 29 & \setlength{\hangindent}{10pt}Martyre de S. Jean Baptiste, \textcolor{red}{Mémoire~obligatoire.}\\
\null & \null & \textcolor{red}{Au Luxembourg :} \setlength{\hangindent}{10pt}DÉDICACE DE LA CATHÉDRALE, \textcolor{red}{Fête.}\\
d & 30 & \textcolor{red}{En Afrique du Nord :} \setlength{\hangindent}{10pt}Ss. Alype et Possidius, évêques, \textcolor{red}{Mémoire~obligatoire.}\\
\null & \null & \textcolor{red}{Au Luxembourg :} \setlength{\hangindent}{10pt}Martyre de S. Jean Baptiste, \textcolor{red}{Mémoire~obligatoire.}\\
e & 31 & \textcolor{red}{En Belgique :} \setlength{\hangindent}{10pt}\textit{La bienheureuse Vierge Marie Médiatrice}, \textcolor{red}{Mémoire~facultative.}\\
\null & \null & \null\\[1pt]
\null & \null & \null\\[1pt] \null & \null & \multicolumn{1}{c}{{\normalsize \textcolor{red}{Septembre}}}\\[5pt]
\null & \null & \setlength{\hangindent}{10pt}Premier lundi de septembre : \textcolor{red}{Au Canada :} \textit{Fête du travail}, \textcolor{red}{Mémoire~facultative.}\\
f & 1 & \null\\
g & 2 & \textcolor{red}{Au Canada :} \setlength{\hangindent}{10pt}\textit{Bienheureux André Grasset, prêtre et martyr}, \textcolor{red}{Mémoire~facultative.}\\
\textcolor{red}{A} & 3 & \setlength{\hangindent}{10pt}S. Grégoire le Grand, pape et docteur de l'Église, \textcolor{red}{Mémoire~obligatoire.}\\
b & 4 & \textcolor{red}{Au Canada :} \setlength{\hangindent}{10pt}\textit{Bienheureuse Dina Bélanger, vierge}, \textcolor{red}{Mémoire~facultative.}\\
c & 5 & \null\\
d & 6 & \null\\
e & 7 & \null\\
f & 8 & \setlength{\hangindent}{10pt}NATIVITÉ DE LA BIENHEUREUSE VIERGE MARIE, \textcolor{red}{Fête.}\\
g & 9 & \setlength{\hangindent}{10pt}\textit{S. Pierre Claver, prêtre}, \textcolor{red}{Mémoire~facultative.}\\
\textcolor{red}{A} & 10 & \textcolor{red}{En Afrique du Nord :} \setlength{\hangindent}{10pt}\textit{S. Némésianus et ses compagnons, martyrs}, \textcolor{red}{Mémoire~facultative.}\\
b & 11 & \null\\
c & 12 & \setlength{\hangindent}{10pt}\textit{Saint Nom de Marie}, \textcolor{red}{Mémoire~facultative.}\\
\null & \null & \textcolor{red}{En Afrique du Nord :} \setlength{\hangindent}{10pt}\textit{S. Marcellin, martyr}, \textcolor{red}{Mémoire~facultative.}\\
d & 13 & \setlength{\hangindent}{10pt}S. Jean Chrysostome, évêque et docteur de l'Église, \textcolor{red}{Mémoire~obligatoire.}\\
e & 14 & \setlength{\hangindent}{10pt}LA CROIX GLORIEUSE, \textcolor{red}{Fête.}\\
f & 15 & \setlength{\hangindent}{10pt}Bienheureuse Vierge Marie des Douleurs, \textcolor{red}{Mémoire~obligatoire.}\\
g & 16 & \setlength{\hangindent}{10pt}Ss. Corneille, pape, et Cyprien, évêque, martyrs, \textcolor{red}{Mémoire~obligatoire.}\\
\null & \null & \textcolor{red}{En Afrique du Nord :} \setlength{\hangindent}{10pt}S. CYPRIEN, ÉVÊQUE ET MARTYR, \textcolor{red}{Solennité.}\\
\textcolor{red}{A} & 17 & \setlength{\hangindent}{10pt}\textit{S. Robert Bellarmin, évêque et docteur de l'Église}, \textcolor{red}{Mémoire~facultative.}\\
\null & \null & \setlength{\hangindent}{10pt}\textit{Ste Hildegarde de Bingen, vierge et docteur de l'Église}, \textcolor{red}{Mémoire~facultative.}\\
\null & \null & \textcolor{red}{En Belgique :} \setlength{\hangindent}{10pt}\textit{S. Lambert, évêque et martyr}, \textcolor{red}{Mémoire~facultative.}\\
b & 18 & \textcolor{red}{Au Luxembourg :} \setlength{\hangindent}{10pt}\textit{S. Lambert, évêque et martyr}, \textcolor{red}{Mémoire~facultative.}\\
\null & \null & \textcolor{red}{En Afrique du Nord :} \setlength{\hangindent}{10pt}S. Corneille, pape et martyr, \textcolor{red}{Mémoire~obligatoire.}\\
c & 19 & \setlength{\hangindent}{10pt}\textit{S. Janvier, évêque et martyr}, \textcolor{red}{Mémoire~facultative.}\\
\null & \null & \textcolor{red}{En France :} \setlength{\hangindent}{10pt}\textit{Bienheureuse Vierge Marie de La Salette}, \textcolor{red}{Mémoire~facultative.}\\
d & 20 & \setlength{\hangindent}{10pt}Ss. André Kim Tae-gon, prêtre, Paul Chong Ha-sang, et leurs compagnons, martyrs, \textcolor{red}{Mémoire~obligatoire.}\\
e & 21 & \setlength{\hangindent}{10pt}S. MATTHIEU, APÔTRE ET ÉVANGÉLISTE, \textcolor{red}{Fête.}\\
f & 22 & \textcolor{red}{En Suisse :} \setlength{\hangindent}{10pt}S. Maurice et ses compagnons, martyrs, \textcolor{red}{Mémoire~obligatoire.}\\
g & 23 & \setlength{\hangindent}{10pt}S. Pio de Pietrelcina, prêtre, \textcolor{red}{Mémoire~obligatoire.}\\
\textcolor{red}{A} & 24 & \textcolor{red}{Au Canada :} \setlength{\hangindent}{10pt}\textit{Bienheureuse Émilie Tavernier-Gamelin, religieuse}, \textcolor{red}{Mémoire~facultative.}\\
b & 25 & \textcolor{red}{En Suisse :} \setlength{\hangindent}{10pt}S. NICOLAS DE FLÜE, ERMITE, \textcolor{red}{Solennité.}\\
\null & \null & \textcolor{red}{Au Canada :} \setlength{\hangindent}{10pt}\textit{Ss. Côme et Damien, martyrs}, \textcolor{red}{Mémoire~facultative.}\\
c & 26 & \setlength{\hangindent}{10pt}\textit{Ss. Côme et Damien, martyrs}, \textcolor{red}{Mémoire~facultative.}\\
\null & \null & \textcolor{red}{Au Canada :} \setlength{\hangindent}{10pt}S. JEAN DE BRÉBEUF, S. ISAAC JOGUES, PRÊTRES, ET LEURS COMPAGNONS, MARTYRS, \textcolor{red}{Fête.}\\
d & 27 & \setlength{\hangindent}{10pt}S. Vincent de Paul, prêtre, \textcolor{red}{Mémoire~obligatoire.}\\
e & 28 & \setlength{\hangindent}{10pt}\textit{S. Venceslas, martyr}, \textcolor{red}{Mémoire~facultative.}\\
\null & \null & \setlength{\hangindent}{10pt}\textit{S. Laurent Ruiz et ses compagnons, martyrs}, \textcolor{red}{Mémoire~facultative.}\\
f & 29 & \setlength{\hangindent}{10pt}SS. MICHEL, GABRIEL ET RAPHAËL, ARCHANGES, \textcolor{red}{Fête.}\\
g & 30 & \setlength{\hangindent}{10pt}S. Jérôme, prêtre et docteur de l'Église, \textcolor{red}{Mémoire~obligatoire.}\\

\null & \null & \null\\[1pt] \null & \null & \multicolumn{1}{c}{{\normalsize \textcolor{red}{Octobre}}}\\[5pt]
\null & \null & \setlength{\hangindent}{10pt}Deuxième lundi d'octobre : \textcolor{red}{Au Canada :} \textit{Jour de l'action de grâce}, \textcolor{red}{Mémoire~facultative.}\\
\textcolor{red}{A} & 1 & \setlength{\hangindent}{10pt}Ste Thérèse de l'Enfant-Jésus, vierge et docteur de l'Église, \textcolor{red}{Mémoire~obligatoire.}\\
b & 2 & \setlength{\hangindent}{10pt}Ss. anges gardiens, \textcolor{red}{Mémoire~obligatoire.}\\
c & 3 & \null\\
d & 4 & \setlength{\hangindent}{10pt}S. François d'Assise, \textcolor{red}{Mémoire~obligatoire.}\\
e & 5 & \setlength{\hangindent}{10pt}\textit{Ste Faustine Kowalska, vierge}, \textcolor{red}{Mémoire~facultative.}\\
f & 6 & \setlength{\hangindent}{10pt}\textit{S. Bruno, prêtre}, \textcolor{red}{Mémoire~facultative.}\\
\null & \null & \textcolor{red}{Au Canada :} \setlength{\hangindent}{10pt}\textit{Bienheureuse Marie-Rose Durocher, vierge}, \textcolor{red}{Mémoire~facultative.}\\
g & 7 & \setlength{\hangindent}{10pt}Bienheureuse Vierge Marie du Rosaire, \textcolor{red}{Mémoire~obligatoire.}\\
\textcolor{red}{A} & 8 & \null\\
b & 9 & \setlength{\hangindent}{10pt}\textit{S. Denis, évêque, et ses compagnons, martyrs}, \textcolor{red}{Mémoire~facultative.}\\
\null & \null & \setlength{\hangindent}{10pt}\textit{S. Jean Léonardi, prêtre}, \textcolor{red}{Mémoire~facultative.}\\
c & 10 & \null\\
d & 11 & \setlength{\hangindent}{10pt}\textit{S. Jean XXIII, pape}, \textcolor{red}{Mémoire~facultative.}\\
e & 12 & \null\\
f & 13 & \null\\
g & 14 & \setlength{\hangindent}{10pt}\textit{S. Calliste I\textsuperscript{er}, pape et martyr}, \textcolor{red}{Mémoire~facultative.}\\
\textcolor{red}{A} & 15 & \setlength{\hangindent}{10pt}Ste Thérèse de Jésus, vierge et docteur de l'Église, \textcolor{red}{Mémoire~obligatoire.}\\
b & 16 & \setlength{\hangindent}{10pt}\textit{Ste Edwige, religieuse}, \textcolor{red}{Mémoire~facultative.}\\
\null & \null & \setlength{\hangindent}{10pt}\textit{Ste Marguerite-Marie Alacoque, vierge}, \textcolor{red}{Mémoire~facultative.}\\
\null & \null & \textcolor{red}{Au Canada :} \setlength{\hangindent}{10pt}Ste Marie-Marguerite d'Youville, \textcolor{red}{Mémoire~obligatoire.}\\
c & 17 & \setlength{\hangindent}{10pt}S. Ignace d'Antioche, évêque et martyr, \textcolor{red}{Mémoire~obligatoire.}\\
d & 18 & \setlength{\hangindent}{10pt}S. LUC, ÉVANGÉLISTE, \textcolor{red}{Fête.}\\
e & 19 & \setlength{\hangindent}{10pt}\textit{Ss. Jean de Brébeuf et Isaac Jogues, prêtres, et leurs compagnons, martyrs}, \textcolor{red}{Mémoire~facultative.}\\
\null & \null & \setlength{\hangindent}{10pt}\textit{S. Paul de la Croix, prêtre}, \textcolor{red}{Mémoire~facultative.}\\
f & 20 & \textcolor{red}{Au Canada :} \setlength{\hangindent}{10pt}\textit{Ste Edwige, religieuse}, \textcolor{red}{Mémoire~facultative.}\\
\null & \null & \textcolor{red}{Au Canada :} \setlength{\hangindent}{10pt}\textit{Ste Marguerite-Marie Alacoque, vierge}, \textcolor{red}{Mémoire~facultative.}\\
g & 21 & \null\\
\textcolor{red}{A} & 22 & \setlength{\hangindent}{10pt}\textit{S. Jean-Paul II, pape}, \textcolor{red}{Mémoire~facultative.}\\
b & 23 & \setlength{\hangindent}{10pt}\textit{S. Jean de Capistran, prêtre}, \textcolor{red}{Mémoire~facultative.}\\
c & 24 & \setlength{\hangindent}{10pt}\textit{S. Antoine-Marie Claret, évêque}, \textcolor{red}{Mémoire~facultative.}\\
d & 25 & \textcolor{red}{En Afrique du Nord, Belgique, Canada, France, Luxembourg :} \setlength{\hangindent}{10pt}DÉDICACE DES ÉGLISES CONSACRÉES DONT ON NE CONNAÎT PAS LA DATE DE CONSÉCRATION, \textcolor{red}{Solennité.}\\
e & 26 & \null\\
f & 27 & \null\\
g & 28 & \setlength{\hangindent}{10pt}SS. SIMON ET JUDE, APÔTRES, \textcolor{red}{Fête.}\\
\textcolor{red}{A} & 29 & \null\\
b & 30 & \textcolor{red}{En Afrique du Nord :} \setlength{\hangindent}{10pt}Ss. Marcel et Maximilien, martyrs, \textcolor{red}{Mémoire~obligatoire.}\\
c & 31 & \null\\
\null & \null & \null\\[1pt] \null & \null & \multicolumn{1}{c}{{\normalsize \textcolor{red}{Novembre}}}\\[5pt]d & 1 & \setlength{\hangindent}{10pt}TOUS LES SAINTS, \textcolor{red}{Solennité.}\\
e & 2 & \setlength{\hangindent}{10pt}\textit{Commémoration de tous les fidèles défunts}, \textcolor{red}{Mémoire~facultative.}\\
f & 3 & \setlength{\hangindent}{10pt}\textit{S. Martin de Porrès, religieux}, \textcolor{red}{Mémoire~facultative.}\\
\null & \null & \textcolor{red}{En Belgique et au Luxembourg :} \setlength{\hangindent}{10pt}\textit{S. Hubert, évêque}, \textcolor{red}{Mémoire~facultative.}\\
g & 4 & \setlength{\hangindent}{10pt}S. Charles Borromée, évêque, \textcolor{red}{Mémoire~obligatoire.}\\
\textcolor{red}{A} & 5 & \null\\
b & 6 & \null\\
c & 7 & \textcolor{red}{Au Luxembourg :} \setlength{\hangindent}{10pt}S. WILLIBRORD, ÉVÊQUE, \textcolor{red}{Fête.}\\
d & 8 & \null\\
e & 9 & \setlength{\hangindent}{10pt}DÉDICACE DE LA BASILIQUE DU LATRAN, \textcolor{red}{Fête.}\\
f & 10 & \setlength{\hangindent}{10pt}S. Léon le Grand, pape et docteur de l'Église, \textcolor{red}{Mémoire~obligatoire.}\\
g & 11 & \setlength{\hangindent}{10pt}S. Martin de Tours, évêque, \textcolor{red}{Mémoire~obligatoire.}\\
\textcolor{red}{A} & 12 & \setlength{\hangindent}{10pt}S. Josaphat, évêque et martyr, \textcolor{red}{Mémoire~obligatoire.}\\
b & 13 & \null\\
c & 14 & \null\\
d & 15 & \setlength{\hangindent}{10pt}\textit{S. Albert le Grand, évêque et docteur de l'Église}, \textcolor{red}{Mémoire~facultative.}\\
e & 16 & \setlength{\hangindent}{10pt}\textit{Ste Marguerite d'Écosse}, \textcolor{red}{Mémoire~facultative.}\\
\null & \null & \setlength{\hangindent}{10pt}\textit{Ste Gertrude, vierge}, \textcolor{red}{Mémoire~facultative.}\\
f & 17 & \setlength{\hangindent}{10pt}Ste Élisabeth de Hongrie, religieuse, \textcolor{red}{Mémoire~obligatoire.}\\
g & 18 & \setlength{\hangindent}{10pt}\textit{Dédicace des basiliques Saint-Pierre et Saint-Paul, Apôtres}, \textcolor{red}{Mémoire~facultative.}\\
\textcolor{red}{A} & 19 & \null\\
b & 20 & \null\\
c & 21 & \setlength{\hangindent}{10pt}Présentation de la bienheureuse Vierge Marie, \textcolor{red}{Mémoire~obligatoire.}\\
d & 22 & \setlength{\hangindent}{10pt}Ste Cécile, vierge et martyre, \textcolor{red}{Mémoire~obligatoire.}\\
e & 23 & \setlength{\hangindent}{10pt}\textit{S. Clément I\textsuperscript{er}, pape et martyr}, \textcolor{red}{Mémoire~facultative.}\\
\null & \null & \setlength{\hangindent}{10pt}\textit{S. Colomban, abbé}, \textcolor{red}{Mémoire~facultative.}\\
f & 24 & \setlength{\hangindent}{10pt}S. André Dung-Lac, prêtre, et ses compagnons, martyrs, \textcolor{red}{Mémoire~obligatoire.}\\
g & 25 & \setlength{\hangindent}{10pt}\textit{Ste Catherine d'Alexandrie, vierge et martyre}, \textcolor{red}{Mémoire~facultative.}\\
\textcolor{red}{A} & 26 & \textcolor{red}{En Belgique :} \setlength{\hangindent}{10pt}\textit{S. Jean Berchmans, religieux}, \textcolor{red}{Mémoire~facultative.}\\
b & 27 & \null\\
c & 28 & \null\\
d & 29 & \null\\
e & 30 & \setlength{\hangindent}{10pt}S. ANDRÉ, APÔTRE, \textcolor{red}{Fête.}\\

\null & \null & \null\\[1pt] \null & \null & \multicolumn{1}{c}{{\normalsize \textcolor{red}{Décembre}}}\\[5pt]f & 1 & \textcolor{red}{En Afrique du Nord :} \setlength{\hangindent}{10pt}\textit{S. Charles de Foucauld, prêtre}, \textcolor{red}{Mémoire~facultative.}\\
g & 2 & \null\\
\textcolor{red}{A} & 3 & \setlength{\hangindent}{10pt}S. François Xavier, prêtre, \textcolor{red}{Mémoire~obligatoire.}\\
b & 4 & \setlength{\hangindent}{10pt}\textit{S. Jean de Damas, prêtre et docteur de l'Église}, \textcolor{red}{Mémoire~facultative.}\\
c & 5 & \textcolor{red}{En Afrique du Nord :} \setlength{\hangindent}{10pt}Ste Crispine, martyre, \textcolor{red}{Mémoire~obligatoire.}\\
d & 6 & \setlength{\hangindent}{10pt}\textit{S. Nicolas, évêque}, \textcolor{red}{Mémoire~facultative.}\\
e & 7 & \setlength{\hangindent}{10pt}S. Ambroise, évêque et docteur de l'Église, \textcolor{red}{Mémoire~obligatoire.}\\
f & 8 & \setlength{\hangindent}{10pt}IMMACULÉE CONCEPTION DE LA BIENHEUREUSE VIERGE MARIE, \textcolor{red}{Solennité.}\\
g & 9 & \setlength{\hangindent}{10pt}\textit{S. Juan Diego Cuauhtlatoatzin}, \textcolor{red}{Mémoire~facultative.}\\
\textcolor{red}{A} & 10 & \setlength{\hangindent}{10pt}\textit{Bienheureuse Vierge Marie de Lorette}, \textcolor{red}{Mémoire~facultative.}\\
b & 11 & \setlength{\hangindent}{10pt}\textit{S. Damase I\textsuperscript{er}, pape}, \textcolor{red}{Mémoire~facultative.}\\
c & 12 & \setlength{\hangindent}{10pt}\textit{Bienheureuse Vierge Marie de Guadaloupé}, \textcolor{red}{Mémoire~facultative.}\\
d & 13 & \setlength{\hangindent}{10pt}Ste Lucie, vierge et martyre, \textcolor{red}{Mémoire~obligatoire.}\\
e & 14 & \setlength{\hangindent}{10pt}S. Jean de la Croix, prêtre et docteur de l'Église, \textcolor{red}{Mémoire~obligatoire.}\\
f & 15 & \null\\
g & 16 & \null\\
\textcolor{red}{A} & 17 & \null\\
b & 18 & \null\\
c & 19 & \null\\
d & 20 & \null\\
e & 21 & \setlength{\hangindent}{10pt}\textit{S. Pierre Canisius, prêtre et docteur de l'Église}, \textcolor{red}{Mémoire~facultative.}\\
f & 22 & \null\\
g & 23 & \setlength{\hangindent}{10pt}\textit{S. Jean de Kenty, prêtre}, \textcolor{red}{Mémoire~facultative.}\\
\textcolor{red}{A} & 24 & \null\\
b & 25 & \setlength{\hangindent}{10pt}NATIVITÉ DU SEIGNEUR, \textcolor{red}{Solennité.}\\
c & 26 & \setlength{\hangindent}{10pt}S. ÉTIENNE, PREMIER MARTYR, \textcolor{red}{Fête.}\\
d & 27 & \setlength{\hangindent}{10pt}S. JEAN, APÔTRE ET ÉVANGÉLISTE, \textcolor{red}{Fête.}\\
e & 28 & \setlength{\hangindent}{10pt}LES SAINTS INNOCENTS, MARTYRS, \textcolor{red}{Fête.}\\
f & 29 & \setlength{\hangindent}{10pt}\textit{S. Thomas Becket, évêque et martyr}, \textcolor{red}{Mémoire~facultative.}\\
g & 30 & \null\\
\textcolor{red}{A} & 31 & \setlength{\hangindent}{10pt}\textit{S. Sylvestre I\textsuperscript{er}, pape}, \textcolor{red}{Mémoire~facultative.}\\
\end{longtable}
\normalsize

\end{document}