% !TEX TS-program = lualatex
% !TEX encoding = UTF-8

\documentclass[nocturnale-romanum.tex]{subfiles}

\ifcsname preamble@file\endcsname
  \setcounter{page}{\getpagerefnumber{M-nr01_prolegomena}}
\fi

\begin{document}
\addcontentsline{toc}{section}{Prolegomena}

\feast{OR}{Rubricæ Generales Divini Officii ad Matutinum}
	{Prolegomena}{Rubricæ Generales}{2}{}{}{}{}{}{}
\addcontentsline{toc}{section}{Rubricæ Generales}

\rubric{Ante Officium, genuis flexis, licet ab omnibus secreto dicere \normaltext{Aperi, Dómine} et \normaltext{Dómine, in unióne}.}

\rubric{Posteaquam omnes locum sibi proprium occupaverunt, omnes assurgent et, si fit consuetudinem, dicunt totum secreto \normaltext{Pater}, \normaltext{Ave}, \normaltext{Credo}.}

\rubric{Tum Hebdomadarius cantat \normaltext{Dómine, lábia mea}, et respondet chorus \normaltext{Et os meum}; deinde continuat Hebdomadarius \normaltext{Deus in adjutórium}. Ad \normaltext{Dómine, lábia mea} se signant omnes in ore a pollice dextero, et ad \normaltext{Deus in adjutórium} se signant omnes a manu. Dum cantatur \normaltext{Glória Patri}, omnes caput profunde inclinabunt.}

\rubric{Postea cantatur conveniens invitatorium. Invitatorium ante psalmum \normaltext{Veníte} Cantores cantant, et omnes repetunt. Post singulos ejusdem psalmi versus, a Cantores cantatum, invitatorium vel integram vel ab asterisco \GreSpecial{*} alternatim omnes cantatur. Expleto psalmo, inchoatur hymnus ab Hebdomadario, et a duobus choris alternatim cantatur. Quum in ultimo versu nominabitur SS. Trinitas, caput ab omnibus inclinabitur.}

\rubric{Deinde cantantur antiphonæ convenientes, quæ in Officiis ritus duplicis ante et post psalmos integræ cantantur; in Officiis autem ritus semiduplicis initio psalmi inchoantur usque ad asteriscum \GreSpecial{*}, atque in fine integræ cantantur. Incepto primo versu primi psalmi, donec antiphona ultimi psalmi cujuslibet Nocturni repetita sit, sedent omnes. Antiphonæ inchoantur in primis ab Hebdomadario, deinde e dignissimo clero ad minus dignos fideles, secundum consuetudinem. Post ultimam antiphonam cujuslibet Nocturni, omnes stantes, cantat Cantor versum ut in Psalterio aut in Proprio, et respondent omnes.}

\rubric{Post versum cujuslibet Nocturni inchoat Hebdomadarius \normaltext{Pater noster}, deinde secreto usque ad :}
\versiculus{Et ne nos indúcas in tentatiónem.}{Sed líbera nos a malo.}

\rubric{Hebdomadarius cantat Absolutio.}

\rubric{Lector, qui canturus sit lectionem primam, junctis manibus et ad Hebdomadarium conversus, cantat \normaltext{Jube, Domne, benedícere}. Hebdomadarius cantat congruens Benedictio.}

\rubric{Si in choro non est diaconos vel sacerdos, ante singulas Lectiones Matutini, cantat Lector \normaltext{Jube, Dómine, benedícere}, ad Crucem conversus, et ipse Lector cantat congruens Benedictio. Digniores sive seniores ultimas Lectiones, minus antiqui primas cantent oportet.}

\rubric{Post \normaltext{Amen} a choro responsum, sedent omnes. Lectio peracta, fit Lector reverentiam ad Crucem, si Canonicus, secus genuflexionem, et cantat \normaltext{Tu autem}, etc.}

\rubric{Post quamlibet vero Lectionem, quae Hymnum \normaltext{Te Deum} immediate non præcedat, congruens dicitur Responsorium, et in fine ultimi Responsorii cujuslibet Nocturni additur Versus:  \normaltext{Glória Patri, et Fílio, et Spirítui Sancto,} et Responsorium a Signo\/ {\upshape \dag,} et quidem a secundo\/ {\upshape \ddag} aut tertio\/  {\upshape\P,} si duo vel tria fuerint, repetitur.}

\rubric{Pro Lectionibus secunda, tertia, quinta, sexta et octava, unus Hebdomadarius ad benedictionem assurget.}

\rubric{Pro 9. Lectionem, Hebdomadarius, conversus ad digniorem de choro, benedictionem ab illo petet, et dignior respondebit \normaltext{Ad societátem}, etc.}

\rubric{Ab Epsicopo autem, ultimam Lectionem cantaturo, item cantatur: \normaltext{Jube, Dómine, benedícere}, et respondetur a Choro \normaltext{Amen}, Benedictio omissa, et omnes stant quando fit Lectio.}

\rubric{Si \normaltext{Te Deum} cantatur, omnes stantes, inchoatur a Hebdomadario. Ad versus \normaltext{Te ergo}, omnes genuflectent.}

\rubric{Deinde cantantur Laudes matutinas. Si Laudes non sequent, cantat Hebdomadarius \normaltext{Dóminus vobíscum} aut \normaltext{Dómine, exáudi} si in ordine diaconatus non est constitutus, et Orationem, ut ad Laudes, sed semper sine commemorationes, et item \normaltext{Dóminus vobíscum}, et cantant Cantores \normaltext{Benedicámus Dómino}. Omnes respondent \normaltext{Deo grátias}.}

\rubric{Item cantat Hebdomadarius, voce recta et depressa, omnes se manu signans, versus \normaltext{Fidélium ánimæ}.}

\rubric{Omnes, genua flectantes, secreto dicent \normaltext{Pater noster}, si fit consuetudinem.}

\end{document}