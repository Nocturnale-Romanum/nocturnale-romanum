% !TEX TS-program = lualatex
% !TEX encoding = UTF-8

\documentclass[nocturnale-romanum.tex]{subfiles}

\ifcsname preamble@file\endcsname
  \setcounter{page}{\getpagerefnumber{M-nr20_psalterium_ordinarium}}
\fi

\begin{document}
\feast{OR}{Ordinarium Divini Officii ad Matutinum}{Ordinarium}{Ordinarium}{1}{}{}{}{}{}{}
\addcontentsline{toc}{chapter}{Ordinarium Divini Officii ad Matutinum}

\intermediatetitle{Ante Divinum Officium}
\rubric{Ante quam inchoëtur Officium, laudabiliter dicitur, sub singulari semper numero, sequens Oratio; pro qua Summus Pontifex Pius X Indulgentiam centum dierum concessit.}
\smalltitle{Oratio}
\begin{multicols}{2}
Aperi, Dómine, os meum ad benedicéndum nomen sanctum tuum: munda quoque cor meum ab ómnibus vanis, pervérsis et aliénis 
cogitatiónibus; intelléctum illúmina, afféctum inflámma, ut digne, atténte ac devóte hoc officium recitáre váleam, et exaudíri mérear
ante conspéctum divinae Majestátis tuae. Per Christum Dóminum nostrum. Amen.

Dómine, in unióne illíus divínæ intentiónis, qua ipse in terris laudes Deo persolvísti, has tibi horas \rubric{(vel} hanc tibi horam\rubric{)} persólvo.
\end{multicols}
\smalltitle{Ante singulas Horas}
\begin{multicols}{2}
Pater noster, qui es in cœlis, sanctificétur nomen tuum. Advéniat regnum tuum. Fiat volúntas tua, sicut in cœlo et in terra.
Panem nostrum quotidiánum da nobis hódie. Et dimítte nobis débita nostra, sicut et nos dimíttimus debitóribus nostris. Et ne nos
indúcas in tentatiónem: sed líbera nos a malo. Amen.

Ave Maria, grátia plena, Dóminus tecum: benedíctus fructus ventris tui Jesus. Sancta María, Mater Dei, ora pro nobis 
peccatóribus, nunc et in hora mortis nostræ. Amen.

Credo in Deum, Patrem omnipoténtem, Creatórem cœli et terræ. Et in Jesum Christum, Fílium ejus únicum, Dóminum nostrum,
qui concéptus est de spíritu Sancto, natus ex María Virgine, passus sub Póntio Piláto, crucifixus, mórtuus et sepúltus: descéndit
ad ínferos: tértia die resurréxit a mórtuis; ascéndit ad cœlos, sedet ad déxteram Patris omnipoténtis: inde ventúrus est judicáre
vivos et mórtuos. Credo in Spíritum sanctum, sanctam Ecclésiam cathólicam, Sanctórum communiónem, remissiónem peccatórum,
carnis resurrectiónem, vitam ætérnam. Amen.
\end{multicols}

\rubric{Post Pater noster, Ave Maria et Credo cantatur hic Versus, nisi in Matutinis Defunctorum aut in triduo ante Pascham in quo absolute cum prima Antiphona Officium incipiatur.}

\gscore{ORIa}{T}{}{Domine labia mea!Tonus simplex}
\gscore{ORIb}{T}{}{Domine labia mea!Tonus festivus}

\rubric{Postea dicitur conveniens Invitatorium, quod ante Psalmum bis cantatur, et ad singulos ejusdem Psalmi versus vel integrum vel dimidiatum ab Asterisco \greheightstar alternis vocibus, ut infra, repitur.}

\rubric{Expleto Psalmo, dicitur Hymnus Invitatorio respondens.}
\intermediatetitle{Per Annum}
\smalltitle{In Officio dominicali}
\rubric{In omnibus Dominicis extra Octavas Nativitatis Epiphaniæ, sanctissimi Corporis D.N.J.C.\ et Cordis D.N.J.C., Invitatorium et Hymnusder Feria currenti ut in Psalterio.}
\smalltitle{In Officio feriali}
\rubric{In omnibus Feriis per Annum, et a Feria \Rnum{4} Cinerum usque ad Sabbatum sequens inclusive, Invitatorium et Hymnus ut in Psalterio.}
\intermediatetitle{Tempore Adventus}
\rubric{A Dominica \Rnum{1} Adventus usque ad Sabbatum ante Dominicam \Rnum{3} inclusive, tam in dominicale quam in feriali Officio, dicitur quotidie sequens Invitatorium :}
\gscore{A1F1I}{I}{}{Regem venturum Dominum!In Dominicis}
\gscore{A1F2I}{I}{}{Regem venturum Dominum!In Feriis}
\rubric{A Dominica vero \Rnum{3} Adventus usque ad ultimam diem ante Vigiliam Nativitatis inclusive, tam in dominicali quam in feriali Officio, dicitur quotidie sequens Invitatorium. Sed in Vigilia Nativitatis Domini, etiam in Domincam \Rnum{4} incidente, Invitatorium ut in Proprio de Tempore.}
\gscore{A3F1I}{I}{}{Prope est jam Dominus!In Dominicis}
\gscore{A3F2I}{I}{}{Prope est jam Dominus!In Feriis}
\gscore{A1H}{H}{}{Verbum supernam prodiens}
\rubric{Conclusio communis, in Hymno praecedenti, et in aliis ejusdem metri habentibus, semper omittitur, quando specialis in omnibus Horis adhibenda præscribitur; et, si plures Conclusiones propriæ occurrant, sumitur Conclusio Officii currentis, aut secus Officii ipsa die et primo quidem loco inter cetera propriam Conclusionem habentia commemorandi, aut demum de occurrenti Octava communi vel de Tempore.  Conclusio tamen \normaltext{Jesu, tibi sit glória, Qui natus es de Virgine}, in Officiis de Tempore Adventus num-quam adhibetur.}
\intermediatetitle{Tempore Quadragesimæ}
\rubric{A Dominica \Rnum{1} usque ad Sabbatum ante Dominicam Passionis inclusive, tam in dominicali quam in feriali Officio, dicitur quotidie sequens Invitatorium :}
\gscore{Q1I}{I}{}{Non sit vobis vanum}
\gscore{Q1H}{H}{}{Ex more docti mystico}
\intermediatetitle{Tempore Passionis}
\rubric{A Dominica Passionis usque ad Feriam \Rnum{4} Majoris Hebdomadæ inclusive, tam in dominicali quam in feriali Officio, dicitur quotidie sequens :}
\gscore{Q5I}{I}{}{Hodie si vocem}
\gscore{Q5H}{H}{}{Pange lingua... Lauream}
\intermediatetitle{Tempore Paschali}
\rubric{A Dominica in Albis usque ad Vigiliam Ascensionis inclusive, tam in dominicali quam in feriali Officio, dicitur quotidie sequens.}
\gscore{P1I}{I}{}{Surrexit Dominus vere}
\gscore{P1H}{H}{}{Rex sempiterne coelitum}
\rubric{Sic terminantur omnes Hymni ejusdem metri usque ad Vigiliam Ascensionis inclusive, etiam in Officiis Sanctorum, nisi conclusio magis propria sit in eis adhibenda.}
\intermediatetitle{In Festis}
\rubric{In Officio cujuslibet Festi vel Octavæ, ac sanctæ Mariæ in Sabbato, Invitatorium et Hymnus ut in Proprio vel Communi; in Vigiliis autem Epiphaniæ ac Pentecostes, in Dominicis, etiam translatis, infra Octavas Nativitatis, Ascensionis ac sanctissimi Corporis ac sacratissimi Cordis Jesu, si de eis fiat Officium, Invitatorium et Hymnus ut in Proprio de Tempore.}

\label{PostHymn}

\intermediatetitle{In Officio novem Lectionum}
\rubric{Expleto Hymno dicuntur Antiphonæ convenientes, quæ in Officiis ritus Duplicis ante et post Psalmos integræ recitantur; in Officiis autem ritus Semiduplicis initio Psalmi inchoantur tantum et usque ad Asteriscum \greheightstar\ perducuntur, atque in fine integræ pronuntiantur.}
\nocturn{1}
\rubric{Sub congruentibus Antiphonis dicuntur tres Psalmi, ac deinde subjungitur Versus, prouti Officium occurrens requirit. Post Versum cujuslibet Nocturni dicitur \normaltext{Pater noster} secreto usque ad \normaltext{\vvrub Et ne nos indúcas in tentatiónem. \rrrub Sed líbera nos a malo.}}
\smalltitle{Absolutio}
\versiculus{Exáudi Dómine Jesu Christe, preces servórum tuorum, et miserére nobis: Qui cum Patre et Spíritu Sancto vivis et regnas in sǽcula sǽculorum.}{Amen.}

\vv Jube, domne, benedícere.

\rubric{Extra Chorum, quando ab uno tantum recitatur Officium, ante singulas Lectiones Matutini, dicitur: \normaltext{Jube, Dómine, benedícere} et subjungitur congruens Benedictio. Ab Epsicopo autem, ultimam Matutini Lectionem cantaturo, item dicitur: \normaltext{Jube, Dómine, benedícere;} et respondetur a Choro: \normaltext{Amen.}}

\smalltitle{Benedictiones}
\versiculus{\rubric{1.} Benedictióne perpétua benedícat nos Pater ætérnus.}{Amen.}

\rubric{Deinde dicuntur in unoquoque Nocturno Lectiones, prouti Officium Occurrens requirit, et in fine cujuslibet Lectionis additur :}

\versiculus{Tu autem, Dómine, miserére nobis.}{Deo grátias.}

\rubric{Post quamlibet vero Lectionem, quae Hymnum \normaltext{Te Deum} immediate non præcedat, congruens dicitur Responsorium, et in fine ultimi Responsorii cujuslibet Nocturni additur Versus:  \normaltext{Glória Patri, et Fílio, et Spirítui Sancto,} et Responsorium a Signo\/ {\upshape \dag,} et quidem a secundo\/ {\upshape \ddag} aut tertio\/  {\upshape\P,} si duo vel tria fuerint, repetitur.}

\versiculus{\rubric{2.} Unigénitus Dei Fílius nos benedícere et adjuváre dignétur.}{Amen.}
\versiculus{\rubric{3.} Spíritus Sancti grátia illúminet sensus et corda nostra.}{Amen.}

\nocturn{2}

\rubric{Sub congruentibus item Antiphonis dicuntur tres Psalmi et Versus, sicut in \Rnum{1} Nocturno. Post Versus dicitur \normaltext{Pater noster} secreto usque ad \normaltext{\vvrub Et ne nos indúcas in tentatiónem. \rrrub Sed líbera nos a malo.}}

\smalltitle{Absolutio}
\versiculus{Ipsíus píetas et misericódia nos ádjuvet, qui cum Patre et Spíritu Sancto vivit et regnat in sǽcula sǽculorum.}{Amen.}

\smalltitle{Benedictiones}

\versiculus{\rubric{4.} Deus Pater omnípotens sit nobis propítius et clemens.}{Amen.}
\versiculus{\rubric{5.} Christus perpétuæ det nobis gaúdia vitæ.}{Amen.}
\versiculus{\rubric{6.} Ignem sui amóris accéndat Deus in códibus nostris.}{Amen.}

\nocturn{3}

\rubric{Sub congruentibus item Antiphonis dicuntur tres Psalmi et Versus, sicut in \Rnum{1} Nocturno. Post Versus dicitur \normaltext{Pater noster} secreto usque ad \normaltext{\vvrub Et ne nos indúcas in tentatiónem. \rrrub Sed líbera nos a malo.}}

\smalltitle{Absolutio}
\versiculus{A vínculis peccatórum nostrórum absólvat nos omnípotens et miséricors Dóminus.}{Amen.}

\smalltitle{Benedictiones}

\versiculus{\rubric{7.} Evangélica léctio sit nobis salus et protéctio.}{Amen.}

\rubric{In Festis Domini et in Dominicis :}
\versiculus{\rubric{8.} Divínum auxílium máneat semper nobíscum.}{Amen.}

\rubric{In Festis beatæ Mariæ Viginis :}
\versiculus{\rubric{8.} Cujus festum cólimus, ipsa Virgo vírginum intercédat pro nobis ad Dóminum.}{Amen.}

\rubric{In Festis Sanctorum :}
\versiculus{\rubric{8.} Cujus \rubric{(vel} Quarum\rubric{)} festum cólimus, ipse \rubric{(vel} ipsa \rubric{aut} ipsæ\rubric{)} intercédat \rubric{(vel} intercédant\rubric{)} pro nobis ad Dóminum.}{Amen.}

\versiculus{\rubric{9.} Ignem sui amóris accéndat Deus in códibus nostris.}{Amen.}

\rubric{Si autem legenda \Rnum{9} Lectio sit de Homilia cum Evangelio Dominicæ, vel Feriæ, aut Vigilia :}
\versiculus{Per evangélica dicta deleántur nostra delícta.}{Amen.}

\tedeumrubric

\intermediatetitle{In Officio trium Lectionum}

\rubric{In Festis et Octavis Paschatis et Pentecostes, omnia dicuntur ut in Proprio de Tempore.
In ceteris trium Lectionum Officiis, post Hymnum dicuntur Antiphonæ convenientes, quæ initio Psalmi inchoantur tantum et usque ad Asteriscum \greheightstar\ perducuntur, ac deinde in fine integræ pronuntiantur.}

\rubric{Sub eisdem vero Antiphonis dicuntur novem Psalmi Feriæ currentis, quibus subjungitur Versus in \Rnum{3} Nocturno positus, omissis Versibus pro \Rnum{1} et \Rnum{2} Nocturno assignatis. Post Versus dicitur \normaltext{Pater noster} secreto usque ad \normaltext{\vvrub Et ne nos indúcas in tentatiónem. \rrrub Sed líbera nos a malo.}}

\smalltitle{Absolutio}
\rubric{Feria \Rnum{2} et \Rnum{5} \normaltext{Exáudi, Dómine}, ut in \Rnum{1} Nocturno. Feria \Rnum{3} et \Rnum{6} \normaltext{Ipsíus píetas},  ut in \Rnum{2} Nocturno. Feria \Rnum{4} et Sabbato \normaltext{A vínculis}, ut in \Rnum{3} Nocturno.}

\rubric{Deinde leguntur Lectiones cum Responsoriis, prouti Officium occurens requirit ; et ante eas dicuntur sequentes.}

\smalltitle{Benedictiones}

\rubric{In Feriis, quando legitur Homilia cum Evangelio :}

\versiculus{\rubric{1.} Evangélica léctio sit nobis salus et protéctio.}{Amen.}
\versiculus{\rubric{2.} Divínum auxílium máneat semper nobíscum.}{Amen.}
\versiculus{\rubric{3.} Ad societátem cívium supernórum perdúcat nos Rex Angelórum.}{Amen.}

\rubric{In Feriis, quando non legitur Homilia cum Evangelio, Feria \Rnum{2} et \Rnum{5} Benedictiones ut in \Rnum{1} Nocturno, Feria \Rnum{3} et \Rnum{6}  ut in \Rnum{2} Nocturno Officii novem Lectionum ; 
Feria autem \rnum{4} et Sabbato :}

\versiculus{\rubric{1.} Ille nos benedícat, qui sine fine vivit et regnat.}{Amen.}
\versiculus{\rubric{2.} Divínum auxílium máneat semper nobíscum.}{Amen.}
\versiculus{\rubric{3.} Ad societátem cívium supernórum perdúcat nos Rex Angelórum.}{Amen.}

\rubric{In Festis Sanctorum :}

\versiculus{\rubric{1.} Ille nos benedícat, qui sine fine vivit et regnat.}{Amen.}
\versiculus{\rubric{2.} Cujus \rubric{(vel} Quarum\rubric{)} festum cólimus, ipse \rubric{(vel} ipsa \rubric{aut} ipsæ\rubric{)} intercédat \rubric{(vel} intercédant\rubric{)} pro nobis ad Dóminum.}{Amen.}
\versiculus{\rubric{3.} Ad societátem cívium supernórum perdúcat nos Rex Angelórum.}{Amen.}

\intermediatetitle{B.M.V.\ in Sabbato}

\rubric{In Nocturno Antiphonæ, Psalmi et Versus de Sabbato. Post Versus dicitur \normaltext{Pater noster} secreto usque ad \normaltext{\vvrub Et ne nos indúcas in tentatiónem. \rrrub Sed líbera nos a malo.}}

\smalltitle{Absolutio}
\versiculus{Précibus et méritis beátæ Maríæ semper Virginis et ómnium Sanctórum, perdúcat nos Dóminus ad regna cœlórem.}{Amen.}

\smalltitle{Benedictiones}

\versiculus{\rubric{1.} Nos cum prole pia benedíccat Virgo María.}{Amen.}
\versiculus{\rubric{2.} Ipsa Virgo vírginum intercédat pro nobis ad Dóminum.}{Amen.}
\versiculus{\rubric{3.} Per Virginem matrem concédat nobis Dóminus salútem et pacem.}{Amen.}

\rubric{Post ultimam Lectionem, in omnibus Dominicis per Annum minoribus, etiam repositis vel anticipatis, in Vigilia Epiphaniæ, in Festis cujusvis ritus, excepto tamen sanctorum Innocentium Festo, nisi hoc in Dominicam indicat, aut ritu gaudeat Duplici I classis, per omnes Octavas, et in Officio sanctæ Mariæ in Sabbato, dicitur, Hymnus Ambrosianus. In Adventu autem, et a Dominica Septuagesimæ usque ad Sabbatum sanctum inclusive, non dicitur nisi in Festis; a Paschate vero usque ad Pentecosten inclusive, dicitur etiam in feriali Officio, excepta Feria \Rnum{2} Rogationum. Aliis Temporibus numquam dicitur in feriali Officio. Quando vero Hymnus prædictus omittitur, ejus loco dicitur \rnum{9} aut \rnum{3} Responsorium.}

\feast{TC}{Toni Communes}
	{Toni Communes}{Toni Communes}{1}{}{}{}{}{}{}
\addcontentsline{toc}{chapter}{Toni communes}

\feast{TCI}{Psalmi Toni Invitatorii}
	{Toni Communes}{Psalmi Toni Invitatorii}{2}{}{}{}{}{}{}
\smalltitle{Tonus II Modo}
\gscore{ORIP2}{P}{}{Venite exsultemus!Modo 2}
\smalltitle{Tonus III Modo}
\gscore{ORIP3a}{P}{}{Venite exsultemus!Modo 3g simplex}
\gscore{ORIP3b}{P}{}{Venite exsultemus!Modo 3f solemnis}
\smalltitle{Tonus IV Modo}
\gscore{ORIP4a}{P}{}{Venite exsultemus!Modo 4e simplex}
\gscore{ORIP4b}{P}{}{Venite exsultemus!Modo 4g festivus}
\gscore{ORIP4c}{P}{}{Venite exsultemus!Modo 4d solemnis}
\smalltitle{Tonus V Modo}
\gscore{ORIP5}{P}{}{Venite exsultemus!Modo 5g}
\smalltitle{Tonus VI Modo}
\gscore{ORIP6a}{P}{}{Venite exsultemus!Modo 6a simplex}
\gscore{ORIP6b}{P}{}{Venite exsultemus!Modo 6f festivus}
\gscore{ORIP6c}{P}{}{Venite exsultemus!Modo 6f solemnis}
\smalltitle{Tonus VII Modo}
\gscore{ORIP7a}{P}{}{Venite exsultemus!Modo 7a simplex}
\gscore{ORIP7b}{P}{}{Venite exsultemus!Modo 7a solemnis}

\feast{TCTD}{Hymnus Ambrosianus}
	{Toni Communes}{Hymnus Ambrosianus}{2}{}{}{}{}{}{}
\gscore{ORTDa}{H}{}{Te Deum laudamus!Tonus solemnis}
\gscore{ORTDb}{H}{}{Te Deum laudamus!Tonus simplex}

\feast{TCGP}{Toni Communes ad \og~Gloria Patri\fg}
	{Toni Communes}{Hymnus Ambrosianus}{2}{}{}{}{}{}{}
\gscore[n]{ORGP1}{T}{}{Toni Communes Ad Gloria Patri!Modo 1}
\gscore[n]{ORGP2}{T}{}{Toni Communes Ad Gloria Patri!Modo 2}
\gscore[n]{ORGP3}{T}{}{Toni Communes Ad Gloria Patri!Modo 3}
\gscore[n]{ORGP4}{T}{}{Toni Communes Ad Gloria Patri!Modo 4}
\gscore[n]{ORGP5}{T}{}{Toni Communes Ad Gloria Patri!Modo 5}
\gscore[n]{ORGP6}{T}{}{Toni Communes Ad Gloria Patri!Modo 6}
\gscore[n]{ORGP7}{T}{}{Toni Communes Ad Gloria Patri!Modo 7}
\gscore[n]{ORGP8}{T}{}{Toni Communes Ad Gloria Patri!Modo 8}
\gscore{ORBDa}{T}{}{Benedicamus Domino!In Festis Solemnibus}
\gscore{ORBDb}{T}{}{Benedicamus Domino!In Festis Duplicibus}
\gscore{ORBDc}{T}{}{Benedicamus Domino!In Festis Semiduplicibus}
\gscore{ORBDd}{T}{}{Benedicamus Domino!In Festis Beatae Mariae Virginis}
\gscore{ORBDe}{T}{}{Benedicamus Domino!In Dominicis per annum}
\gscore{ORBDf}{T}{}{Benedicamus Domino!In Festis Simplicibus}
\gscore{ORBDg}{T}{}{Benedicamus Domino!In Officio B.M.V in Sabbato}
\gscore{ORBDh}{T}{}{Benedicamus Domino!In Feriis per annum}
\gscore{ORBDi}{T}{}{Benedicamus Domino!In Feriis Adventus Quadrageimae et Passionis}
\gscore{ORBDj}{T}{}{Benedicamus Domino!In Octavam Paschae}
\gscore{ORBDk}{T}{}{Benedicamus Domino!In Dominicis Temporis Paschalis}
\gscore{ORBDl}{T}{}{Benedicamus Domino!In Feriis Temporis Paschalis}
\gscore{ORBDm}{T}{}{Benedicamus Domino!In Dominicis Adventus et Quadragesimae}

\feast{TCGP}{Toni Communes ad \og~Alleluia\fg in Fine Antiphonarum}
	{Toni Communes}{Toni Communes}{2}{}{}{}{}{}{}
\gscore[n]{ORAL1}{T}{}{Toni Communes Ad Alleluia in Fine Antiphonarum}

\feast{TCGP}{Toni Communes ad \og~Alleluia\fg in Fine Responsorium}
	{Toni Communes}{Toni Communes}{2}{}{}{}{}{}{}
\gscore[n]{ORAL2}{T}{}{Toni Communes Ad Alleluia In Fine Responsorium}

\feast{TCGP}{Toni Communes Pneumata In Fine Antiphonarum}
	{Toni Communes}{Toni Communes}{2}{}{}{}{}{}{}
\rubric{Pneumata in fine ultimæ cujuscumque Nocturni Antiphonæ in Festis majoribus ubi mos est, adjungi possunt.}
\gscore[n]{ORAL3}{T}{}{Toni Communes Pneumata In Fine Antiphonarum}

\end{document}