% !TEX TS-program = lualatex
% !TEX encoding = UTF-8

\documentclass[nocturnale-romanum.tex]{subfiles}

\ifcsname preamble@file\endcsname
  \setcounter{page}{\getpagerefnumber{M-nr20_psalterium_ordinarium}}
\fi

\begin{document}
\feast{OR}{Ordinarium Divini Offici\\ ad Matutinum}
	{Ordinarium}{Ordinarium}{1}{}{}{}{}{}{}
\addcontentsline{toc}{chapter}{Ordinarium Divini Officii ad Matutinum}

\intermediatetitle{Ante Divinum Officium}
\rubric{Ante quam inchoëtur Officium, laudabiliter dicitur,
sub singulari semper numero, sequens Orationes.}

\begin{multicols}{2}
\lettrine{A}{peri}, Dómine, os meum ad benedicéndum nomen sanctum tuum:
munda quoque cor meum ab ómnibus vanis, pervérsis et aliénis cogitatiónibus; 
intelléctum illúmina, afféctum inflámma, ut digne, atténte ac devóte
hoc officium recitáre váleam, et exaudíri mérear
ante conspéctum divinae Majestátis tuae. Per Christum Dóminum nostrum. Amen.

\lettrine{D}{ómine}, in unióne illíus divínæ intentiónis,
qua ipse in terris laudes Deo persolvísti,
has tibi horas \rubric{(vel} hanc tibi horam\rubric{)} persólvo.
\end{multicols}

\rubric{Officium ad Matutinum, nisi aliter suo loco indicetur,
per totum annum persolvitur juxta formam,
quæ in Rubricis hujus Ordinarii exhibetur. Ante Matutinum dicitur secreto:}

\begin{multicols}{2}
\lettrine{P}{ater noster}, qui es in cælis, sanctificétur nomen tuum.
Advéniat regnum tuum. Fiat volúntas tua, sicut in cælo et in terra.
Panem nostrum quotidiánum da nobis hódie.
Et dimítte nobis débita nostra, sicut et nos dimíttimus debitóribus nostris.
Et ne nos indúcas in tentatiónem: sed líbera nos a malo. Amen.

\lettrine{A}{ve María}, grátia plena, Dóminus tecum:
benedícta tu in muliéribus, et benedíctus fructus ventris tui Jesus.
Sancta María, Mater Dei, ora pro nobis peccatóribus,
nunc et in hora mortis nostræ. Amen.

\lettrine{C}{redo in Deum}, Patrem omnipoténtem, Creatórem cæli et terræ.
Et in Jesum Christum, Fílium ejus únicum, Dóminum nostrum,
qui concéptus est de spíritu Sancto, natus ex María Virgine,
passus sub Póntio Piláto, crucifixus, mórtuus et sepúltus:
descéndit ad ínferos: tértia die resurréxit a mórtuis;
ascéndit ad cælos, sedet ad déxteram Patris omnipoténtis:
inde ventúrus est judicáre vivos et mórtuos.
Credo in Spíritum sanctum, sanctam Ecclésiam cathólicam,
Sanctórum communiónem, remissiónem peccatórum,
carnis resurrectiónem, vitam ætérnam. Amen.
\end{multicols}

\gscore{ORIa}{T}{}{Domine labia mea!Tonus simplex}
\gscore{ORIb}{T}{}{Domine labia mea!Tonus festivus}

\intermediatetitle{Invitatorium et Hymnus}

\rubric{Postea dicitur conveniens Invitatorium, quod ante Psalmum bis cantatur,
et ad singulos ejusdem Psalmi versus
vel integrum vel dimidiatum ab asterisco \normaltext{\GreSpecial{*}} repetitur.}

\rubric{Expleto Psalmo, dicitur Hymnus Invitatorio respondens.}

\intermediatetitle{In Officio novem Lectionum}

\rubric{Expleto Hymno, dicuntur Antiphonæ convenientes,
quæ in Officiis ritus Duplicis ante et post Psalmos integræ recitantur;
in Officiis autem ritus Semiduplicis initio Psalmi inchoantur tantum
et usque ad asteriscum \normaltext{\GreSpecial{*}} perducuntur,
atque in fine integræ pronuntiantur.}

\rubric{In fine omnium Psalmorum additur \vvrub \normaltext{Glória Patri, et Fílio,~\psstar et Spirítui Sancto.}
\rrrub \normaltext{Sicut erat in princípio, et nunc, et semper,~\psstar et in sǽcula sæculórum. Amen.}
nisi aliter notatur.}

\nocturn{1}
\rubric{Sub congruentibus Antiphonis dicuntur tres Psalmi,
ac deinde subjungitur Versus, prouti Officium occurrens requirit.
Post Versum cujuslibet Nocturni dicitur:}

\gscore[n]{ORPN}{T}{}{Pater Noster}

\smalltitle{Absolutio}
\gscore[n]{ORA}{T}{}{Absolutio}

\smalltitle{Benedictiones et Lectiones}

\gscore[n]{ORLb}{T}{}{Benedictio!Tonus simplex}
\gscore[n]{ORLc}{T}{}{Benedictio!Tonus solemnis}

\rubric{Extra Chorum, quando ab uno tantum recitatur Officium,
ante singulas Lectiones, dicitur: \normaltext{Jube, Dómine, benedícere}
et subjungitur congruens Benedictio.
Ab Epsicopo autem, ultimam Matutini Lectionem cantaturo,
item dicitur: \normaltext{Jube, Dómine, benedícere};
et respondetur a Choro: \normaltext{Amen.}}

\rubric{Deinde dicuntur in unoquoque Nocturno Lectiones,
prouti Officium occurrens requirit, et in fine cujuslibet Lectionis additur:}

\gscore[n]{ORLd}{T}{}{In fine lectionum!Tonus simplex}
\gscore[n]{ORLe}{T}{}{In fine lectionum!Tonus solemnis}

\rubric{Post quamlibet vero Lectionem, quae Hymnum \normaltext{Te Deum}
immediate non præcedat, congruens dicitur Responsorium,
et in fine ultimi Responsorii cujuslibet Nocturni additur Versus:
\normaltext{Glória Patri, et Fílio, et Spirítui Sancto},
et Responsorium a signo \GreSpecial{+} repetitur, nisi aliter notatur.}

\rubric{Benedictiones pro aliis Lectionibus:}

\rubric{\emph{Benedictio 2.}} Unigénitus \textit{Dei} \textbf{Fí}lius~\GreSpecial{*}
nos benedícere et adjuváre dignétur.
\hspace{\specialcharhsep}\rr Amen.

\rubric{\emph{Benedictio 3.}} Spíritus \textit{Sancti} \textbf{grá}tia~\GreSpecial{*}
illúminet sensus et corda nostra.
\hspace{\specialcharhsep}\rr Amen.

\nocturn{2}

\rubric{Sub congruentibus item Antiphonis dicuntur tres Psalmi et Versus,
sicut in \Rnum{1} Nocturno.
Post Versus dicitur \normaltext{Pater noster} secreto usque ad
\vvrub \normaltext{Et ne nos indúcas in tentatiónem.}
\rrrub \normaltext{Sed líbera nos a malo.}}

\smalltitle{Absolutio}
\rubric{\emph{Absolutio 2.}}
Ipsíus píetas et misericódi\textit{a nos} \textbf{ád}juvet,~\GreSpecial{*}
qui cum Patre et Spíritu Sancto vivit et regnat in sǽcula sæculórum.
\hspace{\specialcharhsep}\rr Amen.

\smalltitle{Benedictiones}

\rubric{\emph{Benedictio 4.}} Deus Pa\textit{ter om}\textbf{ní}potens~\GreSpecial{*}
sit nobis propítius et clemens.
\hspace{\specialcharhsep}\rr Amen.

\rubric{\emph{Benedictio 5.}} Chris\textit{tus per}\textbf{pé}tuæ~\GreSpecial{*}
det nobis gaúdia vitæ.
\hspace{\specialcharhsep}\rr Amen.

\rubric{\emph{Benedictio 6.}} Ignem su\textit{i a}\textbf{mó}ris~\GreSpecial{*}
accéndat Deus in córdibus nostris.
\hspace{\specialcharhsep}\rr Amen.

\nocturn{3}

\rubric{Sub congruentibus item Antiphonis dicuntur tres Psalmi et Versus,
sicut in \Rnum{1} Nocturno.
Post Versus dicitur \normaltext{Pater noster} secreto usque ad
\vvrub \normaltext{Et ne nos indúcas in tentatiónem.}
\rrrub \normaltext{Sed líbera nos a malo.}}

\smalltitle{Absolutio}
\rubric{\emph{Absolutio 3.}}
A vínculis peccató\textit{rum nos}\textbf{tró}rum~\GreSpecial{*}
absólvat nos omnípotens et miséricors Dóminus.
\hspace{\specialcharhsep}\rr Amen.

\smalltitle{Benedictiones}

\rubric{\emph{Benedictio 7.}}
Evangé\textit{lica} \textbf{léc}tio~\GreSpecial{*}
sit nobis salus et protéctio.
\hspace{\specialcharhsep}\rr Amen.

\rubric{In Festis Domini et in Dominicis:}

\rubric{\emph{Benedictio 8.}}
Diví\textit{num au}\textbf{xí}lium~\GreSpecial{*}
máneat semper nobíscum.
\hspace{\specialcharhsep}\rr Amen.

\rubric{In Festis beatæ Mariæ Viginis:}

\rubric{\emph{Benedictio 8.}}
Cujus \textit{festum} \textbf{có}limus,~\GreSpecial{*}
ipsa Virgo vírginum intercédat pro nobis ad Dóminum.
\hspace{\specialcharhsep}\rr Amen.

\rubric{In Festis Sanctorum:}

\rubric{\emph{Benedictio 8.}}
Cujus \rubric{(vel} Quarum\rubric{)} \textit{festum} \textbf{có}limus,~\GreSpecial{*}
ipse \rubric{(vel} ipsa \rubric{aut} ipsæ\rubric{)}
intercédat \rubric{(vel} intercédant\rubric{)} pro nobis ad Dóminum.
\hspace{\specialcharhsep}\rr Amen.

\rubric{\emph{Benedictio 9.}}
Ad societátem cívium \textit{super}\textbf{nó}rum~\GreSpecial{*}
perdúcat nos Rex Angelórum.
\hspace{\specialcharhsep}\rr Amen.

\rubric{Si autem legenda ultima Lectio sit de Homilia cum Evangelio Dominicæ,
vel Feriæ, aut Vigiliæ:}

\rubric{\emph{Benedictio 9.}}
Per evangé\textit{lica} \textbf{dic}ta~\GreSpecial{*}
deleántur nostra delícta.
\hspace{\specialcharhsep}\rr Amen.

\intermediatetitle{In Officio trium Lectionum}

\intermediatetitle{In Nocturno}

\rubric{In Festis et Octavis Paschatis et Pentecostes,
omnia dicuntur ut in Proprio de Tempore.
In ceteris trium Lectionum Officiis, post Hymnum dicuntur Antiphonæ convenientes, 
quæ initio Psalmi inchoantur tantum et usque ad asteriscum
\normaltext{\GreSpecial{*}} perducuntur, ac deinde in fine integræ pronuntiantur.}

\rubric{Sub eisdem vero Antiphonis dicuntur novem Psalmi Feriæ currentis,
quibus subjungitur Versus in \Rnum{3} Nocturno positus,
omissis Versibus pro \Rnum{1} et \Rnum{2} Nocturno assignatis.
Post Versus dicitur \normaltext{Pater noster} secreto usque ad
\vvrub \normaltext{Et ne nos indúcas in tentatiónem.}
\rrrub \normaltext{Sed líbera nos a malo.}}

\smalltitle{Absolutio}

\rubric{Feria \Rnum{2} et \Rnum{5} \normaltext{Exáudi, Dómine},
ut in \Rnum{1} Nocturno.
Feria \Rnum{3} et \Rnum{6} \normaltext{Ipsíus píetas},
ut in \Rnum{2} Nocturno.
Feria \Rnum{4} et Sabbato \normaltext{A vínculis},
ut in \Rnum{3} Nocturno.}

\rubric{Deinde leguntur Lectiones cum Responsoriis,
prouti Officium occurens requirit ; et ante eas dicuntur sequentes.}

\smalltitle{Benedictiones}

\rubric{In Feriis, quando legitur Homilia cum Evangelio:}

\rubric{\emph{Benedictio 1.}}
Evangé\textit{lica} \textbf{léc}tio~\GreSpecial{*}
sit nobis salus et protéctio.
\hspace{\specialcharhsep}\rr Amen.

\rubric{\emph{Benedictio 2.}}
Diví\textit{num au}\textbf{xí}lium~\GreSpecial{*}
máneat semper nobíscum.
\hspace{\specialcharhsep}\rr Amen.

\rubric{\emph{Benedictio 3.}}
Ad societátem cívium \textit{super}\textbf{nó}rum~\GreSpecial{*}
perdúcat nos Rex Angelórum.
\hspace{\specialcharhsep}\rr Amen.

\rubric{In Feriis, quando non legitur Homilia cum Evangelio,
Feria \Rnum{2} et \Rnum{5} Benedictiones ut in \Rnum{1} Nocturno,
Feria \Rnum{3} et \Rnum{6}  ut in \Rnum{2} Nocturno Officii novem Lectionum;
Feria autem \Rnum{4} et Sabbato:}

\rubric{\emph{Benedictio 1.}}
Ille nos \textit{bene}\textbf{dí}cat,~\GreSpecial{*}
qui sine fine vivit et regnat.
\hspace{\specialcharhsep}\rr Amen.

\rubric{\emph{Benedictio 2.}}
Diví\textit{num au}\textbf{xí}lium~\GreSpecial{*}
máneat semper nobíscum.
\hspace{\specialcharhsep}\rr Amen.

\rubric{\emph{Benedictio 3.}}
Ad societátem cívium \textit{super}\textbf{nó}rum~\GreSpecial{*}
perdúcat nos Rex Angelórum.
\hspace{\specialcharhsep}\rr Amen.

\rubric{In Festis Sanctorum:}

\rubric{\emph{Benedictio 1.}}
Ille nos \textit{bene}\textbf{dí}cat,~\GreSpecial{*}
qui sine fine vivit et regnat.
\hspace{\specialcharhsep}\rr Amen.

\rubric{\emph{Benedictio 2.}}
Cujus \rubric{(vel} Quarum\rubric{)} \textit{festum} \textbf{có}limus,~\GreSpecial{*}
ipse \rubric{(vel} ipsa \rubric{aut} ipsæ\rubric{)}
intercédat \rubric{(vel} intercédant\rubric{)} pro nobis ad Dóminum.
\hspace{\specialcharhsep}\rr Amen.

\rubric{\emph{Benedictio 3.}}
Ad societátem cívium \textit{super}\textbf{nó}rum~\GreSpecial{*}
perdúcat nos Rex Angelórum.
\hspace{\specialcharhsep}\rr Amen.

\intermediatetitle{B.M.V.\ in Sabbato}

\rubric{In Nocturno Antiphonæ, Psalmi et Versus de Sabbato.
Post Versus dicitur \normaltext{Pater noster} secreto usque ad
\vvrub \normaltext{Et ne nos indúcas in tentatiónem.}
\rrrub \normaltext{Sed líbera nos a malo.}}

\smalltitle{Absolutio}

\rubric{\emph{Absolutio.}}
Précibus et méritis beátæ Maríæ semper Virginis
et ómni\textit{um San}\textbf{ctó}rum,~\GreSpecial{*}
perdúcat nos Dóminus ad regna cælórem.
\hspace{\specialcharhsep}\rr Amen.

\smalltitle{Benedictiones}

\rubric{\emph{Benedictio 1.}}
Nos cum \textit{prole} \textbf{pi}a~\GreSpecial{*}
benedícat Virgo María.
\hspace{\specialcharhsep}\rr Amen.

\rubric{\emph{Benedictio 2.}}
Ipsa \textit{Virgo} \textbf{vír}ginum~\GreSpecial{*}
intercédat pro nobis ad Dóminum.
\hspace{\specialcharhsep}\rr Amen.

\rubric{\emph{Benedictio 3.}}
Per Vír\textit{ginem} \textbf{ma}trem~\GreSpecial{*}
concédat nobis Dóminus salútem et pacem.
\hspace{\specialcharhsep}\rr Amen.

\intermediatetitle{Te Deum}

\rubric{Post ultimam Lectionem, in omnibus Dominicis per Annum,
etiam repositis vel anticipatis, in Vigilia Epiphaniæ,
in Festis cujusvis ritus, excepto tamen sanctorum Innocentium Festo,
nisi hoc in Dominicam indicat, aut ritu gaudeat Duplici I classis,
per omnes Octavas, et in Officio sanctæ Mariæ in Sabbato,
dicitur Hymnus Ambrosianus.
In Adventu autem, et a Dominica Septuagesimæ usque ad Sabbatum sanctum inclusive,
non dicitur nisi in Festis; a Paschate vero usque ad Pentecosten inclusive,
dicitur etiam in feriali Officio, excepta Feria \Rnum{2} Rogationum.
Aliis Temporibus numquam dicitur in feriali Officio.
Quando vero Hymnus prædictus omittitur,
ejus loco dicitur \Rnum{9} aut \Rnum{3} Responsorium.}

\intermediatetitle{Conclusio}

\rubric{Dicto \normaltext{Te Deum}, aut ultimo Responsorio,
statim incipiuntur Laudes a Versu \normaltext{Deus, in adjutórium}.
Si non sequent Laudes, post Hymnum \normaltext{Te Deum},
vel post ultimum Responsorium, dicitur:}

\gscore[n]{ORDV}{T}{}{Dominus vobiscum}

\rubric{Hic versus non dicitur ab eo, qui non est saltem in ordine diaconatus;
sed ejus loco substituitur:}

\versiculus{Dómine, exáudi oratiónem meam.}{Et clamor meus ad te véniat.}
\vv Orémus.
\rubric{Deinde cantatur oratio et respondetur \normaltext{Amen.}}

\versiculus{Dóminus vobíscum.}{Et cum spíritu tuo.}
\rubric{vel \normaltext{Dómine, exáudi}, etc.}
\rubric{\normaltext{Benedicámus Dómino} secundum diem, pag.\ \pageref{M-TCBD}.}
\versiculus{Fidélium ánimæ per misericórdiam Dei requiéscant in pace.}{Amen.}

\rubric{Deinde dicitur \normaltext{Pater noster} totum secreto.}

\newpage

\feast{TC}{Toni Communes}
	{Toni Communes}{Toni Communes}{1}{}{}{}{}{}{}
\addcontentsline{toc}{chapter}{Toni communes}

\feast{TCI}{Psalmi Toni Invitatorii}
	{Toni Communes}{Psalmi Toni Invitatorii}{2}{}{}{}{}{}{}
	
\smalltitle{Tonus 2d}
\gscore{ORIP2d}{P}{}{Venite exsultemus!Tonus 2d}

\smalltitle{Tonus 3c}
\gscore{ORIP3c}{P}{}{Venite exsultemus!Tonus 3c}

\smalltitle{Tonus 3c2}
\rubric{Hic tonus ad libitum pro tonum 3c substitui potest}
\gscore{ORIP3c2}{P}{}{Venite exsultemus!Tonus 3c2 ad lib.}

\smalltitle{Tonus 3e}
\gscore{ORIP3e}{P}{}{Venite exsultemus!Tonus 3e}

\smalltitle{Tonus 4e}
\gscore{ORIP4e}{P}{}{Venite exsultemus!Tonus 4e}

\smalltitle{Tonus 4g}
\gscore{ORIP4g}{P}{}{Venite exsultemus!Tonus 4g}

\smalltitle{Tonus 4d}
\gscore{ORIP4d}{P}{}{Venite exsultemus!Tonus 4d}

\smalltitle{Tonus 4d2}
\rubric{Hic tonus ad libitum pro tonos 4g et 4d substitui potest}
\gscore{ORIP4d2}{P}{}{Venite exsultemus!Tonus 4d2 ad lib.}

\smalltitle{Tonus 5g}
\gscore{ORIP5g}{P}{}{Venite exsultemus!Tonus 5g}

\smalltitle{Tonus 6a}
\gscore{ORIP6a}{P}{}{Venite exsultemus!Tonus 6a}

\smalltitle{Tonus 6f}
\gscore{ORIP6f}{P}{}{Venite exsultemus!Tonus 6f}

\smalltitle{Tonus 6f2}
\rubric{Hic tonus ad libitum pro tonum 6f substitui potest}
\gscore{ORIP6f2}{P}{}{Venite exsultemus!Tonus 6f2 ad lib.}

\smalltitle{Tonus 6f3}
\rubric{Hic tonus ad libitum pro tonum 6f substitui potest}
\gscore{ORIP6f3}{P}{}{Venite exsultemus!Tonus 6f3 ad lib.}

\smalltitle{Tonus 7a}
\gscore{ORIP7a}{P}{}{Venite exsultemus!Tonus 7a}

\smalltitle{Tonus 7g}
\gscore{ORIP7g}{P}{}{Venite exsultemus!Tonus 7g}

\smalltitle{Tonus 7g2}
\rubric{Hic tonus ad libitum pro tonum 7g substitui potest}
\gscore{ORIP7g2}{P}{}{Venite exsultemus!Tonus 7g2}

\feast{TCTD}{Hymnus Ambrosianus}
	{Toni Communes}{Hymnus Ambrosianus}{2}{}{}{}{}{}{}
\gscore{ORTDa}{H}{}{Te Deum laudamus!Tonus solemnis}
\gscore{ORTDb}{H}{}{Te Deum laudamus!Tonus simplex}

\newpage

\feast{TCGP}{Toni Communes ad «Gloria Patri»}
	{Toni Communes}{ad «Gloria Patri»}{2}{}{}{}{}{}{}
\gscore[n]{ORGP1}{T}{}{Toni Communes Ad Gloria Patri!Modo 1}
\gscore[n]{ORGP2}{T}{}{Toni Communes Ad Gloria Patri!Modo 2}
\gscore[n]{ORGP3}{T}{}{Toni Communes Ad Gloria Patri!Modo 3}
\gscore[n]{ORGP4}{T}{}{Toni Communes Ad Gloria Patri!Modo 4}
\gscore[n]{ORGP5}{T}{}{Toni Communes Ad Gloria Patri!Modo 5}
\gscore[n]{ORGP6}{T}{}{Toni Communes Ad Gloria Patri!Modo 6}
\gscore[n]{ORGP7}{T}{}{Toni Communes Ad Gloria Patri!Modo 7}
\gscore[n]{ORGP8}{T}{}{Toni Communes Ad Gloria Patri!Modo 8}

\pagebreak

\feast{TCBD}{Toni Communes ad «Benedicamus Domino»}
	{Toni Communes}{ad «Benedicamus Domino»}{2}{}{}{}{}{}{}
\rubric{In Festis Solemnibus}
\gscore{ORBDa}{T}{}{Benedicamus Domino!In Festis Solemnibus}
\rubric{In Festis Duplicibus}
\gscore{ORBDb}{T}{}{Benedicamus Domino!In Festis Duplicibus}
\rubric{Per Octavam Paschae}
\gscore{ORBDj}{T}{}{Benedicamus Domino!Per Octavam Paschae}
\rubric{In Festis Beatæ Mariæ Virginis}
\gscore{ORBDd}{T}{}{Benedicamus Domino!In Festis Beatae Mariae Virginis}
\pagebreak
\rubric{In Dominicis Adventus et Quadragesimæ}
\gscore{ORBDm}{T}{}{Benedicamus Domino!In Dominicis Adventus et Quadragesimae}
\rubric{In Dominicis Temporis Paschalis}
\gscore{ORBDk}{T}{}{Benedicamus Domino!In Dominicis Temporis Paschalis}
\rubric{In Dominicis per annum}
\gscore{ORBDe}{T}{}{Benedicamus Domino!In Dominicis per annum}
\rubric{In Festis Semiduplicibus}
\gscore{ORBDc}{T}{}{Benedicamus Domino!In Festis Semiduplicibus}
\pagebreak
\rubric{In Festis Simplicibus}
\gscore{ORBDf}{T}{}{Benedicamus Domino!In Festis Simplicibus}
\rubric{In Officio Beatæ Mariæ Virginis in Sabbato}
\gscore{ORBDg}{T}{}{Benedicamus Domino!In Officio B.M.V in Sabbato}
\rubric{In Feriis Temporis Paschalis}
\gscore{ORBDl}{T}{}{Benedicamus Domino!In Feriis Temporis Paschalis}
\rubric{In Feriis Adventus, Quadragesimæ et Passionis}
\gscore{ORBDi}{T}{}{Benedicamus Domino!In Feriis Adventus, Quadragesimae et Passionis}
\rubric{In Feriis per annum}
\gscore{ORBDh}{T}{}{Benedicamus Domino!In Feriis per annum}

\newpage

\feast{TCALA}{Toni Communes ad «Alleluia» in Fine Antiphonarum}
	{Toni Communes}{ad «Alleluia»}{2}{}{}{}{}{}{}
\gscore[n]{ORAL1}{T}{}{Toni Communes Ad Alleluia in Fine Antiphonarum}

\feast{TCALR}{Toni Communes ad «Alleluia» in Fine Responsorium}
	{Toni Communes}{ad «Alleluia»}{2}{}{}{}{}{}{}
\gscore[n]{ORAL2}{T}{}{Toni Communes Ad Alleluia In Fine Responsorium}

\pagebreak

\feast{TCPNM}{Toni Communes Pneumata In Fine Antiphonarum}
	{Toni Communes}{Pneumata}{2}{}{}{}{}{}{}
\rubric{Pneumata in fine ultimæ cujuscumque Nocturni Antiphonæ in Festis majoribus ubi mos est, adjungi possunt.}
\gscore[n]{ORAL3}{T}{}{Toni Communes Pneumata In Fine Antiphonarum}

\newpage

\feast{IH}{Invitatoria et Hymni}{Invitatoria et Hymni}{Invitatoria et Hymni}{1}{}{}{}{}{}{}
\addcontentsline{toc}{chapter}{Invitatoria et Hymni}

\intermediatetitle{Per Annum}
\smalltitle{In Officio dominicali}
\rubric{In omnibus Dominicis extra Octavas Nativitatis Epiphaniæ, sanctissimi Corporis D.N.J.C.\ et Cordis D.N.J.C., Invitatorium et Hymnus ut in Psalterio.}
\smalltitle{In Officio feriali}
\rubric{In omnibus Feriis per Annum, et a Feria \Rnum{4} Cinerum usque ad Sabbatum sequens inclusive, Invitatorium et Hymnus ut in Psalterio.}
\intermediatetitle{Tempore Adventus}
\rubric{A Dominica \Rnum{1} Adventus usque ad Sabbatum ante Dominicam \Rnum{3} inclusive, tam in dominicale quam in feriali Officio, dicitur quotidie sequens Invitatorium :}
\gscore{A1F1I}{I}{}{Regem venturum Dominum!In Dominicis}
\gscore{A1F2I}{I}{}{Regem venturum Dominum!In Feriis}
\rubric{A Dominica vero \Rnum{3} Adventus usque ad ultimam diem ante Vigiliam Nativitatis inclusive, tam in dominicali quam in feriali Officio, dicitur quotidie sequens Invitatorium. Sed in Vigilia Nativitatis Domini, etiam in Domincam \Rnum{4} incidente, Invitatorium ut in Proprio de Tempore.}
\gscore{A3F1I}{I}{}{Prope est jam Dominus!In Dominicis}
\gscore{A3F2I}{I}{}{Prope est jam Dominus!In Feriis}
\gscore{A1H}{H}{}{Verbum supernum prodiens}
\rubric{Conclusio communis, in Hymno praecedenti, et in aliis ejusdem metri habentibus, semper omittitur, quando specialis in omnibus Horis adhibenda præscribitur; et, si plures Conclusiones propriæ occurrant, sumitur Conclusio Officii currentis, aut secus Officii ipsa die et primo quidem loco inter cetera propriam Conclusionem habentia commemorandi, aut demum de occurrenti Octava communi vel de Tempore.  Conclusio tamen \normaltext{Jesu, tibi sit glória, Qui natus es de Virgine}, in Officiis de Tempore Adventus num-quam adhibetur.}
\intermediatetitle{Tempore Quadragesimæ}
\rubric{A Dominica \Rnum{1} usque ad Sabbatum ante Dominicam Passionis inclusive, tam in dominicali quam in feriali Officio, dicitur quotidie sequens Invitatorium :}
\gscore{Q1I}{I}{}{Non sit vobis vanum}
\gscore{Q1H}{H}{}{Ex more docti mystico}
\intermediatetitle{Tempore Passionis}
\rubric{A Dominica Passionis usque ad Feriam \Rnum{4} Majoris Hebdomadæ inclusive, tam in dominicali quam in feriali Officio, dicitur quotidie sequens :}
\gscore{Q5I}{I}{}{Hodie si vocem}
\gscore{Q5H}{H}{}{Pange lingua... Lauream}
\intermediatetitle{Tempore Paschali}
\label{P1I}% For the referencel pointing here from the Proper of Time / TP
\rubric{A Dominica in Albis usque ad Vigiliam Ascensionis inclusive, tam in dominicali quam in feriali Officio, dicitur quotidie sequens.}
\gscore{P0I}{I}{}{Surrexit Dominus vere}
\gscore{P1H}{H}{}{Rex sempiterne coelitum}
\rubric{Sic terminantur omnes Hymni ejusdem metri usque ad Vigiliam Ascensionis inclusive, etiam in Officiis Sanctorum, nisi conclusio magis propria sit in eis adhibenda.}
\intermediatetitle{In Festis}
\rubric{In Officio cujuslibet Festi vel Octavæ, ac sanctæ Mariæ in Sabbato, Invitatorium et Hymnus ut in Proprio vel Communi; in Vigiliis autem Epiphaniæ ac Pentecostes, in Dominicis, etiam translatis, infra Octavas Nativitatis, Ascensionis ac sanctissimi Corporis ac sacratissimi Cordis Jesu, si de eis fiat Officium, Invitatorium et Hymnus ut in Proprio de Tempore.}


\end{document}