% !TEX TS-program = lualatex
% !TEX encoding = UTF-8

\documentclass[nocturnale-romanum.tex]{subfiles}

\ifcsname preamble@file\endcsname
  \setcounter{page}{\getpagerefnumber{M-nr20_psalterium_ordinarium}}
\fi

\begin{document}
\feast{OR}{Ordinarium Divini Officii ad Matutinum}{Ordinarium}{Ordinarium}{1}{}{}{}{}{}
\gscore{ORIa}{T}{}{Domine labia mea}
\gscore{ORIb}{T}{}{Domine labia mea (in festis)}
\gscore{ORIP2}{P}{}{Venite exsultemus (mode 2d)}
\gscore{ORIP3a}{P}{}{Venite exsultemus (mode 3g simplex)}
\gscore{ORIP3b}{P}{}{Venite exsultemus (mode 3f solemnis)}
\gscore{ORIP4a}{P}{}{Venite exsultemus (mode 4e simplex)}
\gscore{ORIP4b}{P}{}{Venite exsultemus (mode 4g festivus)}
\gscore{ORIP4c}{P}{}{Venite exsultemus (mode 4d solemnis)}
\gscore{ORIP5}{P}{}{Venite exsultemus (mode 5g)}
\gscore{ORIP6a}{P}{}{Venite exsultemus (mode 6a simplex)}
\gscore{ORIP6b}{P}{}{Venite exsultemus (mode 6f festivus)}
\gscore{ORIP6c}{P}{}{Venite exsultemus (mode 6f solemnis)}
\gscore{ORIP7a}{P}{}{Venite exsultemus (mode 7a simplex)}
\gscore{ORIP7b}{P}{}{Venite exsultemus (mode 7a solemnis)}
\gscore{A1F1I}{I}{}{Regem venturum Dominum (in Dominicis)}
\gscore{A1F2I}{I}{}{Regem venturum Dominum (in Feriis)}
\gscore{A3F1I}{I}{}{Prope est jam Dominus (in Dominicis)}
\gscore{A3F2I}{I}{}{Prope est jam Dominus (in Feriis)}
\gscore{A1H}{H}{}{Verbum supernam prodiens}
\gscore{Q1I}{I}{}{Non sit vobis vanum}
\gscore{Q1H}{H}{}{Ex more docti mystico}
\gscore{Q5I}{I}{}{Hodie si vocem}
\gscore{Q5H}{H}{}{Pange lingua... Lauream}
\gscore{P1I}{I}{}{Surrexit Dominus vere}
\gscore{P1H}{H}{}{Rex sempiterne coelitum}
\gscore{ORTDa}{H}{}{Te Deum laudamus (tonus solemnis)}
\gscore{ORTDb}{H}{}{Te Deum laudamus (tonus simplex)}
\gscore{ORGP1}{T}{}{Toni Communes Ad Gloria Patri (Modo I)}
\gscore{ORGP2}{T}{}{Toni Communes Ad Gloria Patri (Modo II)}
\gscore{ORGP3}{T}{}{Toni Communes Ad Gloria Patri (Modo III)}
\gscore{ORGP4}{T}{}{Toni Communes Ad Gloria Patri (Modo IV)}
\gscore{ORGP5}{T}{}{Toni Communes Ad Gloria Patri (Modo V)}
\gscore{ORGP6}{T}{}{Toni Communes Ad Gloria Patri (Modo VI)}
\gscore{ORGP7}{T}{}{Toni Communes Ad Gloria Patri (Modo VII)}
\gscore{ORGP8}{T}{}{Toni Communes Ad Gloria Patri (Modo VIII)}
\gscore{ORBDa}{T}{}{Benedicamus Domino (in Festis Solemnibus)}
\gscore{ORBDb}{T}{}{Benedicamus Domino (in Festis Duplicibus)}
\gscore{ORBDc}{T}{}{Benedicamus Domino (in Festis Semiduplicibus)}
\gscore{ORBDd}{T}{}{Benedicamus Domino (in Festis Beatae Mariae Virginis)}
\gscore{ORBDe}{T}{}{Benedicamus Domino (in Dominicis per annum)}
\gscore{ORBDf}{T}{}{Benedicamus Domino (in Festis Simplicibus)}
\gscore{ORBDg}{T}{}{Benedicamus Domino (In Officio B.M.V in Sabbato)}
\gscore{ORBDh}{T}{}{Benedicamus Domino (In Feriis per annum)}
\gscore{ORBDi}{T}{}{Benedicamus Domino (In Feriis Adventus Quadrageimae et Passionis)}
\gscore{ORBDj}{T}{}{Benedicamus Domino (In Octavam Paschae)}
\gscore{ORBDk}{T}{}{Benedicamus Domino (In Dominicis Temporis Paschalis)}
\gscore{ORBDl}{T}{}{Benedicamus Domino (In Feriis Temporis Paschalis)}
\gscore{ORBDm}{T}{}{Benedicamus Domino (In Dominicis Adventus et Quadragesimae)}
\gscore{ORAL1}{T}{}{Toni Communes Ad Alleluia in Fine Antiphonarum}
\gscore{ORAL2}{T}{}{Toni Communes Ad Alleluia In Fine Responsorium}
\gscore{ORAL3}{T}{}{Toni Communes Pneumata In Fine Antiphonarum}
\end{document}