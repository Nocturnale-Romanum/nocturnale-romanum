% !TEX TS-program = lualatex
% !TEX encoding = UTF-8

\documentclass[nocturnale-romanum.tex]{subfiles}

\ifcsname preamble@file\endcsname
  \setcounter{page}{\getpagerefnumber{M-nr35_commune_bmv}}
\fi

\begin{document}
\feast{CBMV}{Commune Beatæ Mariæ Virginis}
	{Commune Sanctorum}{Commune Beatæ Mariæ Virginis}{2}{}{}{}{}{}{}
\addcontentsline{toc}{section}{Commune Beatæ Mariæ Virginis}

\gscore{CBMVI}{I}{}{Sancta Maria Dei Genitrix Virgo}
\gscore{CBMVH}{H}{}{Quem terra pontus sidera}
\nocturn{1}
\gscore{CBMVN1A1}{A}{1}{Benedicta tu in mulieribus}
\psalmus{2}{}
\gscore{CBMVN1A2}{A}{2}{Sicut myrrha}
\psalmus{18}{}
\gscore{CBMVN1A3}{A}{3}{Ante torum}
\psalmus{23}{}
\versiculustpall{Spécie tua et pulchritúdine tua.}{Inténde, próspere procéde, et regna.}
\gscore{CBMVN1R1}{R}{1}{Sancta et Immaculata V Benedicta tu}
\gscore{CBMVN1R2}{R}{2}{Congratulamini...parvula V Beatam me dicent}
\gscore{CBMVN1R3}{R}{3}{Beata es Maria V Ave Maria V Gloria}
\nocturn{2}
\gscore{CBMVN2A1}{A}{4}{Specie tua}
\psalmus{44i}{}
\gscore{CBMVN2A2}{A}{5}{Adjuvabit eam}
\psalmus{45}{}
\gscore{CBMVN2A3}{A}{6}{Sicut laetantium}
\psalmus{86}{}
\versiculustpall{Adjuvábit eam Deus vultu suo.}{Deus in médio ejus, non commovébitur.}
\gscore{CBMVN2R1}{R}{4}{Sicut cedrus}
\gscore{CBMVN2R2}{R}{5}{Quae est ista quae processit}
\gscore{CBMVN2R3}{R}{6}{Ornatam in monilibus}
\nocturn{3}
\gscore{CBMVN3A1}{A}{7}{Gaude Maria Virgo}
\psalmus{95}{}
\gscore{CBMVN3A2}{A}{8}{Dignare me laudare}
\psalmus{96}{}
\gscore{CBMVN3A3a}{A}{9}{Post partum Virgo}
\gscore{CBMVN3A3b}{A}{9}{Angelus Domini nuntiavit}
\psalmus{97}{}
\versiculustpall{Elégit eam Deus, et præelégit eam.}{In tabernáculo suo habitáre facit eam.}
\gscore{CBMVN3R1}{R}{7}{Felix namque es}
\vspace{-0.25in}
\begin{tabbing}
{\hskip 0.5em}\={\hskip 1.8em}\={\hskip 2.2em}\={\hskip 2.0em}\={\hskip 1.5em}\={\hskip 1.8em}\={\hskip 2em}\={\hskip 1.7em}\={\hskip 3.2em}\={\hskip 2.8em}\=\\
%\>1\>2\>3\>4\>5\>6\>7\>8\>9\>0\\
\>\>\>\>{\smaller sanctam}\>\>Fe-\>sti-\>vi-\>tá-\>tem. \\
\>\>\>\>\>Vi-\>si-\>ta-\>ti-\>ó-\>nem.\\
\>{\smaller so{\bfseries lé}mnem}\\
\>\>\>\>\hspace{-0.75em}Com-\>me-\>mo-\>ra-\>ti-\>ó-\>nem.\\
\>\>\>\>\>\>As-\>sum\>pti-\>ó-\>nem.\\
\>\>\>\>\>\>Na-\>ti-\>vi-\>tá-\>tem.\\
\>\underline{\smaller tui}\>{\smaller sancti}\>{\smaller {\bfseries Nó}minis}\\
\>\>\>\>\hspace{-0.75em}Com-\>me-\>mo-\>ra-\>ti-\>ó-\>nem.\\
\>\underline{{\smaller sanc{\bfseries tís}simi}}\>\>{\smaller Ro{\bfseries sá}rii}\>\>\>So-\>le-\>mni-\>tá-\>tem.\\
\>{\bfseries san}ctam\>\>\>\>\hspace{-0.1em}Præ-\>sen-\>ta-\>ti-\>ó-\>nem. \\
\>\underline{\smaller tuum}\>{\smaller{\ \bfseries san}ctum}\>\>implo-\>\>rant\>au-\>xí-\>li-\>um.\\
\end{tabbing}
\gscore{CBMVN3R2}{R}{8}{Beatam me dicent V Et misericordia ejus V Gloria}
\tedeumrubric

\feast{CSMS}{De Sancta Maria in Sabbato}
	{Commune Sanctorum}{De Sancta Maria in Sabbato}{2}{}
	{}{}{Sabbato Sanctae Mariae}
	{}{}
\addcontentsline{toc}{section}{De Sancta Maria in Sabbato}
\rubric{Omnibus Sabbatis, extra Tempus Adventus, Quadrageismæ (a Sabbato post Cineres inclusive), Passionis, et Quatuor Temporum Septembris, nisi agendum fuerit de aliquo Festo Duplici, etiam translato, aut Semiduplici, vel de Octva aut Viglia occurente, aut de Dominica anticipata juxta Rubricas, fit Officium de Sancta Maria, in quo, sumptis Antiphonis et Psalmis de Feria occurenti, reliqua dicuntur ut infra.}
\gscore{CSMSI}{I}{}{Ave Maria gratia plena Dominus tecum}
\rubric{Hymnus \scorename{CBMVH}, ut in Communi B.M.V. pag.\ \pageref{M-CBMVH}. In Nocturno Antiphonæ, Psalmi et Versus de Sabbato. Lectio \Rnum{1}.\ et \Rnum{2}.\ de Scriptura occurenti, cum suis Responsoriis de Tempore. Lectio vero \Rnum{3} erit una propria, juxta ordinem Mensium ut in Breviario. Post ultimam Lectionem dicitur Hymnus \emph{Te Deum}.}

\feast{OPBM}{Officium Parvum\\Beatæ Mariæ Virginis}
	{Officium Parvum B. M. V.}{Officium Parvum B. M. V.}{2}{}{}{}{}{}{}
\addcontentsline{toc}{section}{Officium Parvum B. M. V.}

\intermediatetitle{Extra Tempus Adventus}

\rubric{Quod dicitur a diei 3 Februarii usque ad Feria \Rnum{4} Hebdomadæ Sanctæ inclusive, præterquam in Festo Annuntiationis, et a Festo Sanctissimæ Trinitatis usque ad Sabbato ante Dominicam \Rnum{1} Adventus inclusive.}

\rubric{\normaltext{Ave Maria} secreto, deinde \vvrub\normaltext{Dómine, lábia mea} et \vvrub\normaltext{Deus in adjutórium}. Invitatorium \scorename{CSMSI}, ut supra. Hymnus \scorename{CBMVH}, ut in Communi B.M.V. pag.\ \pageref{M-CBMVH}.}

\rubric{Antiphonæ et Psalmi ut in Communi B.M.V., sciliet:  Dominica, Feria \Rnum{2} et \Rnum{5}, de \Rnum{1} Nocturno, pag.\ \pageref{M-CBMVN1A1}. Feria \Rnum{3} et \Rnum{6}, de \Rnum{2} Nocturno, pag.\ \pageref{M-CBMVN2A1}. Feria \Rnum{4} et Sabbato, de \Rnum{3} Nocturno, pag.\ \pageref{M-CBMVN3A1}.}

\versiculus{Diffúsa est grátia in lábiis tuis.}{Proptérea benedíxit te Deus in ætérnum.}

\respref{1}{CBMVN1R1}{}
\resprefsinegp{2}{CBMVN1R3}
\respref{3}{CBMVN3R1}{cum \vvrub \normaltext{...commemoratiónem} et \vvrub \normaltext{Glória Patri} in fine}
\rubric{Extra Septuagesimam et Quadragesimam omittitur \rrrub 3. et dicitur \normaltext{Te Deum}.}

\rubric{Tempore Paschali Invitatorio, Antiphonis, et Responsoriis \emph{Allelúja} non additur.}

\intermediatetitle{Tempore Adventus}

\rubric{Quod dicitur a Dominica \Rnum{1} Adventus usque ad Vigiliam Nativitatis Domini inclusive, et in Festo Annuntiationis. Omnia dicuntur ut per Annum notatur, præter sequentia.}

\gscore{OPBMR1}{R}{1}{Missus est Gabriel V Dabit ei Dominux}
\gscore{OPBMR2}{R}{2}{Ave Maria gratia plena V Quomodo fiet istud}
\gscore{OPBMR3}{R}{3}{Suscipe verbum V Ut paries}

\intermediatetitle{Tempore Nativitatis}



\end{document}