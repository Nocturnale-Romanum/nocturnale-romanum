% !TEX TS-program = lualatex
% !TEX encoding = UTF-8

\documentclass[nocturnale-romanum.tex]{subfiles}

\ifcsname preamble@file\endcsname
  \setcounter{page}{\getpagerefnumber{M-nr41_sanctorale_februarii}}
\fi

\begin{document}

\feast{0200}{Festa Februarii}{Proprium Sanctorum}{Festa Februarii}{1}{}{}{}{}{}{}

\feast{0201}{S. Ignatii Antiocheni Episcopi et Martyris}
	{Proprium Sanctorum}{Festa Februarii}{2}{1 Februarii}
	{Duplex}{III. classis}{Ignatii Antiocheni}
	{Commune unius Martyris, \pageref{M-UMEX}.}
	{}

\feast{0202}{In Purificatione Beatæ Mariæ Virginis}
	{Proprium Sanctorum}{Festa Februarii}{2}{2 Februarii}
	{Duplex II. classis}{II. classis}{Mariæ!Purificatio}
	{Hymnus, Antiphonæ, Psalmi et Versus Nocturnorum ut in Communi B.M.V. \pageref{M-CBMV}. Reliqua ut infra.}
	{}
\gscore{0202I}{I}{}{Ecce venit ad templum}
\nocturn{1}
\gscore{0202N1R1}{R}{1}{Adorna thalamum tuum}
\gscore{0202N1R2}{R}{2}{Postquam impleti sunt}
\gscore{0202N1R3}{R}{3}{Obtulerunt pro eo}
\nocturn{2}
\gscore{0202N2R1}{R}{4}{Simeon justus}
\gscore{0202N2R2}{R}{5}{Responsum acceperat}
\gscore{0202N2R3}{R}{6}{Cum inducerent... Simeon}
\nocturn{3}
\gscore{0202N3R1}{R}{7}{Suscipiens Jesum in ulnas}
\gscore{0202N3R2}{R}{8}{Senex puerum portabat}
\tedeumrubric

\feast{0203}{S. Blasii Episcopi et Martyris}
	{Proprium Sanctorum}{Festa Februarii}{2}{3 Februarii}
	{Simplex}{(Commemoratio tantum)}{Blasii}
	{Commune unius Martyris, \pageref{M-UMEX}.}
	{}

\feast{0204}{S. Andreæ Corsini Episcopi et Confessoris}
	{Proprium Sanctorum}{Festa Februarii}{2}{4 Februarii}
	{Duplex m.t.v.}{III. classis}{Andreæ Corsini}
	{Commune Confessoris Pontificis, \pageref{M-COPO}.}
	{}

\feast{0205}{S. Agathæ Virginis et Martyris}
	{Proprium Sanctorum}{Festa Februarii}{2}{5 Februarii}
	{Duplex}{III. classis}{Agathæ}
	{Invitatorium et Hymnus pro Martyre, de Communi Virginum, pag.\ \pageref{M-MU}. Psalmi ut in Communi unius Martyris, pag.\ \pageref{M-UMEX}, præter ultimus. Reliqua ut infra.}
	{}
\nocturn{1}
\gscore{0205N1A1}{A}{1}{Ingenua sum}
\gscore{0205N1A2}{A}{2}{Summa ingenuitas}
\gscore{0205N1A3}{A}{3}{Ancilla Christi sum}
\versiculus{Spécie tua et pulchritúdine tua.}{Inténde, próspere procéde, et regna.}
\gscore{0205N1R1}{R}{1}{Dum ingrederetur beata Agathes}
\gscore{0205N1R2}{R}{2}{Agathes laetissima}
\gscore{0205N1R3}{R}{3}{Quis es tu qui venisti}
\nocturn{2}
\gscore{0205N2A1}{A}{4}{Agatha sancta dixit}
\gscore{0205N2A2}{A}{5}{Si ignem adhibeas}
\gscore{0205N2A3}{A}{6}{Agathes laetissima}
\versiculus{Adjuvábit eam Deus vultu suo.}{Deus in  médio ejus, non commovébitur.}
\gscore{0205N2R1}{R}{4}{Ego autem adjuvata}
\gscore{0205N2R2}{R}{5}{Ipse me coronavit}
\gscore{0205N2R3}{R}{6}{Vidisti Domine et expectavi}
\nocturn{3}
\gscore{0205N3A1}{A}{7}{Nisi diligenter perfeceris}
\gscore{0205N3A2}{A}{8}{Vidisti Domine agonem meum}
\gscore{0205N3A3}{A}{9}{Propter fidem castitatis}
\psalmus{15}{}
\versiculus{Elégit eam Deus, et præelégit eam.}{In tabernáculo suo habitáre facit eam.}
\gscore{0205N3R1}{R}{7}{Beata Agatha}
\gscore{0205N3R2}{R}{8}{Medicinam carnalem}
\tedeumrubric

\feast{0206}{S. Titi Episcopi et Confessoris}
	{Proprium Sanctorum}{Festa Februarii}{2}{6 Februarii}
	{Duplex m.t.v.}{III. classis}{Titi}
	{Commune Confessoris Pontificis, \pageref{M-COPO}.}
	{}

\feast{0207}{S. Romualdi Abbatis}
	{Proprium Sanctorum}{Festa Februarii}{2}{7 Februarii}
	{Duplex m.t.v.}{III. classis}{Romualdi}
	{Commune Abbatum, \pageref{M-COAB}.}
	{}

\feast{0208}{S. Joannis de Matha Confessoris}
	{Proprium Sanctorum}{Festa Februarii}{2}{8 Februarii}
	{Duplex m.t.v.}{III. classis}{Joannis de Matha}
	{Commune Confessoris non Pontificis, \pageref{M-CONP}.}
	{}

\feast{0209}{S. Cyrilli Episcopi Alexandrini\\Confessoris et Ecclesiæ Doctoris}
	{Proprium Sanctorum}{Festa Februarii}{2}{9 Februarii}
	{Duplex m.t.v.}{III. classis}{Cyrilli Alexandrini}
	{Commune Doctorum Ecclesiæ, \pageref{M-CODO}.}
	{}

\feast{0210}{S. Scholasticæ Virginis}
	{Proprium Sanctorum}{Festa Februarii}{2}{10 Februarii}
	{Duplex}{III. classis}{Scholasticæ}
	{Commune Virginum, \pageref{M-MU}.}
	{}

\feast{0211}{In Apparitione Beatæ Mariæ Virginis Immaculatæ}
	{Proprium Sanctorum}{Festa Februarii}{2}{11 Februarii}
	{Duplex majus}{III. classis}{Mariæ!Apparitio Immaculatæ}
	{Psalmi de Communi Festorum B.M.V. \pageref{M-CBMV}, reliqua ut infra.}
	{}
\gscore{1208I}{I}{}{Immaculatam Conceptionem}
\gscore{0211H}{H}{}{Te dicimus praeconio}
\nocturn{1}
\gscore{0211N1A1}{A}{1}{Ave gratia plena dominus tecum}
\gscore{0211N1A2}{A}{2}{Benedicta tu inter mulieres}
\gscore{0211N1A3}{A}{3}{Ne timeas Maria}
\versiculus{Deus omnípotens præcínxit me virtúte.}{Et pósuit immaculátam viam meam.}
\gscore{0211N1R1}{R}{1}{Sapientia quae}
\gscore{0211N1R2}{R}{2}{Quasi arcus}
\gscore{0211N1R3}{R}{3}{Surge amica mea}
\nocturn{2}
\gscore{0211N2A1}{A}{4}{Fecit mihi magna qui potens est}
\gscore{0211N2A2}{A}{5}{Sanctificavit tabernacula}
\gscore{0211N2A3}{A}{6}{Dominus possedit te}
\versiculus{Adjuvábit eam Deus vultu suo.}{Deus in médio ejus, non commovébitur.}
\gscore{0211N2R1}{R}{4}{Quae est ista quae progreditur}
\gscore{0211N2R2}{R}{5}{Erit in novissimis diebus}
\gscore{0211N2R3}{R}{6}{Praevenisti eam Domine}
\nocturn{3}
\gscore{0211N3A1}{A}{7}{Manus Domini confortavit}
\gscore{0211N3A2}{A}{8}{Noli metuere}
\gscore{0211N3A3}{A}{9}{Benedixit te Dominus in virtute}
\versiculus{Diffúsa est grátia in lábiis tuis.}{Proptérea benedíxit te Deus in ætérnum.}
\gscore{0211N3R1}{R}{7}{Tu erga invoca}
\gscore{0211N3R2}{R}{8}{Plantavit Dominus Deus paradisum}
\tedeumrubric

\feast{0212}{Ss. Septem Fundatorum\\Ordinis Servorum B. M. V. Confessorum}
	{Proprium Sanctorum}{Festa Februarii}{2}{12 Februarii}
	{Duplex}{III. classis}{Septem Fundatorum Ordinis Servorum B. M. V.}
	{Commune Confessoris non Pontificis, \pageref{M-CONP}.}
	{}
\gscore{0212H}{H}{}{Bella dum late furerent}

\feast{0214}{S. Valentini Presbyteri et Martyris}
	{Proprium Sanctorum}{Festa Februarii}{2}{14 Februarii}
	{Simplex}{(Commemoratio tantum)}{Valentini}
	{Commune unius Martyris \pageref{M-UMEX}.}
	{}

\feast{0215}{Ss. Faustini et Jovitæ Martyrum}
	{Proprium Sanctorum}{Festa Februarii}{2}{15 Februarii}
	{Simplex}{(Commemoratio tantum)}{Faustini et Jovitæ}
	{Commune plurimorum Martyrum, \pageref{M-PMEX}.}
	{}

\feast{0218}{S. Simeonis Episcopi et Martyris}
	{Proprium Sanctorum}{Festa Februarii}{2}{18 Februarii}
	{Simplex}{(Commemoratio tantum)}{Simeonis}
	{Commune unius Martyris \pageref{M-UMEX}.}
	{}

\feast{0222}{In Cathedra S. Petri Apostoli Antiochiæ}
	{Proprium Sanctorum}{Festa Februarii}{2}{22 Februarii}
	{Duplex majus}{II. classis}{Petri Apostoli!Cathedra Antiochiæ}
	{Omnia ut in Cathedra S.\ Petri Apostoli Romæ, pag.\ \pageref{M-0118}.}
	{}

\feast{0223}{S. Petri Damiani\\Episcopi Confessoris et Ecclesiæ Doctoris}
	{Proprium Sanctorum}{Festa Februarii}{2}{23 Februarii}
	{Duplex}{III. classis}{Petri Damiani}
	{Commune Doctorum Ecclesiæ, \pageref{M-CODO}.}
	{}

\feast{0224b}{In Vigilia S. Matthiæ Apostoli}
	{Proprium Sanctorum}{Festa Februarii}{2}{24 Februarii in anno bissextili extra Quadragesimam}
	{Simplex}{(Omittitur)}{Matthiæ!vigilia}
	{Officium fit de Feria, præter Lectiones, cum Responsoriis de Feria.}
	{}

\feast{0224}{S. Matthiæ Apostoli}
	{Proprium Sanctorum}{Festa Februarii}{2}{24 Februarii aut 25 in anno bissextili}
	{Duplex II. classis}{II. classis}{Matthiæ}
	{Omnia de Communi Apostolorum, \pageref{M-APEX}.}
	{}

\feast{0227}{S. Gabrielis a Virgine Perdolente Confessoris}
	{Proprium Sanctorum}{Festa Februarii}{2}{27 Februarii aut 28 in anno bissextili}
	{Duplex}{III. classis}{Gabrielis a Virgine Perdolente}
	{Commune Confessoris non Pontificis, \pageref{M-CONP}.}
	{}

\end{document}