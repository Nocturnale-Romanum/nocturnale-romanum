% !TEX TS-program = lualatex
% !TEX encoding = UTF-8

\documentclass[nocturnale-romanum.tex]{subfiles}

\ifcsname preamble@file\endcsname
  \setcounter{page}{\getpagerefnumber{M-nr30_commune_apostolorum}}
\fi

\begin{document}
\feast{CS}{Commune Sanctorum}{Commune Sanctorum}{Commune Sanctorum}{1}{}{}{}{}{}{}
\addcontentsline{toc}{chapter}{Commune Sanctorum}

\rubric{
In omnibus Festis novem Lectionum Apostolorum vel  Evangelistarum, in Duplicibus \Rnum{1} et \Rnum{2} classis aliorum Sanctorum, in Dedicatione Ecclesiæ, in Festis beatæ Mariæ Virginis, non autem per eorum Octavas, Officium integrum præter ea quæ suis locis propria assignantur, dicitur de respectivo Communi, ut infra.
In reliquis Sanctorum Festis, in Octavis non privilegiatis quibuslibet et in Officio sanctæ Mariæ in Sabbato, præter ea quæ pariter suis locis assignantur propria, Antiphonæ et Psalmi ad omnes Horas et Versus Nocturnorum sumuntur de occurenti hebdomadæ die; Lectiones \Rnum{1} Nocturni aut Lectio \rnum{1} et \rnum{2} cum suis Responsoriis dicuntur de Scriptura occurenti, juxta Rubricas; reliqua omnia de respectivo Communi, ut infra; in Communi autem Confessorum, Virginum et non Virginum, dedicationis Ecclesiæ et Festorum beatæ Mariæ Virginis, necnon in Officio ejusdem in Sabbato, omnia dicuntur ut extra Tempus Paschale, sed additur unum \normaltext{Allelúja} Invitatorio, Antiphonis, versibus eorumque Responsoriis, necnon et responsoriis post Lectiones Nocturnorum ante Versum; Psalmi dicuntur in quolibet Nocturno sub prima Antiphona, ut infra suis locis ponitur.}

\feast{VIAP}{In Vigiliis Apostolorum}
	{Commune Sanctorum}{In Vigiliis Apostolorum}{2}{}
	{}{}{}
	{}{}
\rubric{Officium fit de Feria, ut in Ordinario et Psalterio, præter Lectiones et Orationem, quæ, nisi propria suis locis assignentur, cum Responsoriis tamen de Feria currenti, ut in proprio de Tempore.
Ad Nocturnum vero in Feria \Rnum{4} tres ultimæ Antiphonæ cum suis Psalmis sumuntur.}

\feast{APEX}{Commune Apostolorum\\extra Tempus Paschale}
	{Commune Sanctorum}{Commune Apostolorum extra T. P.}{2}{}
	{}{}{}
	{}
	{}
\addcontentsline{toc}{section}{Commune Apostolorum}
\rubric{In festis}
\gscore{APEXIa}{I}{}{Regem apostolorum (tonus simplex)}
\rubric{In festis solemnibus}
\gscore{APEXIb}{I}{}{Regem apostolorum (tonus solemnis)}
\rubric{Ad libitum aliis ejusdem metri melodiis Hymnus cantatur nisi modus specialis juxta rubricas cantaretur.}
\gscore{APEXH}{H}{}{Aeterna Christi munera}
\nocturn{1}
\gscore{APEXN1A1}{A}{1}{In omnem terram exivit sonus}
\psalmus{18}{}
\gscore{APEXN1A2}{A}{2}{Clamaverunt justi et Dominus}
\psalmus{33}{}
\gscore{APEXN1A3}{A}{3}{Constitues eos principes}
\psalmus{44}{}
\versiculus{In omnem terram exívit sonus eórum.}{Et in fines orbis terræ verba eórum.}
\gscore{APEXN1R1}{R}{1}{Ecce ego mitto vos sicut}
\gscore{APEXN1R2}{R}{2}{Tollite jugum meum super vos}
\gscore{APEXN1R3}{R}{3}{Dum steteritis ante reges}
\nocturn{2}
\gscore{APEXN2A1}{A}{4}{Principes populorum congregati sunt}
\psalmus{46}{}
\gscore{APEXN2A2}{A}{5}{Dedisti hereditatem timentibus}
\psalmus{60}{}
\gscore{APEXN2A3}{A}{6}{Adnuntiaverunt opera Dei}
\psalmus{63}{}
\versiculus{Constítues eos príncipes super omnem terram.}{Mémores erunt nóminis tui, Domine.}
\gscore{APEXN2R1}{R}{4}{Vidi conjunctos viros}
\gscore{APEXN2R2}{R}{5}{Beati eritis cum maledixerint}
\gscore{APEXN2R3}{R}{6}{Isti sunt triumphatores}
\nocturn{3}
\gscore{APEXN3A1}{A}{7}{Exaltabuntur cornua justi}
\psalmus{74}{}
\rubric{Ante Septuagesimam:}
\gscore{APEXN3A2}{A}{8}{Lux orta \emph{cum} Alleluia}
\rubric{Post Septuagesimam:}
\gscore{0806N3A2}{A}{8}{Lux orta \emph{sine} Alleluia}
\psalmus{96}{}
\rubric{Ante Septuagesimam:}
\gscore{APEXN3A3}{A}{9}{Custodiebant \emph{cum} Alleluia}
\rubric{Post Septuagesimam:}
\gscore{APEXN3A3b}{A}{9}{Custodiebant \emph{sine} Alleluia}
\psalmusref{98}{}
\versiculus{Nimis honoráti sunt amíci tui, Dómine.}{Nimis confortátus est principátus eórum.}
\gscore{APEXN3R1}{R}{7}{Isti sunt qui viventes}
\gscore{APEXN3R2}{R}{8}{Isti sunt viri sancti}
\tedeumrubric

\feast{EVEX}{Commune Evangelistarum\\extra Tempus Paschale}
	{Commune Sanctorum}{Commune Evangelistarum extra T. P.}{2}{}
	{}{}{}
	{}
	{}
\rubric{Omnia ut in Communi Apostolorum extra Tempus Paschale.}

\feast{APTP}{Commune Apostolorum\\et Evangelistarum\\Tempore Paschali}
	{Commune Sanctorum}{Commune Apostolorum et Evangelistarum T. P.}{2}{}
	{}{}{}
	{}
	{}
\rubric{Sub una Antiphona in quolibet Nocturno dicuntur Psalmi ut supra extra Tempus Paschale, pag.\ \pageref{M-APEXN1A1}.  Lectiones trium Nocturnorum, ut extra Tempus Paschale, cum Responsoriis sequentibus.}
\gscore{APTPI}{I}{}{Regem apostolorum Dominum venite... alleluia}
\gscore{APTPH}{H}{}{Tristes erant Apostoli}
\rubric{Ab Ascensione ad Pentecosten, non dicitur \normaltext{Qui surrexísti a mórtuis} sed \normaltext{Qui scandis super sídera}.}
\nocturn{1}
\gscore{APTPN1A}{A}{1}{Stabunt justi in magna constantia}
\versiculus{Sancti et justi, in Dómino gaudéte, allelúja.}{Vos elégit Deus in hereditátem sibi, allelúja.}
\gscore{APTPN1R1}{R}{1}{Beatus vir qui metuit Dominum}
\gscore{APTPN1R2}{R}{2}{Tristitia vestra}
\gscore{APTPN1R3}{R}{3}{Pretiosa in conspectu}
\nocturn{2}
\gscore{APTPN2A}{A}{2}{Ecce quomodo computati sunt}
\versiculus{Lux perpétua lucébit Sanctis tuis, Dómine, allelúja.}{Et ætérnitas témporum, allelúja.}
\gscore{APTPN2R1}{R}{4}{Lux perpetua lucebit}
\gscore{APTPN2R2}{R}{5}{Virtute magna}
\rubric{Feria \Rnum{2} et \Rnum{4} infra hebdomadam \Rnum{1} et \Rnum{2} post Octavam Paschæ, quoties in \Rnum{1} Nocturno Lectiones fuerint de Scriptura occurenti cum suis Responsoriis de Tempore, loco præcedentis Responsorii de Tempore dicitur \rrrub \scorename{APTPN1R3}, ut supra pag.\pageref{M-APTPN1R3}, sed sine \normaltext{Glória Patri}.}
\gscore{APTPN2R3}{R}{6}{Isti sunt agni novelli}
\nocturn{3}
\gscore{APTPN3A}{A}{3}{Lux perpetua lucebit Sanctis tuis}
\versiculus{Lætítia sempitérna super cápita eórum, allelúja.}{Gáudeium et exsultatiónem obtinébunt, allelúja.}
\gscore{APTPN3R1}{R}{7}{Ego sum vitis vera}
\rubric{Feria \Rnum{3} et \Rnum{6} infra hebdomadam \Rnum{1} et \Rnum{2} post Octavam Paschæ, quoties in \Rnum{1} Nocturno Lectiones fuerint de Scriptura occurenti cum suis Responsoriis de Tempore, loco præcedentis Responsorii de Tempore dicitur \rrrub \scorename{APTPN1R2}, ut supra pag.\ \pageref{M-APTPN1R2}.}
\gscore{APTPN3R2}{R}{8}{Candidi facti sunt}
\tedeumrubric

\end{document}