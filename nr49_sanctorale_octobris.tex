% !TEX TS-program = lualatex
% !TEX encoding = UTF-8

\documentclass[nocturnale-romanum.tex]{subfiles}

\ifcsname preamble@file\endcsname
  \setcounter{page}{\getpagerefnumber{M-nr49_sanctorale_octobris}}
\fi

\begin{document}

\feast{1000}{Festa Octobris}{Proprium Sanctorum}{Festa Octobris}{1}{}{}{}{}{}{}

\feast{1001}{S. Remigii Episcopi et Confessoris}
	{Proprium Sanctorum}{Festa Octobris}{2}{1 Octobris}
	{Simplex m.t.v.}{(Commemoratio tantum)}{Remigii}
	{Commune Confessoris Pontificis, pag.\ \pageref{M-COPO}.}
	{}

\feast{1002}{Ss. Angelorum Custodum}
	{Proprium Sanctorum}{Festa Octobris}{2}{2 Octobris}
	{Duplex majus}{III. classis}{Angelorum Custodum}
	{Psalmi trium Nocturnorum ut in Festo Gabrielis Archangeli, pag.\ \pageref{M-0324}.}
	{}
\gscore{1002I}{I}{}{Regem Angelorum Dominum Venite adoremus}
\gscore{1002H}{H}{}{Custodes hominum psallimus Angelos}
\nocturn{1}
\gscore{1002N1A1}{A}{1}{Dominus Deus coeli et terrae ipse mittet}
\gscore{1002N1A2}{A}{2}{Deus meus misit Angelum suum}
\gscore{1002N1A3}{A}{3}{Bene ambuletis et Dominus sit in itinere}
\versiculus{Stetit Angelus juxta aram templi.}{Habens thuribulum aúreum in manu sua.}
\gscore{1002N1R1}{R}{1}{Angelis suis mandavit de te}
\gscore{1002N1R2}{R}{2}{Respondit Angelus Domini}
\gscore{1002N1R3}{R}{3}{In conspectu Gentium}
\nocturn{2}
\gscore{1002N2A1}{A}{4}{Cum essem vobiscum per voluntatem}
\gscore{1002N2A2}{A}{5}{Tollens se Angelus Domini}
\gscore{1002N2A3}{A}{6}{Imittet Angelus Domini in circuitu}
\versiculus{Ascéndit fumus aromátum in conspéctu Dómini.}{De manu Angeli.}
\gscore{1002N2R1}{R}{4}{Vivit ipse Dominus quoniam}
\gscore{1002N2R2}{R}{5}{Angelus Domini descendit cum Azaria}
\gscore{1002N2R3}{R}{6}{In omni tribulatione eorum}
\nocturn{3}
\gscore{1002N3A1}{A}{7}{Misit Dominus Angelum suum qui percussit}
\gscore{1002N3A2}{A}{8}{Adorate Dominum omnes angeli}
\gscore{1002N3A3}{A}{9}{Benedicite Domino omnes angeli}
\versiculus{In conspéctu Angelórum psallam tibi, Deus meus.}{Adorábo ad templum sanctum tuum, et confitébor nómini tuo.}
\gscore{1002N3R1}{R}{7}{Machabaeus et qui cum eo}
\gscore{1002N3R2}{R}{8}{Tu Domine qui misisti Angelum}

\feast{1003}{S. Teresiæ a Jesu Infante Virginis}
	{Proprium Sanctorum}{Festa Octobris}{2}{3 Octobris}
	{Duplex}{III. classis}{Teresiæ a Jesu Infante}
	{Commune aut Virginum aut non Virginum, pag.\ \pageref{M-MU}.}
	{}

\feast{1004}{S. Francisci Confessoris}
	{Proprium Sanctorum}{Festa Octobris}{2}{4 Octobris}
	{Duplex majus}{III. classis}{Francisci}
	{Commune Confessoris non Pontificis, pag.\ \pageref{M-CONP}.}
	{}

\feast{1005}{Ss. Placidi et Sociorum Martyrum}
	{Proprium Sanctorum}{Festa Octobris}{2}{5 Octobris}
	{Simplex}{(Commemoratio tantum)}{Placidi}
	{Commune plurimorum Martyrum, pag.\ \pageref{M-PMEX}.}
	{}

\feast{1006}{S. Brunonis Confessoris}
	{Proprium Sanctorum}{Festa Octobris}{2}{6 Octobris}
	{Duplex}{III. classis}{Brunonis}
	{Commune Confessoris non Pontificis, pag.\ \pageref{M-CONP}.}
	{}

\feast{1007}{In Festo Sacratissimi Rosarii\\Beatæ Mariæ Virginis}
	{Proprium Sanctorum}{Festa Octobris}{2}{7 Octobris}
	{Duplex II. classis}{II. classis}{Mariæ!Rosarium}
	{Omnia de Communi Festorum B.M.V.\ pag.\ \pageref{M-CBMV}, præter ea quæ hic habentur propria.}
	{}
\gscore{1007I}{I}{}{Solemnitatem Rosarii}
\gscore{1007H}{H}{}{In monte olivis consito Redemptor}
\nocturn{1}
\gscore{1007N1A1}{A}{1}{Angelus Gabriel nuntiavit Mariae}
\gscore{1007N1A2}{A}{2}{Intravit Maria in domum}
\gscore{1007N1A3}{A}{3}{Peperit filium suum primogenitum}
\versiculus{Sancta Dei Génitrix, semper Virgo María.}{Intercéde pro nobis ad Dóminum, Deum nostrum.}
\gscore{1007N1R1}{R}{1}{Sumite psalterium}
\gscore{1007N1R2}{R}{2}{Vidi speciosam sicut}
\resprefcumgp{3}{CBMVN2R2}
\nocturn{2}
\gscore{1007N2A1}{A}{4}{Cum inducerent Jesum}
\gscore{1007N2A2}{A}{5}{Requirentes Jesum parentes}
\gscore{1007N2A3}{A}{6}{Coepit constristari et factus}
\versiculus{Post partum, Virgo, invioláta permansísti.}{Dei Génitrix intercéde pro nobis.}
\gscore{1007N2R1}{R}{4}{Tu gloria Jerusalem}
\gscore{1007N2R2}{R}{5}{Dextera tua magnificata est}
\respref{6}{1208N2R3}{}
\nocturn{3}
\gscore{1007N3A1}{A}{7}{Apprehendit Pilatus Jesum}
\gscore{1007N3A2}{A}{8}{Milites plectentes coronam}
\gscore{1007N3A3}{A}{9}{Crucis imperium super humerum}
\versiculus{Speciósa facta es et suávis.}{In delíciis tuis, sancta Dei Génitrix.}
\gscore{1007N3R1}{R}{7}{Ego quasi vitis}
\gscore{1007N3R2}{R}{8}{Surge propera amica mea jam enim}
\tedeumrubric

\feast{1008}{S. Birgittæ Viduæ}
	{Proprium Sanctorum}{Festa Octobris}{2}{8 Octobris}
	{Duplex}{III. classis}{Birgittæ}
	{Commune aut Virginum aut non Virginum, pag.\ \pageref{M-MU}.}
	{}

\feast{1009}{S. Joannis Leonardi Confessoris}
	{Proprium Sanctorum}{Festa Octobris}{2}{9 Octobris}
	{Duplex}{III. classis}{Joannis Leonardi}
	{Commune Confessoris non Pont., pag.\ \pageref{M-CONP}.}
	{}

\feast{1010}{S. Francisci Borgiæ Confessoris}
	{Proprium Sanctorum}{Festa Octobris}{2}{10 Octobris}
	{Semiduplex m.t.v.}{III. classis}{Francisci Borgiæ}
	{Commune Confessoris non Pont., pag.\ \pageref{M-CONP}.}
	{}

\feast{1011}{In Maternitate Beatæ Mariæ Virginis}
	{Proprium Sanctorum}{Festa Octobris}{2}{11 Octobris}
	{Duplex II. classis}{II. classis}{Mariæ|Maternitas}
	{Omnia sicut Communi Festorum B.M.V.\ pag.\ \pageref{M-CBMV}, præter ea quæ hic habentur propria.}
	{}
\gscore{1011I}{I}{}{Maternitatem beatae Virginis Mariae celebremus}
\gscore{1011H}{H}{}{Coelo Redemptor praetulit}
\nocturn{1}
\gscore{1011N1R1}{R}{1}{Felix es sacra Virgo V Maternitatem}
\gscore{1011N1R2}{R}{2}{Sine tactu pudoris V Benedicta tu}
\gscore{1011N1R3}{R}{3}{Multae filiae congregaverunt V Sentiant omnes tuum V Gloria}
\nocturn{2}
\gscore{1011N2R1}{R}{4}{Gloriosae... Maternitatem V Christo canamus gloriam}
\gscore{1011N2R2}{R}{5}{Benedicta filia tua V Nostras deprecationes}
\gscore{1011N2R3}{R}{6}{Benedicta tu inter mulieres... unde hoc mihi V Respexit V Gloria}
\nocturn{3}
\gscore{1011N3R1}{R}{7}{Beata es Virgo Maria quae V Diffusa est}
\resprefcumgp{8}{CBMVN1R2}
\tedeumrubric

\feast{1013}{S. Eduardi Regis Confessoris}
	{Proprium Sanctorum}{Festa Octobris}{2}{13 Octobris}
	{Semiduplex m.t.v.}{III. classis}{Eduardi}
	{Commune Confessoris non Pont., pag.\ \pageref{M-CONP}.}
	{}

\feast{1014}{S. Callisti Papæ et Martyris}
	{Proprium Sanctorum}{Festa Octobris}{2}{14 Octobris}
	{Duplex}{III. classis}{Callisti}
	{Commune unius Martyris, pag.\ \pageref{M-UMEX}.}
	{}

\feast{1015}{S. Teresiæ Virginis}
	{Proprium Sanctorum}{Festa Octobris}{2}{15 Octobris}
	{Duplex}{III. classis}{Teresiæ}
	{Commune aut Virginum aut non Virginum, pag.\ \pageref{M-MU}, præter Hymnus.}
	{}
\gscore{1015H}{H}{}{Regis superni nuntia}

\feast{1016}{S. Hedwigis Viduæ}
	{Proprium Sanctorum}{Festa Octobris}{2}{16 Octobris}
	{Semiduplex}{III. classis}{Hedwigis}
	{Commune aut Virginum aut non Virginum, pag.\ \pageref{M-MU}.}
	{}

\feast{1017}{S. Margaritæ Mariæ Alacoque Virginis}
	{Proprium Sanctorum}{Festa Octobris}{2}{17 Octobris}
	{Duplex}{III. classis}{Margaritæ Mariæ Alacoque}
	{Commune aut Virginum aut non Virginum, pag.\ \pageref{M-MU}.}
	{}

\feast{1018}{S. Lucæ Evangelistæ}
	{Proprium Sanctorum}{Festa Octobris}{2}{18 Octobris}
	{Duplex II. classis}{II. classis}{Lucæ}
	{Omnia de Communi Apostolorum, pag.\ \pageref{M-APEX}.}
	{}

\feast{1019}{S. Petri de Alcantara Confessoris}
	{Proprium Sanctorum}{Festa Octobris}{2}{19 Octobris}
	{Duplex m.t.v.}{III. classis}{Petri de Alcantara}
	{Commune Confessoris non Pont., pag.\ \pageref{M-CONP}.}
	{}

\feast{1020}{S. Joannis Cantii Confessoris}
	{Proprium Sanctorum}{Festa Octobris}{2}{20 Octobris}
	{Duplex}{III. classis}{Joannis Cantii}
	{Commune Confessoris non Pont., pag.\ \pageref{M-CONP}, præter Hymnus.}
	{}
\gscore{1020H}{H}{}{Corpus domas jejuniis}

\feast{1021}{S. Hilarionis Abbatis}
	{Proprium Sanctorum}{Festa Octobris}{2}{21 Octobris}
	{Simplex}{(Commemoratio tantum)}{Hilarionis}
	{Commune Abbatum, pag.\ \pageref{M-COAB}.}
	{}

\feast{1023}{S. Antonii Mariæ Claret Episcopi et Confessoris}
	{Proprium Sanctorum}{Festa Octobris}{2}{23 Octobris}
	{Duplex}{III. classis}{Antonii Mariæ Claret}
	{Commune Confessoris Pont., pag.\ \pageref{M-COPO}.}
	{}

\feast{1024}{S. Raphaëlis Archangeli}
	{Proprium Sanctorum}{Festa Octobris}{2}{24 Octobris}
	{Duplex majus}{III. classis}{Raphaëlis}
	{Invitatorium \scorename{0324I} ut pag.\ \pageref{M-0324I}, Hymnus ut infra. Psalmi trium Nocturnorum ut in Festo S. Gabrielis Archangeli pag.\ \pageref{M-0324}.}
	{}
\gscore{1024H}{H}{}{Christe sanctorum decus Angelorum}
\nocturn{1}
\gscore{1024N1A1}{A}{1}{Egressus Tobias invenit}
\gscore{1024N1A2}{A}{2}{Angelus Raphael seipsum}
\gscore{1024N1A3}{A}{3}{Sanum ducam filium}
\versiculus{Data sunt Angelo incénsa multa.}{Ut adoléret ea ante altáre áureum, quod est ante óculos Dómini.}
\gscore{1024N1R1}{R}{1}{In illo tempore exauditae}
\gscore{1024N1R2}{R}{2}{Egressus Tobias invenit}
\gscore{1024N1R3}{R}{3}{Ingressus Angelus ad Tobiam}
\nocturn{2}
\gscore{1024N2A1}{A}{4}{Dixit autem Angelus apprehende}
\gscore{1024N2A2}{A}{5}{Obsecro te Azaria frater}
\gscore{1024N2A3}{A}{6}{Lumina fel sanat sed virtus}
\versiculus{Ascéndit fumus arómatum in conspéctu Dómini.}{De manu Angeli.}
\gscore{1024N2R1}{R}{4}{Interrogavit Tobias Angelum}
\gscore{1024N2R2}{R}{5}{Exivit Tobias ut lavaret}
\gscore{1024N2R3}{R}{6}{Ubi introieris domum tuam}
\nocturn{3}
\gscore{1024N3A1}{A}{7}{Est hic Sara Raguelis}
\gscore{1024N3A2}{A}{8}{Septem viros habuit quos}
\gscore{1024N3A3}{A}{9}{Per tres dies orationi cum uxore}
\versiculus{Apprehéndit Angelus Ráphaël dæmónium.}{Et religávit illud in desérto superióris Ægýpti.}
\gscore{1024N3R1}{R}{7}{Benedicite Deum coeli dixit Angelus}
\gscore{1024N3R2}{R}{8}{Tempus est ut revertar ad eum}
\tedeumrubric

\feast{1025}{Ss. Chrysanthi et Dariæ Martyrum}
	{Proprium Sanctorum}{Festa Octobris}{2}{25 Octobris}
	{Simplex}{(Commemoratio tantum)}{Chrysanthi et Dariæ}
	{Commune plurimorum Martyrum., pag.\ \pageref{M-PMEX}.}
	{}

\feast{1026}{S. Evaristi Papæ et Martyris}
	{Proprium Sanctorum}{Festa Octobris}{2}{26 Octobris}
	{Simplex}{(Commemoratio tantum)}{Evaristi}
	{Commune unius Martyris, pag.\ \pageref{M-UMEX}. Si hodie fuerit Sabbatum, fit Vigilia anticipata Ss.\ Simonis et Judæ Apostolorum, ut die sequente notatur, et de S.\ Evaristo it tantum Commemoratio in Vesperis Feriæ præ cedentis et ad Laudes.}
	{}

\feast{1027}{In Vigilia Ss. Simonis et Judæ}
	{Proprium Sanctorum}{Festa Octobris}{2}{27 Octobris}
	{Simplex}{(Omittitur)}{Simonis et Judæ!Vigilia}
	{Officium fit de Feria, præter Lectiones, cum Responsoriis de Feria.}
	{}

\feast{1028}{Ss. Simonis et Judæ Apostolorum}
	{Proprium Sanctorum}{Festa Octobris}{2}{28 Octobris}
	{Duplex II. classis}{II. classis}{Simonis et Judæ}
	{Omnia de Communi Apostolorum, pag.\ \pageref{M-APEX}.}
	{}

\feast{1031}{In Vigilia Omnium Sanctrum}
	{Proprium Sanctorum}{Festa Octobris}{2}{31 Octobris aut 30 si 31 fuerit Dominica}
	{Simplex}{(Omittitur)}{Omnium Sanctrum!Vigilia}
	{Officium fit de Feria, ut in Ordinario et Psalterio, et Lectiones, quæ dicuntur de Homilia in Evangelium \emph{Descéndens Jesus} ut in Communi plurimorum Martyrum 2 loco, cum Repsonsoriis tamen de Feria currenti, ut in Proprio de Tempore. Ad Nocturnum ero in Feria \Rnum{4} tres ultimæ Antiphonæ sum suis Psalmis, et ad Laudes in qualibet Feria Antiphonæ omnes et Psalmi sumuntur de 2 loco; ad Primam additur quartus Psalmus, ut in Psalterio notatur, et ad omnes Horas dicuntur Preces feriales, ut in Ordinario.}
	{}

\feast{10H5F1}{In Festo Domini Nostri Jesu Christi Regis}
	{Proprium Sanctorum}{Festa Octobris}{2}{Dominica ultima Octobris}
	{Duplex I. classis}{I. classis}{Jesu Christi, Domini nostri!Rex}
	{}
	{}
\gscore{10H5I}{I}{}{Jesum Christum Regem regum}
\gscore{10H5H}{H}{}{Aeterna imago altissimi}
\nocturn{1}
\gscore{10H5N1A1}{A}{1}{Ego autem constitutus}
\psalmus{2}{3}
\gscore{10H5N1A2}{A}{2}{Gloria et honore}
\psalmus{8}{1}
\gscore{10H5N1A3}{A}{3}{Elevamini portae aeternales}
\psalmus{23}{5}
\versiculus{Data est mihi omnis potéstas.}{In cœlo et in terra.}
\gscore{10H5N1R1}{R}{1}{Super Solium David}
\gscore{10H5N1R2}{R}{2}{Aspiciebam in visu}
\gscore{10H5N1R3}{R}{3}{Tu Bethlehem}
\nocturn{2}
\gscore{10H5N2A1}{A}{4}{Sedebit Dominus Rex}
\psalmus{28}{7}
\gscore{10H5N2A2}{A}{5}{Virga directationis}
\psalmus{44}{8}
\gscore{10H5N2A3}{A}{6}{Psallite Rege nostro}
\psalmus{46}{1}
\versiculus{Afférte Dómino, famíliæ populórum.}{Afférte Dómino glóriam et impérium.}
\gscore{10H5N2R1}{R}{4}{Exsulta satis}
\gscore{10H5N2R2}{R}{5}{Opportet illum}
\gscore{10H5N2R3}{R}{6}{Fecit nos regnum}
\nocturn{3}
\gscore{10H5N3A1}{A}{7}{Benedicentur in ipso}
\psalmus{71}{4}
\gscore{10H5N3A2}{A}{8}{Et ego primogenitum}
\psalmus{88j}{2}
\gscore{10H5N3A3}{A}{9}{Thronus ejus}
\psalmus{88ij}{6}
\versiculus{Adorábunt eum omnes reges terræ.}{Omnes gentes sérvient ei.}
\gscore{10H5N3R1}{R}{7}{Factum est regnum huius mundi}
\gscore{10H5N3R2}{R}{8}{Decem cornua quae vidisti}
\tedeumrubric

\end{document}